% !TEX root = ../HTNotes-Demo.tex
% !TEX program = xelatex
% 内容开始
\type{第一类曲线积分}
  \Example{%
    2012-2013-2-期末-填空题-3
  }{%
    设曲线$L$是椭圆$\dfrac{{{x}^{2}}}{4}+\dfrac{{{y}^{2}}}{3}=1$,其周长为$a$,则$\oint_{L}{\left( 3x{{y}^{3}}+3{{x}^{2}}+4{{y}^{2}} \right)\rd s=}$
    \fillin{$12a$}.
  }{%
    12$a$.
  }{%
    环积分满足在曲线$L$上恒有$3{{x}^{2}}+4{{y}^{2}}=12$,且$x{{y}^{3}}$关于$y$奇对称,于是
    \begin{align*}
      & \oint_{L}{\left( 3x{{y}^{3}}+3{{x}^{2}}+4{{y}^{2}} \right)\rd s}
      = 3\oint_{L}{x{{y}^{3}}\rd s}+\oint_{L}{\left( 3{{x}^{2}}+4{{y}^{2}} \right)\rd s}
      = 3\oint_{L}{x{{y}^{3}}\rd s}+12a
      = 12a.
    \end{align*}
  }

  \Example{%
    2013-2014-2-期末-选择题-9
  }{%
    设$L$是抛物线$y={{x}^{2}}$上点$O\left( 0,0 \right)$与点$B\left( 1,1 \right)$的一段弧,则$\displaystyle\int_{L}{\sqrt{y}}\rd s=$
    \pickout{A}
    \options{$\dfrac{1}{12}\left( 5\sqrt{5}-1 \right)$}
      {$\dfrac{1}{12}\left( 5\sqrt{5}-2 \right)$}
      {$\dfrac{1}{12}\left( 5\sqrt{5}-3 \right)$}
      {$\dfrac{1}{12}\left( 5\sqrt{5}-4 \right)$}
  }{%
    A.
  }{%
    先画图,便于理解.
    $y={{x}^{2}}$$\Rightarrow x=\sqrt{y}$,$\rd s=\sqrt{1+{{\left( \frac{ \rd y }{ \rd x } \right)}^{2}}} \rd x =\sqrt{1+4{{x}^{2}}} \rd x $

    则原式$=\displaystyle\int_{0}^{1}{x}\sqrt{1+4{{x}^{2}}} \rd x =\dfrac{1}{8}\displaystyle\int_{0}^{1}{\sqrt{1+4{{x}^{2}}}}\rd \left( 1+4{{x}^{2}} \right)=\dfrac{1}{8}\times \dfrac{2}{3}{{\left( 1+4{{x}^{2}} \right)}^{\frac{3}{2}}}\Big|_0^1=\dfrac{1}{12}\left( 5\sqrt{5}-1 \right)$.
  }

\type{第二类曲线积分}
  \Example{%
    2012-2013-2-期末-证明题-18
  }{%
    设$L$为光滑弧段,其弧长为$l$,函数$P\left( x,y,z \right),Q\left( x,y,z \right),R\left( x,y,z \right)$在曲线$L$上连续,证明:
    \[ \left| \displaystyle\int_{L}{P \rd x +Q \rd y +R \rd z } \right|\le lM \]
    其中$M=\max\limits_{\left( x,y,z \right)\in L} \sqrt{{{P}^{2}}+{{Q}^{2}}+{{R}^{2}}}$.
  }{%
    见解析.
  }{%
    设光滑弧段$L$在任意点处的方向余弦为$\cos \alpha ,\cos \beta ,\cos \gamma $

    则$ \rd x =\cos \alpha  \rd s , \rd y =\cos \beta  \rd s , \rd z =\cos \gamma  \rd s ,$且${{\cos }^{2}}\alpha +{{\cos }^{2}}\beta +{{\cos }^{2}}\gamma =1$,于是
    \begin{align*}
      & \displaystyle\int_{L}{P \rd x +Q \rd y +R \rd z }=\displaystyle\int_{L}{\left( P\cos \alpha +Q\cos \beta +R\cos \gamma  \right)\rd s} \\
      = & \displaystyle\int_{L}{\left( P,Q,R \right)}\cdot \left( \cos \alpha ,\cos \beta ,\cos \lambda  \right)\rd s
      = \displaystyle\int_{L}{\sqrt{{{P}^{2}}+{{Q}^{2}}+{{R}^{2}}}\sqrt{{{\cos }^{2}}\alpha +{{\cos }^{2}}\beta +{{\cos }^{2}}\gamma }}\cos \theta \rd s \\
      = & \displaystyle\int_{L}{\sqrt{{{P}^{2}}+{{Q}^{2}}+{{R}^{2}}}}\cos \theta \rd s
    \end{align*}
    其中$\theta $为向量$\left( P,Q,R \right)$与$\left( \cos \alpha ,\cos \beta ,\cos \gamma  \right)$的夹角,故
    \begin{align*}
      \left| \displaystyle\int_{L}{P \rd x +Q \rd y +R \rd z } \right|
      & =\left| \displaystyle\int_{L}{\sqrt{{{P}^{2}}+{{Q}^{2}}+{{R}^{2}}}}\cos \theta  \rd s  \right|\le \displaystyle\int_{L}{\sqrt{{{P}^{2}}+{{Q}^{2}}+{{R}^{2}}}}\left| \cos \theta  \right| \rd s  \\
      & \le M\displaystyle\int_{L}{ \rd s }=lM.
    \end{align*}
  }

  \Example{%
    2013-2014-2-期末-选择题-10
  }{%
    设$L$是抛物线${{y}^{2}}=x$上从点$A\left( 1,-1 \right)$到点$B\left( 1,1 \right)$的一段弧,则$\displaystyle\int_{L}{xy} \rd x =$
    \pickout{D}
    \options{$\dfrac{1}{5}$}
      {$\dfrac{2}{5}$}
      {$\dfrac{3}{5}$}
      {$\dfrac{4}{5}$}
  }{%
    D.
  }{%
    先画图,便于理解.
      $$\displaystyle\int_{L}{xy} \rd x =\displaystyle\int_{1}^{0}{-x\sqrt{x}} \rd x +\displaystyle\int_{0}^{1}{x\sqrt{x}} \rd x =2\displaystyle\int_{0}^{1}{x\sqrt{x}} \rd x =\dfrac{4}{5}.$$
  }

  \Example{%
    2013-2014-2-期末-解答题-15
  }{%
    计算曲线积分$\displaystyle\int_{\left( 1,2 \right)}^{\left( 3,4 \right)}{\left( 6x{{y}^{2}}-{{y}^{3}} \right)} \rd x +\left( 6{{x}^{2}}y-3x{{y}^{2}} \right) \rd y .$
  }{%
    见解析.
  }{%
    $P=6x{{y}^{2}}-{{y}^{3}},Q=6{{x}^{2}}y-3x{{y}^{2}}$
    则$\dfrac{\partial P}{\partial y}=12xy-2{{y}^{2}},\dfrac{\partial P}{\partial x}=12xy-2{{y}^{2}},$二式相等.

    所以曲线积分与路径无关,则可以通过线段路径:$A\left( 1,2 \right)\to C\left( 3,2 \right)\to B\left( 3,4 \right)$,
    
    由$A$点到$C$点,$y=2, \rd y =0.$ 原积分一部分$=\displaystyle\int_{1}^{3}{\left( 6x\cdot {{2}^{2}}-{{2}^{3}} \right)} \rd x =80$
    
    由$C$点到$B$点,$x=3, \rd x =0.$ 原积分另一部分$\displaystyle\int_{2}^{4}{\left( 6\cdot {{3}^{2}}y-3\cdot 3{{y}^{2}} \right) \rd y }=156$
    则原式$=80+156=236$.
  }

  \Example{%
    2015-2016-2-期末模拟-选择题-7
  }{%
    设$P\left( x,y \right),Q\left( x,y \right)$在单连通域$G$内具有一阶连续导数,$P\left( x,y \right) \rd x +Q\left( x,y \right) \rd y $在$G$内为某一函数$U\left( x,y \right)$的全微分,计算$U\left( x,y \right)$的公式是
    \pickout{C}
    \options{$U\left( x,y \right)=\displaystyle\int_{{{x}_{0}}}^{x}{P\left( x,{{y}_{0}} \right) \rd x }+\displaystyle\int_{{{y}_{0}}}^{y}{Q\left( {{x}_{0}},y \right) \rd y }$}
      {$U\left( x,y \right)=\displaystyle\int_{{{x}_{0}}}^{x}{Q\left( x,{{y}_{0}} \right) \rd x }+\displaystyle\int_{{{y}_{0}}}^{y}{P\left( {{x}_{0}},y \right) \rd y }$}
      {$U\left( x,y \right)=\displaystyle\int_{{{x}_{0}}}^{x}{P\left( x,y \right) \rd x }+\displaystyle\int_{{{y}_{0}}}^{y}{Q\left( {{x}_{0}},y \right) \rd x }$}
      {$U\left( x,y \right)=\displaystyle\int_{{{x}_{0}}}^{x}{P\left( x,y \right) \rd x }+\displaystyle\int_{{{y}_{0}}}^{y}{Q\left( x,y \right) \rd y }$}
  }{%
    C.
  }{%
    沿$\left( {{x}_{0}},{{y}_{0}} \right)\to \left( {{x}_{0}},y \right)\to \left( x,y \right)$路径积分.
  }

  \Example{%
    2016-2017-2-期末模拟-计算题-20
  }{%
    设$L$是平面$x+y+z=2$与柱面$\left| x \right|+\left| y \right|=1$的交线,从$z$轴正向看过去,$L$为逆时针方向,计算
      $$I=\oint_{L}{\left( {{y}^{2}}-{{z}^{2}} \right) \rd x +\left( 2{{z}^{2}}-{{x}^{2}} \right)} \rd y +\left( 3{{x}^{2}}-{{y}^{2}} \right) \rd z .$$
  }{%
    见解析.
  }{%
    设交线所围成的曲面为$\Sigma $,则$\Sigma :\left\{ \left( x,y,z \right)|x+y+z=2,\left| x \right|+\left| y \right|=1 \right\}$,取其单位法向量为
      $$\overrightarrow{n}=\dfrac{1}{\sqrt{1+{{\left( \frac{\partial \varphi }{\partial x} \right)}^{2}}+{{\left( \frac{\partial \varphi }{\partial y} \right)}^{2}}}}\left( -\dfrac{\partial \varphi }{\partial x},-\dfrac{\partial \varphi }{\partial y},1 \right)=\left( \cos \alpha ,\cos \beta ,\cos \gamma  \right)$$
    其中$x=\varphi \left( y,z \right),y=\varphi \left( z,x \right),z=\varphi \left( x,y \right)$. 由$Stokes$公式,有
      $$I=\displaystyle\iint\limits_{\Sigma }{\begin{vmatrix}
      \cos \alpha  & \cos \beta  & \cos \gamma   \\
      \dfrac{\partial }{\partial x} & \dfrac{\partial }{\partial y} & \dfrac{\partial }{\partial z}  \\
      {{y}^{2}}-{{z}^{2}} & 2{{z}^{2}}-{{x}^{2}} & 3{{x}^{2}}-{{y}^{2}}  \\
    \end{vmatrix}} \rd s -\dfrac{2}{\sqrt{3}}\displaystyle\iint\limits_{\Sigma }{\left( 4x+2y+3z \right)\rd S}=-24.$$
  }

  \Example{%
    2012-2013-2-期中-解答题-21
  }{%

  设$f(x,y)$在单位圆域上有连续偏导数,且在边界上取值为零,证明:
    $$\lim_{\delta \to 0^+}\dfrac{-1}{2\pi }\displaystyle\iint\limits_{D}{\dfrac{x{{f}_{x}}+y{{f}_{y}}}{{{x}^{2}}+{{y}^{2}}}}\rd x\rd y=f(0,0)$$
  其中D为圆环域${{\delta }^{2}}\le {{x}^{2}}+{{y}^{2}}\le 1$.
  }{%
    见解析.
  }{%
    首先将被积函数恒等变形可得
    \begin{align*}
      & \displaystyle\iint\limits_{D}{\dfrac{x{{f}_{x}}+y{{f}_{y}}}{{{x}^{2}}+{{y}^{2}}}}\rd x\rd y \\
      = & \displaystyle\iint\limits_{D}{\left[ \dfrac{\partial }{\partial x}\left( \dfrac{x}{{{x}^{2}}+{{y}^{2}}}f \right)+\dfrac{\partial }{\partial y}\left( \dfrac{y}{{{x}^{2}}+{{y}^{2}}}f \right) \right]\rd x\rd y} -\displaystyle\iint\limits_{D}{\left[ \dfrac{\partial }{\partial x}\left( \dfrac{x}{{{x}^{2}}+{{y}^{2}}} \right)+\dfrac{\partial }{\partial y}\left( \dfrac{y}{{{x}^{2}}+{{y}^{2}}} \right) \right] f\left( x,y \right)\rd x\rd y} \\
      = & \displaystyle\iint\limits_{D}{\left( \dfrac{\partial }{\partial x}\left( \dfrac{x}{{{x}^{2}}+{{y}^{2}}}f \right)+\dfrac{\partial }{\partial y}\left( \dfrac{y}{{{x}^{2}}+{{y}^{2}}}f \right) \right)\rd x\rd y}=I
    \end{align*}
    注意到在${{x}^{2}}+{{y}^{2}}=1$上$f\left( x,y \right)$函数值为零,故利用格林公式以及积分中值定理可得,
    \begin{align*}
      {{I}_{1}} = & \oint\limits_{{{x}^{2}}+{{y}^{2}}=1}{\dfrac{x}{{{x}^{2}}+{{y}^{2}}}f\left( x,y \right)\rd y-\dfrac{y}{{{x}^{2}}+{{y}^{2}}}f\left( x,y \right)\rd x} - \oint\limits_{{{x}^{2}}+{{y}^{2}}={{\delta }^{2}}}{\dfrac{x}{{{x}^{2}}+{{y}^{2}}}f\left( x,y \right)\rd y-\dfrac{y}{{{x}^{2}}+{{y}^{2}}}f\left( x,y \right)\rd x} \\
      = & 0-\dfrac{1}{{{\delta }^{2}}}\oint\limits_{{{x}^{2}}+{{y}^{2}}={{\delta }^{2}}}{xf\left( x,y \right)\rd y-yf\left( x,y \right)\rd x}
      = -\dfrac{1}{{{\delta }^{2}}}\displaystyle\iint\limits_{{{x}^{2}}+{{y}^{2}}\le {{\delta }^{2}}}{\left[ \left( f+x{{f}_{x}} \right)+\left( f+y{{f}_{y}} \right) \right]\rd x\rd y} \\
      = & -\pi \left[ 2f\left( \xi ,\eta  \right)+3{{f}_{x}}\left( \xi ,\eta  \right)+\eta {{f}_{y}}\left( \xi ,\eta  \right) \right]
    \end{align*}
    其中${{\xi }^{2}}+{{\eta }^{2}}=1$,故原式$=-\dfrac{1}{2\pi }\cdot \left( -2\pi  \right)f\left( 0,0 \right)=f\left( 0,0 \right)$.
  }

\type{第一类曲面积分}

  \Example{%
    2013-2014-2-期末-填空题-5
  }{%
    设$\Sigma $是锥面${{z}^{2}}=3\left( {{x}^{2}}+{{y}^{2}} \right)$被平面$z=0$及$z=3$所截得的部分,则$\displaystyle\iint\limits_{\Sigma }{\left( {{x}^{2}}+{{y}^{2}} \right) \rd s =}$
    \fillin{$9\pi$}.
  }{%
    $9\pi $.
  }{%
    ${{z}^{2}}=3\left( {{x}^{2}}+{{y}^{2}} \right)$,将$z=3$代入,在$xy$平面上的投影为${{x}^{2}}+{{y}^{2}}\le 3$,由$\rd S=\sqrt{1+z_{x}^{2}+z_{y}^{2}}\rd x\rd y=2 \rd x  \rd y $知

    原式$=\displaystyle\iint\limits_{{{x}^{2}}+{{y}^{2}}\le 3}{2\left( {{x}^{2}}+{{y}^{2}} \right) \rd x  \rd y =2\displaystyle\int_{0}^{2\pi }{ \rd \theta  }\displaystyle\int_{0}^{\sqrt{3}}{{{\rho }^{2}}}}d\rho =9\pi $.
  }

  \Example{%
    2015-2016-2-期末模拟-选择题-6
  }{%

  下列对面积的曲面积分不为零的有
  \pickout{D}
  \options{$\displaystyle\oiint\limits_{{{x}^{2}}+{{y}^{2}}+{{z}^{2}}=1} x\cos x \rd s $}
    {$\displaystyle\iint\limits_{\Sigma }{{{y}^{3}}} \rd s $,其中$\Sigma $是椭圆面$\dfrac{{{x}^{2}}}{4}+\dfrac{{{y}^{2}}}{9}+{{z}^{2}}=1$位于第一和第四象限部分}
    {$\displaystyle\oiint\limits_{{{x}^{2}}+{{y}^{2}}+{{z}^{2}}=1} \dfrac{xy+yz+xz}{\sqrt{{{x}^{2}}+{{y}^{2}}+{{z}^{2}}}} \rd s $}
    {$\displaystyle\oiint\limits_{{{x}^{2}}+{{y}^{2}}+{{z}^{2}}=1} \left( {{x}^{2}}+{{y}^{2}}+x+y \right) \rd s $}
  }{%
    D.
  }{%
    根据曲域是否对称及奇函数可知.
  }

  \Example{%
    2016-2017-2-期末模拟-选择题-13
  }{%
    设曲面$\Sigma $是上半球面$:{{x}^{2}}+{{y}^{2}}+{{z}^{2}}={{R}^{2}}\left( z\ge 0 \right),$曲面${{\Sigma }_{1}}$是曲面$\Sigma $在第一卦限的部分,下列结论正确的是
    \pickout{C}
    \options{$\displaystyle\iint\limits_{\Sigma }{x \rd s =4\displaystyle\iint\limits_{{{\Sigma }_{1}}}{x \rd s }}$}
      {$\displaystyle\iint\limits_{\Sigma }{y \rd s =4\displaystyle\iint\limits_{{{\Sigma }_{1}}}{y \rd s }}$}
      {$\displaystyle\iint\limits_{\Sigma }{z \rd s =4\displaystyle\iint\limits_{{{\Sigma }_{1}}}{z \rd s }}$}
      {$\displaystyle\iint\limits_{\Sigma }{xyz \rd s =4\displaystyle\iint\limits_{{{\Sigma }_{1}}}{xyz \rd s }}$}
  }{%
    C.
  }{%
    对A, 函数$x$关于$yOz$对称,所以$\displaystyle\iint\limits_{\Sigma }{x \rd s =0}$
    对B, 函数$y$关于$xOz$对称,所以$\displaystyle\iint\limits_{\Sigma }{y \rd s =0}$
    同理对D, $\displaystyle\iint\limits_{\Sigma }{xyz \rd s =0}$.
  }

  \Example{%
    2013-2014-2-期末-填空题-4
  }{%
    设$\Sigma $是锥面$z=\sqrt{{{x}^{2}}+{{y}^{2}}}$及平面$z=1$所围成的区域的整个边界曲面,则$\displaystyle\iint\limits_{\Sigma }{\left( {{x}^{2}}+{{y}^{2}} \right)} \rd s =$ \fillin{$\dfrac{1+\sqrt{2}}{2}\pi $}.
  }{%
    $\dfrac{1+\sqrt{2}}{2}\pi $.
  }{%
    $\Sigma $是锥面的整个表面,补曲面${{\Sigma }_{2}}$$:z=1,{{x}^{2}}+{{y}^{2}}=1$,在${{\Sigma }_{2}}$上,$ \rd s =\sqrt{1+{{z}_{x}}^{2}+{{z}_{y}}^{2}} \rd x  \rd y =\sqrt{2} \rd x  \rd y $,${{\Sigma }_{1}}$在$xOy$面的投影${{D}_{xy}}:{{x}^{2}}+{{y}^{2}}\le 1$;在${{\Sigma }_{2}}$上,$ \rd s =1 \rd x  \rd y $,于是
    \begin{align*}
      & \displaystyle\iint\limits_{\Sigma }{\left( {{x}^{2}}+{{y}^{2}} \right)} \rd s
      =\displaystyle\iint\limits_{{{D}_{xy}}}{\left( {{x}^{2}}+{{y}^{2}} \right)\sqrt{2} \rd x  \rd y +\displaystyle\iint\limits_{{{\Sigma }_{2}}}{\left( {{x}^{2}}+{{y}^{2}} \right) \rd s }}
      =\left( \sqrt{2}+1 \right)\displaystyle\int_{0}^{2\pi }{ \rd \theta  \displaystyle\int_{0}^{1}{{{r}^{2}}\cdot r \rd r }} \\
      = & \left( \sqrt{2}+1 \right)\cdot 2\pi \cdot \dfrac{1}{4}{{r}^{4}}\Big|_0^1=\dfrac{1+\sqrt{2}}{2}\pi.
    \end{align*}
  }

\type{第二类曲面积分}
  \Example{%
    2012-2013-2-期末-解答题-13
  }{%
    计算
      $$I=\displaystyle\iint\limits_{\Sigma }{\left( x+{{z}^{2}} \right) \rd y  \rd z +z \rd x  \rd y}$$
    其中$\Sigma $是旋转抛物面$z=\dfrac{1}{2}\left( {{x}^{2}}+{{y}^{2}} \right)$介于$z=0$与$z=2$之间的部分的下侧.
  }{%
    见解析.
  }{%
    作辅助曲面${\Sigma }':z=2,{{x}^{2}}+{{y}^{2}}\le 4$,并取上侧. 记$D:{{x}^{2}}+{{y}^{2}}\le 4$,则
    \begin{align*}
      I & = \displaystyle\oiint\limits_\Sigma +{\Sigma }'
     \left( x+{{z}^{2}} \right)\rd y\rd z+z\rd x\rd y -\displaystyle\iint\limits_{{{\Sigma }'}}{\left( x+{{z}^{2}} \right)\rd y\rd z+z\rd x\rd y} \\
      = & \iiint\limits_{\Omega }{2\rd x\rd y\rd z-\displaystyle\iint\limits_{D}{2\rd x\rd y}}
      = 2\displaystyle\int_{0}^{2\pi }{\rd \theta \displaystyle\int_{0}^{2}{r\rd r\displaystyle\int_{\frac{1}{2}{{r}^{2}}}^{2}{\rd z}}}-8\pi  \\
      = & 8\pi -8\pi =0.
    \end{align*}
  }

  \Example{%
    2013-2014-2-期末-解答题-13
  }{%
    计算$I=\displaystyle\iint\limits_{\Sigma }{\left( z\cos \gamma +y\cos \beta +x\cos \alpha  \right)} \rd s ,$其中$\Sigma $是球面$2z={{x}^{2}}+{{y}^{2}}+{{z}^{2}}$,$\cos \alpha $,$\cos \beta $,$\cos \gamma $是$\Sigma $上点的外法向量的方向余弦.
  }{%
    见解析.
  }{%
    可以直接利用Guass公式:$\iiint\limits_{\Omega }{\left( 1+1+1 \right)dv}=3\iiint\limits_{\Omega }{dv}=3\times \dfrac{4}{3}\pi \times {{1}^{3}}=4\pi $.
  }

  \Example{%
    2015-2016-2-期末模拟-填空题-5
  }{%
    已知$\Sigma $为球面${{x}^{2}}+{{y}^{2}}+{{z}^{2}}=1$的外侧,试计算$\displaystyle\oiint\limits_\Sigma \dfrac{ \rd y  \rd z }{x}+\dfrac{ \rd z  \rd x }{y}+\dfrac{ \rd x  \rd y }{z}=$
    \fillin{$12\pi$}.
  }{%
    $12\pi $.
  }{%
    由对称性得,原式$=3\displaystyle\oiint\limits_\Sigma
   \dfrac{1}{z}  \rd x  \rd y =6\displaystyle\int_{0}^{2\pi }{ \rd \theta  }\displaystyle\int_{0}^{1}{\dfrac{\rho }{\sqrt{1-{{\rho }^{2}}}}d\rho }=12\pi $.
  }

  \Example{%
    2015-2016-2-期末模拟-计算题-六
  }{%
    (1) 设$\Sigma $为下半球面$z=-\sqrt{{{a}^{2}}-{{x}^{2}}-{{y}^{2}}}$的上侧,计算
    $$\displaystyle\iint\limits_{\Sigma }{\dfrac{ax \rd y  \rd z +{{\left( z+a \right)}^{2}} \rd x  \rd y }{{{\left( {{x}^{2}}+{{y}^{2}}+{{z}^{2}} \right)}^{\frac{1}{2}}}}},a>0.$$

    (2) 计算曲线积分
    $$I=\oint_{C}{\left( z-y \right) \rd x +\left( x-z \right) \rd y +\left( x-y \right) \rd z }$$
    其中$C:\begin{cases}
      {{x}^{2}}+{{y}^{2}}=1  \\
      x-y+z=2
    \end{cases}$,从$z$轴正向往负方向看是顺时针.
  }{%
    见解析.
  }{%
    (1) 补平面${{\Sigma }_{1}}$$:z=0,{{x}^{2}}+{{y}^{2}}\le {{a}^{2}}$,取其下侧构成封闭曲面,由高斯公式得
      $$I=\displaystyle\iint\limits_{\Sigma +{{\Sigma }_{1}}}{\dfrac{ax \rd y  \rd z +{{\left( z+a \right)}^{2}} \rd x  \rd y }{{{\left( {{x}^{2}}+{{y}^{2}}+{{z}^{2}} \right)}^{\frac{1}{2}}}}}-\displaystyle\iint\limits_{{{\Sigma }_{1}}}{\dfrac{ax \rd y  \rd z +{{\left( z+a \right)}^{2}} \rd x  \rd y }{{{\left( {{x}^{2}}+{{y}^{2}}+{{z}^{2}} \right)}^{\frac{1}{2}}}}}$$
    其中
    $\displaystyle\iint\limits_{\Sigma +{{\Sigma }_{1}}}{\dfrac{ax \rd y  \rd z +{{\left( z+a \right)}^{2}} \rd x  \rd y }{{{\left( {{x}^{2}}+{{y}^{2}}+{{z}^{2}} \right)}^{\frac{1}{2}}}}}
    =-\dfrac{1}{a}\displaystyle\iint\limits_{\Omega }{\left( 3a+2z \right) \rd x  \rd y  \rd z }$

    $= -\dfrac{1}{a}\displaystyle\int_{0}^{2\pi }{ \rd \theta  }\displaystyle\int_{\frac{\pi }{2}}^{\pi }{ \rd \varphi  }\displaystyle\int_{0}^{a}{\left( 3a+2r\cos \varphi  \right){{r}^{2}}\sin \varphi  \rd r }
    = -\dfrac{3\pi }{2}{{a}^{3}}$

    $\displaystyle\iint\limits_{{{\Sigma }_{1}}}{\dfrac{ax \rd y  \rd z +{{\left( z+a \right)}^{2}} \rd x  \rd y }{{{\left( {{x}^{2}}+{{y}^{2}}+{{z}^{2}} \right)}^{\frac{1}{2}}}}}=-\dfrac{1}{a}\displaystyle\iint\limits_{{{\Sigma }_{1}}}{{{a}^{2}} \rd x  \rd y }=-a\cdot \pi {{a}^{2}}
     = -\pi {{a}^{3}}$

    于是$I=-\dfrac{3\pi }{2}{{a}^{3}}+\pi {{a}^{3}}=-\dfrac{\pi }{2}{{a}^{3}}$.
    (2) $x=\cos \theta ,y=\sin \theta ,z=2-\cos \theta +\sin \theta $,
    \begin{align*}
      \therefore I = & -\displaystyle\int_{0}^{2\pi }{\left[ \left( 2-\cos \theta  \right)\cdot \left( -\sin \theta  \right)+\left( 2\cos \theta -2-\sin \theta  \right)\cdot \cos \theta +\left( \cos \theta -\sin \theta  \right)\cdot \left( \sin \theta +\cos \theta  \right) \right] \rd \theta  } \\
      = & -\displaystyle\int_{0}^{2\pi }{\left( \sin \theta \cos \theta -2\sin \theta +2{{\cos }^{2}}\theta -2\cos \theta -\sin \theta \cos \theta +{{\cos }^{2}}\theta -{{\sin }^{2}}\theta  \right)} \rd \theta   \\
      = & -\displaystyle\int_{0}^{2\pi }{\left( 3{{\cos }^{2}}\theta -{{\sin }^{2}}\theta -2\sin \theta -2\cos \theta  \right)} \rd \theta  =-2\pi.
    \end{align*}
  }

  \Example{%
    2013-2014-2-期末-计算题-14
  }{%

  计算$I=\displaystyle\iint\limits_{\Sigma }{\left( 2x+z \right) \rd y  \rd z +z \rd x  \rd y },$其中$\Sigma $为有向曲面$z={{x}^{2}}+{{y}^{2}}\left( 0\le z\le 1 \right),$其法向量与$z$轴正向的夹角为锐角.
  }{%
    见解析.
  解析:
  利用矢量投影法,由已知$z{{'}_{x}}=2x,z{{'}_{y}}=2y,$于是
    \begin{align*}
    I = & \displaystyle\iint\limits_{\Sigma }{\left( 2x+z \right) \rd y  \rd z +z \rd x  \rd y }=\displaystyle\iint\limits_{\Sigma }{\left[ \left( 2x+z \right)\cdot \left( -z{{'}_{x}} \right)+z \right]} \rd x  \rd y  \\
   = & \displaystyle\iint\limits_{\Sigma }{\left( -4{{x}^{2}}-2xz+z \right) \rd x  \rd y }=\displaystyle\iint\limits_{\Sigma }{\left[ -4{{x}^{2}}-2x\left( {{x}^{2}}+{{y}^{2}} \right)+{{x}^{2}}+{{y}^{2}} \right]} \rd x  \rd y  \\
   = & \displaystyle\int_{0}^{2\pi }{ \rd \theta  \displaystyle\int_{0}^{1}{\left( -4{{r}^{2}}{{\cos }^{2}}\theta -2{{r}^{3}}\cos \theta +{{r}^{2}} \right)}r \rd r } \\
   = & -\dfrac{\pi }{2}.
  \end{align*}

  }

  \Example{%
    2014-2015-2-期末-计算题-16
  }{%
    计算$I=\displaystyle\iint\limits_{\Sigma }{2\left( 1-{{x}^{2}} \right)} \rd y  \rd z +8xy \rd z  \rd x -4xz \rd x  \rd y ,$其中$\Sigma $是由$xOy$面上的弧段$x={{\re}^{y}}\left( 0\le y\le a \right)$绕$x$轴旋转所成旋转曲面,$\Sigma $的法向量与$x$轴正向夹角大于$\dfrac{\pi }{2}$.
  }{%
    $2\pi {{a}^{2}}\left( {{\re}^{2a}}-1 \right)$.
  }{%
    画图,补面${\Sigma }':~\begin{cases}
      x={{\re}^{a}} \\
      {{y}^{2}}+{{z}^{2}}={{a}^{2}}
    \end{cases}$,方向为$x$轴正向,于是由Gauss公式得
    \begin{align*}
      I = & \displaystyle\iint\limits_{\Sigma +{\Sigma }'}{2\left( 1-{{x}^{2}} \right)} \rd y  \rd z +8xy \rd z  \rd x -4xz \rd x  \rd y -\displaystyle\iint\limits_{{{\Sigma }'}}{2\left( 1-{{x}^{2}} \right)} \rd y  \rd z +8xy \rd z  \rd x -4xz \rd x  \rd y  \\
      = & \iiint\limits_{\Omega }{\left( -4x+8x-4x \right)\rd x\rd y\rd z}-\displaystyle\iint\limits_{{{\Sigma }'}}{2\left( 1-{{\re}^{2a}} \right)} \rd y  \rd z
      = 0-\displaystyle\iint\limits_{{{\Sigma }'}}{2\left( 1-{{\re}^{2a}} \right)} \rd y  \rd z  \\
      = & 2\left( {{\re}^{2a}}-1 \right)\cdot \pi {{a}^{2}}.
    \end{align*}
  }