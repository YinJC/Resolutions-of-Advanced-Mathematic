% !TEX root = ../HTNotes-Demo.tex
% !TEX program = xelatex
% 内容开始
\type{较为复杂的极限的计算}

  \hspace*{2em}在已经学习过的基本方法的基础上,进一步结合洛必达法则、泰勒公式与拉格朗日中值定理.

  \Example{%
    2011-2011-1-期中-填空题-3
  }{%
    $\lim\limits_{x \to 0} \dfrac{{{\left( 1+x \right)}^{\frac{1}{x}}}-{\re}}{x}=$
    \fillin{$-\dfrac{1}{2}{\re}$}.
  }{%
    $-\dfrac{1}{2}{\re}$.
  }{%
    原式$=\lim\limits_{x \to 0} \dfrac{{{{\re}}^{\frac{\ln (1+x)}{x}}}-{\re}}{x}
    ={\re}\cdot \lim\limits_{x \to 0} \dfrac{{{{\re}}^{\frac{\ln \left( 1+x \right)}{x}-1}}-1}{x}$,
    由等价无穷小$\ln \left( 1+x \right)\sim x\sim {{{\re}}^{x}}-1\left( x\to 0 \right)$可知

    原式$={\re}\cdot \lim\limits_{x \to 0} \dfrac{\frac{\ln \left( 1+x \right)}{x}-1}{x}
    ={\re}\cdot \lim\limits_{x \to 0} \dfrac{\ln \left( 1+x \right)-x}{{{x}^{2}}}
    ={\re}\cdot \lim\limits_{x \to 0} \dfrac{\frac{1}{1+x}-1}{2x}
    ={\re}\cdot \lim\limits_{x \to 0} \dfrac{1-\left( 1+x \right)}{2x\left( 1+x \right)}
    =-\dfrac{{\re}}{2}$.

    其中用到了洛必达法则(也可以采用$\ln \left( 1+x \right)$的泰勒公式求解).
  }

  \Example{%
    2011-2011-1-期中-选择题-16
  }{%
    若$\lim\limits_{x \to 0} \dfrac{\sin 6x+xf(x)}{{{x}^{3}}}=0$,则$\lim\limits_{x \to +\infty} \dfrac{6+f(x)}{{{x}^{2}}}=$
    \fillin{$36$}.
  }{%
    $36$.
  }{%
    由极限四则运算法则及泰勒公式$\sin x=x-\dfrac{1}{6}{{x}^{3}}+o\left( {{x}^{3}} \right),x\in \left( -\delta ,\delta  \right)$知
    \begin{align*}
      & \lim\limits_{x \to 0} \dfrac{6+f\left( x \right)}{{{x}^{2}}}
      = \lim\limits_{x \to 0} \dfrac{6x+xf\left( x \right)}{{{x}^{3}}}
      = \lim\limits_{x \to 0} \left[ \dfrac{\sin 6x+xf\left( x \right)}{{{x}^{3}}}+\dfrac{6x-\sin 6x}{{{x}^{3}}} \right] \\
      & =\lim\limits_{x \to 0} \dfrac{\sin 6x+xf\left( x \right)}{{{x}^{3}}}+\lim\limits_{x \to 0} \dfrac{6x-\sin 6x}{{{x}^{3}}}
      = 0+\lim\limits_{x \to 0} \dfrac{\frac{1}{6}{{\left( 6x \right)}^{3}}}{{{x}^{3}}}=36.
    \end{align*}
  }

  \Example{%
    2014-2015-1-期中-填空题-1
  }{%
    设曲线$y=f\left( x \right)={{x}^{n}}$在点$\left( 1,1 \right)$处的切线与$x$轴的交点为$\left( {{\xi }_{n}},0 \right)$,则$\lim\limits_{n \to +\infty} f\left( {{\xi }_{n}} \right)=$
    \fillin{$\dfrac{1}{\re}$}.
  }{%
    $\dfrac{1}{{\re}}$.
  }{%
    由${f}'\left( 1 \right)=n$知点$\left( 1,1 \right)$处的切线为$y-1=n\left( x-1 \right)\Rightarrow x=1+\dfrac{y-1}{n}$

    于是${{\xi }_{n}}=1-\dfrac{1}{n}$,
    故$\lim\limits_{n \to +\infty} f\left( {{\xi }_{n}} \right)
    =\lim\limits_{n \to +\infty} {{\left( 1-\dfrac{1}{n} \right)}^{n}}
    =\dfrac{1}{{\re}}$.
  }

  \Example{%
    2017-2018-1-期中-解答题-2
  }{%
    利用泰勒展开式求极限$\lim\limits_{x \to 0} \dfrac{{{\re}^{x}}\sin x-x\left( 1+x \right)}{{{x}^{3}}}$.
  }{%
    见解析.
  }{%
    原式$=\lim\limits_{x \to 0} \dfrac{\left[ 1+x+\frac{1}{2}{{x}^{2}}+o\left( {{x}^{2}} \right) \right]\left[ x-\frac{1}{6}{{x}^{3}}+o\left( {{x}^{3}} \right) \right]-x\left( 1+x \right)}{{{x}^{3}}}$

    $=\lim\limits_{x \to 0} \dfrac{x-\frac{1}{6}{{x}^{3}}+{{x}^{2}}+\frac{1}{2}{{x}^{3}}+o\left( {{x}^{3}} \right)-x-{{x}^{2}}}{{{x}^{3}}}=\dfrac{1}{3}$.
  }

  \Example{%
    2014-2015-1-期中-选择题-18
  }{%
    设当$x\to 0$时,函数$f\left( x \right)=3\sin x-\sin 3x$与$c{{x}^{k}}$是等价无穷小,则
    \pickout{A}
    \options{$k=3,c=4$}
      {$k=3,c=-4$}
      {$k=1,c=-4$}
      {$k=1,c=4$}
  }{%
    A.
  }{%
    $3\sin x-\sin 3x\sim 3(x-\dfrac{1}{6}{{x}^{3}})-(3x-\dfrac{1}{6}{{(3x)}^{3}})
    =4{{x}^{3}}$.
  }

  \Example{%
    2015-2016-1-期中-填空题-9
  }{%
     $\lim\limits_{x \to 0} \dfrac{\frac{{{(1+x)}^{\frac{1}{x}}}}{\re}-1}{x}=$
     \fillin{$-\dfrac{1}{2}$}.
  }{%
    $-\dfrac{1}{2}$.
  }{%
    $\lim\limits_{x \to 0} \dfrac{\dfrac{{{(1+x)}^{\frac{1}{x}}}}{\re}-1}{x}$
    $=\lim\limits_{x \to 0} \dfrac{{{\re}^{\frac{1}{x}\ln (1+x)-1}}-1}{x}$
    $=\lim\limits_{x \to 0} \dfrac{^{\frac{1}{x}\ln (1+x)-1}}{x}$
    $=\lim\limits_{x \to 0} \dfrac{\ln (1+x)-x}{{{x}^{2}}}$

    $=\lim\limits_{x \to 0} \dfrac{x-\frac{1}{2}{{x}^{2}}-x}{{{x}^{2}}}
    =-\dfrac{1}{2}$.
  }

  \Example{%
    2014-2015-1-期末-计算题-14
  }{%
    计算$\lim\limits_{x \to +\infty} x\left( {{\left( 1+\frac{1}{x} \right)}^{x}} - \re \right)$.
  }{%
    见解析.
  }{%
    倒代换,令$t=\dfrac{1}{x}$,将原式化为
    \begin{align*}
      & \lim\limits_{t \to 0} \dfrac{{{(1+t)}^{\frac{1}{t}}}- \re }{t}
      =\lim\limits_{t \to 0} \dfrac{{{\re}^{\frac{1}{t}\ln (1+t)}}- \re }{t}
      =\lim\limits_{t \to 0} \dfrac{ \re ({{\re}^{\frac{1}{t}\ln (1+t)-1}}-1)}{t}
      = \re \lim\limits_{t \to 0} \dfrac{\frac{1}{t}\ln (1+t)-1}{t} \\
      & = \re \lim\limits_{t \to 0} \dfrac{\ln (1+t)-t}{{{t}^{2}}}
      = \re \lim\limits_{t \to 0} \dfrac{\frac{1}{1+t}-1}{2t}
      = \re \lim\limits_{t \to 0} \dfrac{-\frac{1}{{{(1+t)}^{2}}}}{2}
      =-\dfrac{\re}{2}.
    \end{align*}
  }

\type{导数的应用}

  \hspace*{2em}主要用于判定函数的点态特征,包括单调性与极值、凹凸性与拐点、渐近线与曲率圆,以及帮助作出简单函数的图像等.

  \subsection{求切线}
    \Example{%
      2013-2014-1-期中-解答题-19
    }{%
      设$y=y\left( x \right)$由方程
      $\begin{cases}
        x=\arctan t \\
        2y-t{{y}^{2}}+{{{\re}}^{t}}=5 \\
      \end{cases}$
      确定,求在点$\left( 0,2 \right)$处的切线方程.
    }{%
      见解析.
    }{%
      本质上是求参数方程
      $\begin{cases}
        x=\arctan t \\
        2y-t{{y}^{2}}+{{{\re}}^{t}}=5 \\
      \end{cases}$在$\left( 0,2 \right)$处的导数.

      由隐函数的求导方法可得
      $$2y-t{{y}^{2}}+{{{\re}}^{t}}
      =5\Rightarrow 2\dfrac{\rd y}{\rd t}-{{y}^{2}}-2ty\dfrac{\rd y}{\rd t}+{{{\re}}^{t}}=0
      \Rightarrow \dfrac{\rd y}{\rd t}
      =\dfrac{{{{\re}}^{t}}-{{y}^{2}}}{2ty-2},$$
      从而
      $${{\left. \dfrac{\rd y}{\rd x}
      ={\dfrac{\rd y}{\rd t}}/{\dfrac{\rd x}{\rd t}}\;
      =\dfrac{\left( {{{\re}}^{t}}-{{y}^{2}} \right)\left( 1+{{t}^{2}} \right)}{2ty-2}\Rightarrow \dfrac{\rd y}{\rd x} \right|}_{x=0}}
      ={{\left. \dfrac{\left( {{{\re}}^{t}}-{{y}^{2}} \right)\left( 1+{{t}^{2}} \right)}{2ty-2} \right|}_{t=0,y=2}}
      =\dfrac{3}{2},$$
      于是切线方程为$y=\dfrac{3}{2}x+2$.
    }

    \Example{%
      2015-2016-1-期中-填空题-10
    }{%
      设对数螺线$\rho ={{\re}^{\theta }}$,该曲线在对应于$\theta =\dfrac{\pi }{2}$的点处的切线方程为
      \fillin{$x+y={{\re}^{\frac{\pi }{2}}}$}.
    }{%
      $x+y={{\re}^{\frac{\pi }{2}}}$.
    }{%
      $\because x=\rho \cos \theta ,y=\rho \sin \theta $
      $\therefore $对数螺线方程可化为:
      $\begin{cases}
        x={{\re}^{\theta }}\cos \theta \\
        y={{\re}^{\theta }}\sin \theta \\
      \end{cases}$

      $\therefore \dfrac{\rd y}{\rd x}
      =\dfrac{\frac{\rd y}{\rd \theta }}{\frac{\rd x}{\rd \theta }}
      =\dfrac{\sin \theta +\cos \theta }{\cos \theta -\sin \theta }$,
      $\left.\dfrac{\rd y}{\rd x}\right|_{\theta=\frac{\pi }{2}}
      =-1$,

      又$\left( \rho ,\theta  \right)=\left( {{\re}^{\frac{\pi }{2}}},\frac{\pi }{2} \right)$时,$x=0,y={{\re}^{\frac{\pi }{2}}}$

      $\therefore $在$\left( \rho ,\theta  \right)=\left( {{\re}^{\frac{\pi }{2}}},\frac{\pi }{2} \right)$时的切线方程为:$y-{{\re}^{\frac{\pi }{2}}}=-1\left( x-0 \right)$
      即$x+y={{\re}^{\frac{\pi }{2}}}$.
  }

  \subsection{判定单调性与极值/最值}
    \Example{%
      2012-2013-1-期末-选择题-12
    }{%
      设$\text{y=}f'(x)$对一切$x$满足$x{f}''\left( x \right)+3x{{\left[ {f}'\left( x \right) \right]}^{2}}=1-{{{\re}}^{x}}$,若${f}''\left( {{x}_{0}} \right)=0,{{x}_{0}}\ne 0$,则
      \pickout{B}.
      \options{$f({x_{0}})$是$f(x)$的极大值}
        {$f({x_{0}})$是$f(x)$的极小值}
        {$f({x_{0}},f({x_{0}}))$时曲线$y=f(x)$的拐点}
        {以上都不对}
    }{%
      B.
    }{%
      式子带入${{x}_{0}}$即有
      $f''({{x}_{0}})=\dfrac{1-{{\re}^{-{{x}_{0}}}}}{{{x}_{0}}}\ne 0,{{x}_{0}}>0$和${{x}_{0}}<0$时,均有$f''({{x}_{0}})>0$

      则$f({{x}_{0}})$是极小值.
    }

    \Example{%
      2014-2015-1-期末-选择题-7
    }{%
      函数$f\left( x \right)=\ln \left( 1+{{x}^{2}} \right)$在$[-1,2]$上的最大值是
      \pickout{D}
      \options{$\ln 2$}
        {$\ln 3$}
        {$\ln 4$}
        {$\ln 5$}
    }{%
      D.
    }{%
      ${f}'\left( x \right)=\dfrac{2x}{1+{{x}^{2}}}$,于是$\forall x\in \left[ -1,0 \right]$,
      ${f}'\left( x \right)<0$;
      $\forall x\in \left[ 0,2 \right]$,
      ${f}'\left( x \right)>0$

      则$f\left( x \right)$在$\left[ -1,0 \right]$上单调递减,在$\left[ 0,2 \right]$上单调递增

      于是$f\left( x \right)=\ln \left( 1+{{x}^{2}} \right)$在$[-1,2]$上的最大值为$\max \left\{ f\left( -1 \right),f\left( 2 \right) \right\}=\ln 5$.
    }

    \Example{%
      2012-2013-1-期中-填空题-2
    }{%
      $y={{x}^{2}}{{\re}^{-x}}$的极大值是
      \fillin{$\dfrac{4}{{{\re}^{2}}}$}.
    }{%
      $\dfrac{4}{{{\re}^{2}}}$.
    }{%
      $y'=2x{{\re}^{-x}}-{{x}^{2}}{{\re}^{-x}}=x{{\re}^{-x}}(2-x),{{\re}^{-x}}>0$
      令$y'=0$ $\Rightarrow {{x}_{1}}=0,{{x}_{2}}=2$

      则$x \in \left( -\infty ,0 \right),y'<0,y\downarrow$;
      $x \in \left[ 0,2 \right],~y'>0,~y\uparrow$;
      $x \in \left( 2,+\infty  \right),~y'<0,~y\downarrow$

      所以当$x=2$时极大值为$\dfrac{4}{{{\re}^{2}}}$.
    }

    \Example{%
      2015-2016-1-期中-填空题-3
    }{%
      若$y=ex-{{\re}^{-\lambda x}}$有正的极值点,则参数$\lambda $的取值范围是
      \fillin{$-\re<\lambda <0$}.
    }{%
      $-\re<\lambda <0$.
    }{%
      $y'=\re+\lambda {{\re}^{-\lambda x}}$,$y'=0$有正的实根,即$\lambda {{\re}^{-\lambda x}}$$=-\re$,${{\re}^{-\lambda x}}$$=-\dfrac{\re}{\lambda }$

      令$f\left( x \right)={{\re}^{-\lambda x}}$,$g\left( x \right)=-\dfrac{\re}{\lambda }$即$f\left( x \right)=g\left( x \right)$存在$x>0$的点

      即图像有交点,画图可得$0<-\dfrac{\re}{\lambda }<1$,即$-\re<\lambda <0$.
    }

    \Example{%
      2015-2016-1-期中-选择题-6
    }{%
      若$a,b,c,d$成等比数列,则函数$f\left( x \right)=\dfrac{1}{3}a{{x}^{3}}+b{{x}^{2}}+cx+d$
      \pickout{D}
      \options{有极大值,而无极小值}
        {无极大值,而有极小值}
        {有极大值,也有极小值}
        {无极大值,也无极小值}
    }{%
      D.
    }{%
      $\because y=\dfrac{1}{3}a{{x}^{3}}+b{{x}^{2}}+cx+d,\therefore y'=a{{x}^{2}}+2bx+c$.

      而$a,b,c,d$成等比数列,则${{b}^{2}}=ac$
      令$a{{x}^{2}}+2bx+c=0$,则$\Delta =4{{b}^{2}}-4ac=4{{b}^{2}}-4{{b}^{2}}=0$

      $\therefore y'\ge 0$或$y'\le 0$,
      即$f\left( x \right)$无极值.
    }

    \Example{%
      2016-2017-1-期中-选择题-13
    }{%
      设函数$g\left( t \right)$在$\left( -\infty ,+\infty  \right)$内可导,且对任意的${{t}_{1}},{{t}_{2}}$,当${{t}_{1}}>{{t}_{2}}$时,都有$g\left( {{t}_{1}} \right)>g\left( {{t}_{2}} \right)$,则
      \pickout{B}
      \options{对任意的$t$,$g'\left( t \right)>0$}
        {函数$-g\left( -t \right)$单调增加}
        {对任意的$t$,$g'\left( -t \right)\le 0$}
        {函数$g\left( -t \right)$单调增加}
    }{%
      B.
    }{%
      由题可得$g'\left( t \right)\ge 0$,$g\left( t \right)$单调增加

      对于B,$\left[ -g\left( -t \right) \right]'=g'\left( -t \right)\ge 0$,单增

      对于D,$\left[ g\left( -t \right) \right]'=-g'\left( -t \right)\le 0$,单减.
    }

    \Example{%
      2017-2018-1-期中-选择题-6
    }{%
      设$f\left( x \right)$在$x=0$的某邻域内连续,且$f\left( 0 \right)=0,\lim\limits_{x \to 0} \dfrac{f\left( x \right)}{{{x}^{2}}}=1$,则点$x=0$是$f\left( x \right)$的
      \pickout{B}
      \options{极大值点}
        {驻点和极小值点}
        {非驻点}
        {非驻点但是极小值点}
    }{%
      B.
    }{%
      ${f}'\left( 0 \right)
      =\lim\limits_{x \to 0} \dfrac{f\left( x \right)-f\left( 0 \right)}{x-0}
      =\lim\limits_{x \to 0} \dfrac{f\left( x \right)}{x}
      =\lim\limits_{x \to 0} \left[ x\cdot \dfrac{f\left( x \right)}{{{x}^{2}}} \right]
      =\lim\limits_{x \to 0} x\cdot \lim\limits_{x \to 0} \dfrac{f\left( x \right)}{{{x}^{2}}}=0$

      则点$x=0$是$f\left( x \right)$的驻点.
    }

  \subsection{判定拐点与凹凸性}
    \Example{%
      2013-2014-1-期末-填空题-3
    }{%
      曲线$y=\ln ({{x}^{2}}+1)$的拐点是
      \fillin{$(\pm 1,\ln 2)$}.
    }{%
      $(\pm 1,\ln 2)$.
    }{%
      $y'=\dfrac{2x}{{{x}^{2}}+1}$,$y''=\dfrac{2({{x}^{2}}+1)-2x\cdot 2x}{{{({{x}^{2}}+1)}^{2}}}=\dfrac{2-2{{x}^{2}}}{{{x}^{2}}+1{{)}^{2}}}$,
      $y''=0$时,${{x}^{2}}=1,x=\pm 1,y=\ln 2$.
    }

    \Example{%
      2015-2016-1-期末-选择题-8
    }{%
      曲线$y=\begin{cases}
         x{{(x-1)}^{2}}  \\
         {{(x-1)}^{2}}(x-2)  \\
      \end{cases},\begin{cases}
         0\le x\le 1  \\
         1<x\le 2  \\
      \end{cases}$在区间$\left( 0,2 \right)$有
      \pickout{C}
      \options{2个极值点,1个拐点.}
        {2个极值点,2个拐点.}
        {2个极值点,3个拐点.}
        {3个极值点,3个拐点.}
    }{%
      C.
    }{%
      注意$x=1$的特殊性即可.
    }

    \Example{%
      2017-2018-1-期末模拟-选择题-13
    }{%
      曲线$y=(x-1){{(x-2)}^{2}}{{(x-3)}^{3}}{{\left( x-4 \right)}^{4}}$的拐点为
      \pickout{C}
      \options{$\left( 1,0 \right)$}
        {$\left( 2,0 \right)$}
        {$\left( 3,0 \right)$}
        {$\left( 4,0 \right)$}
    }{%
      C.
    }{%
      求$y''$;令$y''=0$,求出使$y''=0$和$y''$不存在的点;

      用这些点把定义域分成若干小区间,讨论$y''$的符号,判断曲线$y$在小区间的凹凸性;

      考察$y''$在$x$两侧的近旁是否变号,如果$y''$变号,那么点$\left( x,f\left( x \right) \right)$是曲线$y$的拐点.
    }

    \Example{%
      2017-2018-1-期中-选择题-7
    }{%
      曲线$y=3{{x}^{5}}-10{{x}^{3}}-360x$的拐点有
      \pickout{C}
      \options{1个}
        {2个}
        {3个}
        {0个}
    }{%
      C.
    }{%
      $y'=15{{x}^{4}}-30{{x}^{2}}-360,y''=60{{x}^{3}}-60x=60x\left( x-1 \right)\left( x+1 \right)$,令${y}''=0$
      并验证根两侧二阶导数的正负可知有3个拐点.
    }

  \subsection{求渐近线}
    \Example{%
      2017-2018-1-期末模拟-选择题-16
    }{%
      曲线$y=\dfrac{1}{x}+\ln (1+{{\re}^{x}})$渐近线条数为
      \pickout{D}
      \options{0}
        {1}
        {2}
        {3}
    }{%
      D.
    }{%
      $\lim\limits_{x \to +\infty} y
      =\lim\limits_{x \to +\infty} \left[ \frac{1}{x}+\ln (1+{{\re}^{x}}) \right]
      =+\infty$,
      $\lim\limits_{x \to -\infty} y
      =\lim\limits_{x \to -\infty} \left[ \frac{1}{x}+\ln (1+{{\re}^{x}}) \right]
      =0,$

      所以$y=0$是曲线的水平渐近线;

      $\lim\limits_{x \to 0} y=\lim\limits_{x \to 0} \left[ \frac{1}{x}+\ln (1+{{\re}^{x}}) \right]=\infty$,
      所以$x=0$是曲线的垂直渐近线;

      $\lim\limits_{x \to +\infty} \dfrac{y}{x}=\lim\limits_{x \to -\infty} \dfrac{\left[ \frac{1}{x}+\ln (1+{{\re}^{x}}) \right]}{x}
      =0+\lim\limits_{x \to +\infty} \dfrac{\ln (1+{{\re}^{x}})}{x}
      =\lim\limits_{x \to +\infty} \dfrac{{{\re}^{x}}}{\frac{1+{{\re}^{x}}}{1}}=1$,

      $b=\lim\limits_{x \to +\infty} \left[ y-x \right]=\lim\limits_{x \to +\infty} \left[ \frac{1}{x}+\ln (1+{{\re}^x})-x \right]=0$

      所以$y=x$是曲线的斜渐近线.

      所以一共3条渐近线.
    }

    \Example{%
      2012-2013-1-期中-选择题-12
    }{%
      曲线$y=\dfrac{{{x}^{2}}+x}{{{x}^{2}}-1}$的渐近线条数为
      \pickout{C}
      \options{0}
        {1}
        {2}
        {3}
    }{%
      C.
    }{%
      由于$y=\dfrac{{{x}^{2}}+x}{{{x}^{2}}-1}$在$x=\pm 1$处没有定义,且$\underset{x\to 1}{\mathop{\lim }}\,\dfrac{{{x}^{2}}+x}{{{x}^{2}}-1}=\infty ,\lim\limits_{x \to -\infty} \dfrac{{{x}^{2}}+x}{{{x}^{2}}-1}=\dfrac{1}{2}$

      $\therefore x=1$是$y$的垂直渐近线;

      又$\lim\limits_{x \to +\infty} \dfrac{{{x}^{2}}+x}{{{x}^{2}}-1}=1$,$\therefore y=1$是曲线的水平渐近线;

      而 $\lim\limits_{x \to +\infty} \dfrac{y}{x}=0$,$\therefore $曲线$y=\dfrac{{{x}^{2}}+x}{{{x}^{2}}-1}$无斜渐近线.

      $\therefore $曲线的渐近线条数为2.
    }

    \Example{%
      2013-2014-1-期中-选择题-17
    }{%
      曲线$y=\dfrac{1+{{\re}^{-{{x}^{2}}}}}{1-{{\re}^{{{x}^{2}}}}}$
      \pickout{D}
      \options{没有渐近线}
        {仅有水平渐近线}
        {仅有铅直渐近线}
        {既有水平渐近线又有铅直渐近线}
    }{%
      D.
    }{%
      $\because \lim\limits_{x \to 0} \dfrac{1+{{\re}^{-{{x}^{2}}}}}{1-{{\re}^{{{x}^{2}}}}}=\infty $,
      $\therefore x=0$是其铅直渐近线;

      又$\because \lim\limits_{x \to +\infty} \dfrac{1+{{\re}^{-{{x}^{2}}}}}{1-{{\re}^{{{x}^{2}}}}}=\dfrac{1+\lim\limits_{x \to +\infty} {{\re}^{-{{x}^{2}}}}}{1-\lim\limits_{x \to +\infty} {{\re}^{-{{x}^{2}}}}}=1$,
      $\therefore y=1$是其水平渐近线.
    }

    \Example{%
      2017-2018-1-期中-选择题-9
    }{%
      曲线$y=x\sin \dfrac{1}{x}$
      \pickout{A}
      \options{只有水平渐近线}
        {只有铅直渐近线}
        {既有水平渐近线又有铅直渐近线}
        {有斜渐近线}
    }{%
      A.
    }{%
      考虑铅直渐近线:$\lim\limits_{x \to 0} x\sin \dfrac{1}{x}=0$,则无铅直渐近线;

      考虑水平渐近线:$\lim\limits_{x \to +\infty} x\sin \dfrac{1}{x}=\lim\limits_{x \to 0} \dfrac{\sin t}{t}=1$,$y=1$是其水平渐近线;

      考虑斜渐近线:$\lim\limits_{x \to +\infty} \dfrac{x\sin \frac{1}{x}}{x}=\lim\limits_{x \to +\infty} \sin \dfrac{1}{x}$,极限不存在,无斜渐近线.
    }

  \subsection{函数性态的综合判断}
    \Example{%
      2013-2014-1-期中-选择题-10
    }{%
      若$f\left( -x \right)=-f\left( x \right)$,在$\left( 0,+\infty  \right)$内$f'\left( x \right)>0,f''\left( x \right)>0$,则$f\left( x \right)$在$\left( -\infty ,0 \right)$内
      \pickout{A}
      \options{$f'\left( x \right)>0,$$f''\left( x \right)<0$}
        {$f'\left( x \right)<0$$,f''\left( x \right)>0$}
        {$f'\left( x \right)<0$,$f''\left( x \right)<0$}
        {$f'\left( x \right)>0,f''\left( x \right)>0$}
    }{%
      A.
    }{%
      $f\left( -x \right)=-f\left( x \right)$,$f\left( x \right)$为奇函数,图像关于原点对称,具有相同的单调性和相反的凹凸性

      又在$\left( 0,+\infty  \right)$内$f'\left( x \right)>0,f''\left( x \right)>0$,所以在$\left( -\infty ,0 \right)$内$f'\left( x \right)>0,$ $f''\left( x \right)<0$.
    }

    \Example{%
      2012-2013-1-期中-选择题-15
    }{%
      设$f\left( -x \right)=f\left( x \right)$,且在$\left( 0,\infty  \right)$内$f'\left( x \right)<0,f''\left( x \right)<0,$则曲线$y=f\left( x \right)$在$\left( -\infty ,0 \right)$内
      \pickout{D}
      \options{单调减且是凹的}
        {单调减且是凸的}
        {单调增且是凹的}
        {单调增且是凸的}
    }{%
      D.
    }{%
      $f\left( -x \right)=f\left( x \right)$,所以$f\left( x \right)$是偶函数,图像关于$y$轴对称

      可知在$y$轴两侧具有相同的凹凸性和相反的单调性

      又$f\left( x \right)$在$\left( 0,\infty  \right)$内$f'\left( x \right)<0$,所以$f\left( x \right)$在$\left( 0,\infty  \right)$是单调减;

      $f'\left( x \right)<0$,$f\left( x \right)$在$\left( 0,\infty  \right)$内是凸函数;

      所以在$\left( -\infty ,0 \right)$内$f\left( x \right)$是单调增且是凸函数.
    }

    \Example{%
      2013-2014-1-期中-选择题-15
    }{%
      设函数$f\left( x \right)$满足关系式$f''\left( x \right)+{{\left[ f'\left( x \right) \right]}^{2}}=x,$且$f'\left( 0 \right)=0$,则
      \pickout{C}
      \options{$x=0$是$f\left( x \right)$的极大值点}
        {$x=0$是$f\left( x \right)$的极小值点}
        {$\left( 0,f\left( 0 \right) \right)$是曲线$y=f\left( x \right)$的拐点}
        {$x=0$不是$f\left( x \right)$的极值点,$\left( 0,f\left( 0 \right) \right)$也不是曲线$y=f\left( x \right)$的拐点}
    }{%
      C.
    }{%
      由$f''\left( x \right)+{{\left[ f'\left( x \right) \right]}^{2}}=x$,得$f\left( x \right)$在其定义域内存在二阶连续导数且$f''\left( 0 \right)=0$

      $f''\left( x \right)$
      $=x$$-{{\left[ f'\left( x \right) \right]}^{2}}$,所以$f'''\left( x \right)=1-2\left[ f'\left( x \right) \right]f''\left( x \right)$,

      所以$f'''\left( 0 \right)=1\ne 0$即$\left( 0,f\left( 0 \right) \right)$是曲线$y=f\left( x \right)$的拐点.
    }

\type{微分中值定理的应用}

  \hspace*{2em}主要用于判定函数的全局性态,例如特殊点、函数值的存在性

  \Example{%
    2015-2016-1期末-证明题-19
  }{%
    设函数$f\left( x \right)$在$\left[ a,b \right]$上具有二阶导数,且$f\left( a \right)=f\left( b \right)=0$,${f}'\left( a \right){f}'\left( b \right)>0$,证明:存在$\xi \in \left( a,b \right),{f}''\left( \xi  \right)=0$.
  }{%
    见解析.
  }{%
    由$f$在$\left[ a,b \right]$上二阶可导可知${f}'$在$\left[ a,b \right]$上连续. 由${f}'\left( a \right){f}'\left( b \right)>0$,不妨设${f}'\left( a \right)>0,{f}'\left( b \right)>0$,结合$f\left( a \right)=f\left( b \right)=0$可知
    $$\exists {{\delta }_{1}},{{\delta }_{2}}>0,\forall x\in \left( a,a+{{\delta }_{1}} \right),f\left( x \right)>0;~\forall x\in \left( b-{{\delta }_{2}},b \right),f\left( x \right)<0$$
    于是由$f$的连续性,借助零点存在定理可知
    $$\exists \eta \in \left( a,b \right),f\left( \eta  \right)=0$$
    于是$\left[ a,b \right]$上二阶可导的函数$f$满足$f\left( a \right)=f\left( \eta  \right)=f\left( b \right)=0$,接下来只需分别在$\left[ a,\eta  \right],\left[ \eta ,b \right]$上分别对$f$使用罗尔定理得到${f}'$的两个不同零点${{\xi }_{1}},{{\xi }_{2}}$,再在$\left[ {{\xi }_{1}},{{\xi }_{2}} \right]\subseteq \left[ a,b \right]$上对${f}'$使用罗尔定理即可证.
  }

  \Example{%
    2016-2017-1-期末-证明题-18
  }{%
    设函数$f(x),g(x)$在$\left[ a,b \right]$上连续,在$\left( a,b \right)$内具有二阶导数且存在相等的最大值,$f(a)=g(a),f(b)=g(b)$,证明:存在$\xi \in (a,b)$,使得$f''(\xi )=g''\left( \xi  \right)$.
  }{%
    见解析.
  }{%
    构造函数$F(x)=f(x)-g(x)$,
    由题$F(a)=F(b)=0$

    设$c$处$f(x)$取最大值,$d$处$g(x)$取最大值,且有
    $$f(c) = g(d), F(c)=f(c)-g(c)>f(c)-g(d)=0,F(d)=f(d)-g(d)<f(c)-g(d)=0$$
    则$\exists \xi '\in \left( c,d \right)\subseteq \left( a,b \right)\,s.t.\,F\left( \xi ' \right)=0$,
    又$F(a)=F(b)=0$则
    $$\exists {{\xi }_{1}}\in \left( a,\xi ' \right)\,s.t.\,F'\left( {{\xi }_{1}} \right)=0,\exists {{\xi }_{2}}\in \left( \xi ',b \right)\,s.t.\,F'\left( {{\xi }_{2}} \right)=0\Rightarrow \exists \xi \in \left( {{\xi }_{1}},{{\xi }_{2}} \right)\subseteq \left( a,b \right)\,s.t.\,F''\left( \xi  \right)=0$$
    即$f''(\xi )=g''\left( \xi  \right)$,得证.
  }

  \Example{%
    2012-2013-1-期中-证明题-21
  }{%
    设函数$f\left( x \right)$在$\left[ 0,1 \right]$上具有二阶导数,且$f\left( 0 \right)=f\left( 1 \right)=0,\min f\left( x \right)=-1,\left( 0\le x\le 1 \right)$,
    证明:至少存在一点$\xi \in \left( 0,1 \right)$,使得$f'''\left( \xi  \right)\ge 8$.
  }{%
    见解析.
  }{%
    由题$f\left( 0 \right)=f\left( 1 \right)=0\ne -1$,设$\min f(x)=f(c)=-1$;则$f'(c)=0$$c\in \left( 0,1 \right)$

    因为$f(x)$在$\left[ 0,1 \right]$上具有二阶函数,且$f(c)=-1$

    所以可将$f(x)$在$x=c$处展开成二阶带拉格朗日余项的泰勒公式,并分别带入$x=0$和$x=1$

    有
    $\begin{cases}
      f(0)=f(c)+f'(c)(0-c)+\dfrac{1}{2}f''({{\xi }_{1}}){{(0-c)}^{2}}, & 0<{{\xi }_{1}}<c \\
      f\left( 1 \right)=f\left( c \right)+f'\left( c \right)(1-c)+\dfrac{1}{2}f''\left( {{\xi }_{2}} \right){{\left( 1-c \right)}^{2}}, & c<{{\xi }_{2}}<1 \\
    \end{cases}$,
    即$\begin{cases}
      f''\left( {{\xi }_{1}} \right)=\dfrac{2}{{{c}^{2}}} \\
      f''\left( {{\xi }_{2}} \right)=\dfrac{2}{{{\left( 1-c \right)}^{2}}} \\
    \end{cases}$

    又$c\in \left( 0,1 \right)$,所以存在$\xi \in \left( {{\xi }_{1}},{{\xi }_{2}} \right)$使得$\max f''\left( \xi  \right)\ge \left\{ f''\left( {{\xi }_{1}} \right),f''\left( {{\xi }_{2}} \right) \right\}\ge \dfrac{2}{\frac{1}{4}}=8$,
    得证.
  }

  \Example{%
    2014-2015-1-期中-证明题-21
  }{%
    设函数$f\left( x \right)$在区间$\left[ 0,1 \right]$上连续,在$\left( 0,1 \right)$内可导,且$f\left( 0 \right)=f\left( 1 \right)=0,f\left( \frac{1}{2} \right)=1$.
    试证:

    (1)存在$\eta \in \left( \frac{1}{2},1 \right)$使$f\left( \eta  \right)=\eta $;

    (2)对于任意实数$\lambda $,必存在$\xi \in \left( 0,\eta  \right)$使得$f'\left( \xi  \right)-\lambda \left[ f\left( \xi  \right)-\xi  \right]=1$.
  }{%
    见解析.
  }{%
    (1)令$g\left( x \right)=f\left( x \right)-x$,则$g\left( x \right)$在$\left[ \frac{1}{2},1 \right]$连续,在$\left( \frac{1}{2},1 \right)$可导

    且$g\left( 1 \right)=f\left( 1 \right)-1=0-1=-1<0$,$g\left( \frac{1}{2} \right)=f\left( \frac{1}{2} \right)-\frac{1}{2}=1-\dfrac{1}{2}=\frac{1}{2}>0$

    $\therefore $由零点定理,$\exists \eta \in \left( \frac{1}{2},1 \right)$,使得$g\left( \eta  \right)=0$,即$f\left( \eta  \right)=\eta $,得证.

    (2)设$h\left( x \right)={{\re}^{-\lambda x}}\left[ f\left( x \right)-x \right],x\in \left[ 0,\eta  \right]$,则$h\left( x \right)$在$\left[ 0,\eta  \right]$连续,在$\left( 0,\eta  \right)$可导,且$h\left( 0 \right)=h\left( \eta  \right)=0$

    $\therefore $由罗尔定理,$\exists \xi \in \left( 0,\eta  \right)$使得$h'\left( \xi  \right)=0$,又$h'\left( x \right)={{\re}^{-\lambda x}}\left[ f'\left( x \right)-1-\lambda \left( f\left( x \right)-x \right) \right]$

    $\therefore $由$h'\left( \xi  \right)=0$得:${{\re}^{-\lambda \xi }}\left[ f'\left( \xi  \right)-1-\lambda \left( f\left( \xi  \right)-\xi  \right) \right]=0$

    $\therefore $$f'\left( \xi  \right)-\lambda \left[ f\left( \xi  \right)-\xi  \right]=1$,得证.
  }

  \Example{%
    2015-2016-1-期中-证明题-22
  }{%
    设函数$f\left( x \right)$是区间$\left[ -1,1 \right]$上的三阶可导函数,且$f\left( -1 \right)=0,f\left( 0 \right)=0,f\left( 1 \right)=1,f'\left( 0 \right)=0.$试证:$\exists \xi \in \left( -1,1 \right),$使得$f'''\left( \xi  \right) \ge 3$.
  }{%
    见解析.
  }{%
    泰勒展开$f\left( x \right)=f\left( 0 \right)+f'\left( 0 \right)x+\dfrac{f''\left( 0 \right)}{2}{{x}^{2}}+\dfrac{f'''\left( c \right)}{6}{{x}^{3}}=\dfrac{f''\left( 0 \right)}{2}{{x}^{2}}+\dfrac{f'''\left( c \right)}{6}{{x}^{3}}$

    $\therefore f\left( 1 \right)=\dfrac{f''\left( 0 \right)}{2}+\dfrac{f'''\left( {{\xi }_{1}} \right)}{6},f\left( -1 \right)=\dfrac{f''\left( 0 \right)}{2}-\dfrac{f'''\left( {{\xi }_{2}} \right)}{6}$,
    两式相减并整理得到$f'''\left( {{\xi }_{1}} \right)+f'''\left( {{\xi }_{2}} \right)=6$

    取$f\left( \xi  \right)=\max \{f\left( {{\xi }_{1}} \right),f\left( {{\xi }_{2}} \right)\}3$即得证.
  }

  \Example{%
    2017-2018-1-期中-填空题-6
  }{%
    $y=\sqrt{x}-1$在区间$\left[ 1,4 \right]$上用拉格朗日中值定理,结论中的点$\xi $=\fillin{$\dfrac{9}{4}$}.
  }{%
    $\dfrac{9}{4}$.
  }{%
    由拉格朗日中值定理得,$\dfrac{1}{2\sqrt{\xi }}=\dfrac{y\left( 4 \right)-y\left( 1 \right)}{4-1}=\dfrac{\sqrt{4}-\sqrt{1}}{3}=\dfrac{1}{3}$$\Rightarrow \sqrt{\xi }=\dfrac{3}{2}\Rightarrow \xi =\dfrac{9}{4}$.
  }

  \Example{%
    2017-2018-1-期中-证明题-2
  }{%
    设$f\left( x \right)$在$\left[ 0,a \right]$存在三阶导数,且$f\left( 0 \right)=f\left( a \right)=0$,设$F\left( x \right)={{x}^{3}}f\left( x \right)$.证明:存在一点$\xi \in \left( 0,a \right)$,使得$F'''\left( \xi  \right)=0$.
  }{%
    见解析.
  }{%
    $F'\left( x \right)=3{{x}^{2}}f\left( x \right)+{{x}^{3}}f'\left( x \right),F''\left( x \right)=6xf\left( x \right)+6{{x}^{2}}f'\left( x \right)+{{x}^{2}}f''\left( x \right)$

    $\Rightarrow F\left( 0 \right)=F\left( a \right)=0,F'\left( 0 \right)=0,F''\left( 0 \right)=0$

    $\Rightarrow F\left( 0 \right)=F\left( a \right)=0\Rightarrow \exists b\in \left( 0,a \right)\,s.t.\,F'\left( b \right)=0$,又$F'\left( 0 \right)=0$

    $\Rightarrow \exists c\in \left( 0,b \right)\,s.t.\,F''\left( c \right)=0$,又$F''\left( 0 \right)=0$

    $\Rightarrow \exists \xi \in \left( 0,c \right)\subseteq \left( 0,a \right) \,s.t.\, F'''\left( \xi  \right)=0$,得证.
  }
