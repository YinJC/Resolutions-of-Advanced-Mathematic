% !TEX root = ../HTNotes-Demo.tex
% !TEX program = xelatex
% 内容开始
\type{微分方程的基本概念与解的结构}
注意:这一节尽管安排在最前面,却需要读者在学习完整章之后再回过头来复习、总结.

ID:2013-2014-2-期末-选择题-7
题目:
微分方程$\frac{\text{d}y}{\text{d}x}\text{+}\sin \frac{x+y}{2}=\sin \frac{x-y}{2}$是(    )
A.可分离变量方程
B.齐次方程
C.一阶线性方程
D.伯努利方程
答案:A
解析:利用和差化积公式进行化简即可得到。

可分离变量的微分方程
ID: 2011-2012-2-期末-选择题-12;2012-2013-2-期末-填空题-1
题目:
设$y={{e}^{x}}$是微分方程$x{y}'+p\left( x \right)y=x$的一个解,则该方程的通解为(    )
A.$y={{e}^{x}}+C{{e}^{x+{{e}^{-x}}}}$
B.$y={{e}^{x}}+{{e}^{x+{{e}^{-x}}}}\text{+}C$
C.$y=C{{e}^{x+{{e}^{-x}}}}$
D.$y={{e}^{x}}+{{e}^{x+{{e}^{-x}}}}$
答案:A
解析:将特解$y={{e}^{x}}$代入$x{y}'+p\left( x \right)y=x$并整理得到$p\left( x \right)=x\left( {{e}^{-x}}-1 \right)$,回代微分方程,得
	\[{y}'+\left( {{\text{e}}^{-x}}-1 \right)y=1\] 
对应齐次方程${y}'+\left( {{e}^{-x}}-1 \right)y=0$是可分离变量方程,解得
	\[\frac{\text{d}y}{y}=\left( 1-{{e}^{-x}} \right)\text{d}x\Rightarrow \ln \left| y \right|=x+{{e}^{-x}}+C\Rightarrow y=C{{e}^{x+{{e}^{-x}}}}\] 
则原方程的通解为$y=C{{e}^{x+{{e}^{-x}}}}+{{e}^{x}}$ 

ID:2014-2015-2-期末-解答题-15
题目:
设函数$y=f\left( x \right)$在区间\[\left[ 0+\infty  \right)\]具有连续的导数,且满足方程\[x\int_{0}^{x}{f\left( t \right)\text{d}t}=\left( x+1 \right)\int_{0}^{x}{tf\left( t \right)\text{d}t}\],求$y=f\left( x \right)$.
答案:$y=C\frac{{{e}^{\frac{1}{x}}}}{{{x}^{3}}}$
解析:
方程两边同时对$x$求导,得
	$\int_{0}^{x}{f\left( t \right)\text{d}t}+xf(x)=x\left( x+1 \right)f\left( x \right)+\int_{0}^{x}{tf\left( t \right)\text{d}t}\Rightarrow \int_{0}^{x}{f\left( t \right)\text{d}t}=\int_{0}^{x}{tf\left( t \right)\text{d}t}+{{x}^{2}}f\left( x \right)$,
再次求导,得\[f\left( x \right)=xf\left( x \right)+2xf\left( x \right)+{{x}^{2}}{f}'\left( x \right)\],即\[{{x}^{2}}{y}'+\left( 3x-1 \right)y=0\]. 注意到这是一个可分离变量的方程,并且可分离为$\frac{\text{d}y}{y}=\frac{1-3x}{{{x}^{2}}}\text{d}x$,解得$\ln \left| y \right|=\frac{1}{x}-3\ln \left| x \right|+C\Rightarrow y=C\frac{{{e}^{\frac{1}{x}}}}{{{x}^{3}}}$.

一阶线性微分方程
ID:2013-2014-2-期末-填空题-1
题目:微分方程${y}'+y={{e}^{-x}}$的通解为(    ).
答案:$y=C{{e}^{-x}}+x{{e}^{-x}}$
解析:
对应齐次方程${y}'+y=0$,其通解为$y=C{{e}^{-x}}$
设方程的一个特解${{y}^{*}}=ax{{e}^{-x}}$
代入方程,得$a{{e}^{-x}}-ax{{e}^{-x}}+ax{{e}^{-x}}={{e}^{-x}}$,即\[a{{e}^{-x}}={{e}^{-x}}\]
解得$a=1$
故原方程通解为$y=C{{e}^{-x}}+x{{e}^{-x}}$

全微分方程
ID:2013-2014-2-期末-填空题-3
题目:若\[2xy\text{d}x+{{x}^{2}}\text{d}y\]在整个$xOy$平面是某二元函数$u\left( x,y \right)$的全微分,则这样的一个$u\left( x,y \right)=$(    ).
答案:${{x}^{2}}y$
解析:观察可得\[2xy\text{d}x+{{x}^{2}}\text{d}y=\left( {{x}^{2}}y \right)\text{d}u\],故$u={{x}^{2}}y$

ID:2013-2014-2-期末-填空题-1
题目:若\[x{{y}^{2}}\text{d}x+{{x}^{2}}y\text{d}y\]是某个函数$u\left( x,y \right)$的全微分,则一个这样的函数$u\left( x,y \right)=$(    ).
答案:$\frac{1}{2}{{x}^{2}}{{y}^{2}}$
解析:$x{{y}^{2}}\text{d}x+{{x}^{2}}y\text{d}y=xy\left( y\text{d}x+x\text{d}y \right)=xy\text{d}\left( xy \right)=\text{d}\left( \frac{1}{2}{{x}^{2}}{{y}^{2}} \right)$

ID:2012-2013-2-期末-解答题-14
题目:
验证$\frac{x\text{d}x+y\text{d}y}{{{x}^{2}}+{{y}^{2}}}$在整个$xOy$平面除去$y$的负半轴及原点的开区域$G$内是某个二次函数的全微分,并求出一个这样的二元函数。
答案:$u\left( x,y \right)=\frac{1}{2}\ln \left( {{x}^{2}}+{{y}^{2}} \right)$
解析:
因为\[P=\frac{x}{{{x}^{2}}+{{y}^{2}}},Q=\frac{y}{{{x}^{2}}+{{y}^{2}}}\],区域$G$单连通,\[P,Q\]在$G$内具有一阶连续偏导数,且$\frac{\partial P}{\partial y}=\frac{-2xy}{{{\left( {{x}^{2}}+{{y}^{2}} \right)}^{2}}}=\frac{\partial Q}{\partial x}$,所以\[\frac{x\text{d}x+y\text{d}y}{{{x}^{2}}+{{y}^{2}}}\]在整个$xOy$平面除去$y$的负半轴及原点的开区域$G$内是某个二元函数的全微分. 记$A\left( 1,0 \right),B\left( x,0 \right),C\left( x,y \right)$并选取积分路径$A\to B,B\to C$,得
	\[\begin{align}
  & u\left( x,y \right)=\int_{\left( 1,0 \right)}^{\left( x,y \right)}{\frac{x\text{d}x+y\text{d}y}{{{x}^{2}}+{{y}^{2}}}}=\int_{AB}{+}\int_{BC}{\frac{x\text{d}x+y\text{d}y}{{{x}^{2}}+{{y}^{2}}}} \\ 
 & =\int_{1}^{x}{\frac{1}{x}\text{d}x}+\int_{0}^{y}{\frac{y\text{d}y}{{{x}^{2}}+{{y}^{2}}}} \\ 
 & =\frac{1}{2}\ln \left( {{x}^{2}}+{{y}^{2}} \right).  
\end{align}\] 


特殊的高阶微分方程
ID:2012-2013-2-期末-解答题-11
题目:求微分方程$\left( 1+{{x}^{2}} \right){y}''=2x{y}'$的通解。
答案:$y={{C}_{1}}\left( x+\frac{1}{3}{{x}^{3}} \right)+{{C}_{2}}$
解析:方程中没有$y$,可降阶. 设${y}'=p$,带入原方程得$\frac{\text{d}p}{p}=\frac{2}{1+{{x}^{2}}}\text{d}x$,解之得$y={{C}_{1}}\left( x+\frac{1}{3}{{x}^{3}} \right)+{{C}_{2}}$.

高阶线性微分方程
ID:2013-2014-2-期末-填空题-3
题目:微分方程${y}''+{y}'-2y=0$的通解为(    ).
答案:$y={{C}_{1}}{{e}^{x}}+{{C}_{2}}{{e}^{2x}}+C$
解析:解特征方程:${{\lambda }^{2}}+\lambda -2=0$得特征根${{\lambda }_{1}}=1,{{\lambda }_{2}}=-2$,则通解为$y={{C}_{1}}{{e}^{x}}+{{C}_{2}}{{e}^{-2x}}$.

ID:2013-2014-2-期末-选择题-6
题目:
微分方程${y}''-y={{e}^{x}}+1$的一个特解应具有形式(    )
A.$a{{e}^{x}}+b$
B.$ax{{e}^{x}}+b$
C.$a{{e}^{x}}+bx$
D.$ax{{e}^{x}}+bx$
答案:B
解析:特征方程${{\lambda }^{2}}-1=0$,特征根为${{\lambda }_{1,2}}=\pm 1$,对应齐次方程的通解为$y={{C}_{1}}{{e}^{x}}+{{C}_{2}}{{e}^{-x}}$. 设其特解为
	\[{{y}^{*}}=ax{{e}^{x}}+bx+c\] 
求出各阶导数${y}'=a{{e}^{x}}+ax{{e}^{x}}+b,{y}''=2a{{e}^{x}}+ax{{e}^{x}}$,代入微分方程,得$2a{{e}^{x}}-bx-c={{e}^{x}}+1$,解得
	\[a=\frac{1}{2},b=0,c=-1\] 
故特解为${{y}^{*}}=\frac{1}{2}x{{e}^{x}}-1$,选项中只有B符合.

ID:2010-2011-2-期末-解答题-17
题目:设$f\left( x \right)$为一连续函数,且满足方程$f\left( x \right)=\sin x-\int_{0}^{x}{\left( x-t \right)f\left( t \right)\text{d}t}$,求$f\left( x \right)$.
答案:$\frac{1}{2}\sin x+\frac{1}{2}x\cos x$
解析:由原方程知$f\left( 0 \right)=0$,且有$f\left( x \right)=\sin x-x\int_{0}^{x}{f\left( t \right)\text{d}t}+\int_{0}^{x}{tf\left( t \right)\text{d}t}$.两边对$x$求导,得
	\[{f}'\left( x \right)=\cos x-\int_{0}^{x}{f\left( t \right)\text{d}t}\] 
知${f}'\left( 0 \right)=1$. 两边再对$x$求导,得${f}''\left( x \right)+f\left( x \right)=-\sin x$,由此可知这是二阶线性微分方程. 由其特征方程
	${{\lambda }^{2}}+1=0$ 
得特征根${{\lambda }_{1,2}}=\pm i$,又\[\lambda +\omega \text{i}=\text{i}\]为方程的单根,故设特解\[{{f}^{*}}\left( x \right)=x\left( A\cos +B\sin x \right)\],得$A=\frac{1}{2},B=0$,于是${{f}^{*}}=\frac{1}{2}x\cos $,从而通解$f\left( x \right)={{C}_{1}}\cos x+{{C}_{2}}\sin x+\frac{1}{2}x\cos x$.再由$f\left( 0 \right)=0,{f}'\left( 0 \right)=1$,得${{C}_{1}}=0,{{C}_{2}}=\frac{1}{2}$,故所求函数为$f\left( x \right)=\frac{1}{2}\sin x+\frac{1}{2}x\cos x$.

ID:2012-2013-2-期末-解答题-15
题目:设函数$y\left( x \right)$具有二阶导数,且满足方程${y}'\left( x \right)-2y\left( x \right)+\int_{0}^{x}{y\left( t \right)\text{d}t={{x}^{2}}}$,且$y\left( 0 \right)=1$,求$y\left( x \right)$.
答案:$y\left( x \right)=\left( -3+3x \right){{e}^{x}}+2x+4$
解析:
原方程两边对$x$求导得${y}''-2{y}'+y=2x$,是二阶线性非齐次方程. 解特征方程${{r}^{2}}-2r+1=0$,得$r=1$(二重根),所以相应齐次方程的通解为\[\bar{y}=\left( {{C}_{1}}+{{C}_{2}}x \right){{e}^{x}}\]. 
设非齐次方程${y}''-2{y}'+y=2x$的特解为${{y}^{*}}=ax+b$,代入方程,求得${{y}^{*}}=2x+4$,故原方程的通解为
	\[y=\left( {{C}_{1}}+{{C}_{2}}x \right){{e}^{x}}+2x+4\] 
再由$y\left( 0 \right)=1,{y}'\left( 0 \right)=2$,得${{C}_{1}}=-3,{{C}_{2}}=3$,于是\[y\left( x \right)=\left( -3+3x \right){{\text{e}}^{x}}+2x+4\].

ID:2013-2014-2-期末-解答题-12
题目:求微分方程${y}''-5{y}'+6y=x{{e}^{2x}}$的通解。
答案:$y={{C}_{1}}{{e}^{2x}}+{{C}_{2}}{{e}^{3x}}+x\left( -\frac{1}{2}x-1 \right){{e}^{2x}}+C$
解析:
首先求${{x}^{2}}-5x+6=0$,得${{x}_{1}}=2,{{x}_{2}}=3$,因为有一个根是2(同${{e}^{2x}}$的2),特解形式为:
${{y}^{\text{*}}}=x\left( {{b}_{1}}x+{{b}_{2}} \right){{e}^{2x}}$,然后分别求其导数${{y}^{*}}^{\prime }$ 、二阶导数${{y}^{*}}^{\prime \prime }$
$\begin{align}
  & {{y}^{*}}^{\prime }=\left[ 2{{b}_{1}}{{x}^{2}}+2\left( {{b}_{1}}+{{b}_{2}} \right)x+{{b}_{2}} \right]{{e}^{2x}} \\ 
 & {{y}^{*}}^{\prime \prime }\text{=}\left[ 4{{b}_{1}}{{x}^{2}}+\left( 8{{b}_{1}}+4{{b}_{2}} \right)x+2{{b}_{1}}+4{{b}_{2}} \right]{{e}^{2x}} \\ 
\end{align}$
代入原式,比较系数,求解线性方程组,得到${{b}_{1}}=-\frac{1}{2},{{b}_{2}}=-1$;
则特解:${{y}^{*}}=x\left( -\frac{1}{2}x-1 \right){{e}^{2x}}$
通解:$y={{C}_{1}}{{e}^{2x}}+{{C}_{2}}{{e}^{3x}}+x\left( -\frac{1}{2}x-1 \right){{e}^{2x}}+C$.

ID:2014-2015-2-期末-解答题-11
题目:求微分方程${y}''-3{y}'+2y=2x{{e}^{x}}$的通解。
答案:$y={{C}_{1}}{{e}^{x}}+{{C}_{2}}{{e}^{2x}}-x\left( x+2 \right)x{{e}^{x}}$
解析:
首先求${{x}^{2}}-3x+2=0$,得${{x}_{1}}=1,{{x}_{2}}=2$,其特解形式为:
${{y}^{\text{*}}}=x\left( {{b}_{1}}x+{{b}_{2}} \right){{e}^{x}}$,然后分别求其导数${{y}^{*}}^{\prime }$ 、二阶导数${{y}^{*}}^{\prime \prime }$
$\begin{align}
  & {{y}^{*}}^{\prime }=\left[ {{b}_{1}}{{x}^{2}}+\left( 2{{b}_{1}}+{{b}_{2}} \right)x+{{b}_{2}} \right]{{e}^{x}} \\ 
 & {{y}^{*}}^{\prime \prime }\text{=}\left[ {{b}_{1}}{{x}^{2}}+\left( 4{{b}_{1}}+{{b}_{2}} \right)x+2{{b}_{1}}+2{{b}_{2}} \right]{{e}^{2x}} \\ 
\end{align}$
代入原式,解线性方程组,得到${{b}_{1}}=-1,{{b}_{2}}=-2$;
则特解:${{y}^{*}}=x\left( -x-2 \right){{e}^{2x}}$
通解:$y={{C}_{1}}{{e}^{x}}+{{C}_{2}}{{e}^{2x}}-x\left( x+2 \right)x{{e}^{x}}$.

ID:2010-2011-2-期末-填空题-6
题目:函数${{y}_{1}}={{\text{e}}^{x}},{{y}_{2}}=2x$所满足的阶数最低的常系数齐次线性微分方程为			.
答案:${y}'''-{y}''=0$
解析:由特解可知该方程的特征方程有特征根${{\lambda }_{1}}=1,{{\lambda }_{2,3}}=0$,于是次数最低的特征方程为
	\[\left( \lambda -1 \right){{\lambda }^{2}}=0\],
反推微分方程得到${y}'''-{y}''=0$.

其他
ID:2013-2014-2-期末-解答题-17
题目:
设$u=f\left( \ln \sqrt{{{x}^{2}}+{{y}^{2}}+{{z}^{2}}} \right)$有二阶连续偏导数,且满足方程$\frac{{{\partial }^{2}}u}{\partial {{x}^{2}}}+\frac{{{\partial }^{2}}u}{\partial {{y}^{2}}}+\frac{{{\partial }^{2}}u}{\partial {{z}^{2}}}={{\left( {{x}^{2}}+{{y}^{2}}+{{z}^{2}} \right)}^{-\frac{3}{2}}}$,求函数$u$.
答案:$u={{e}^{-v}}\left( {{C}_{1}}-v \right)+{{C}_{2}}$
解析:
令$v=\ln \sqrt{{{x}^{2}}+{{y}^{2}}+{{z}^{2}}},w={{x}^{2}}+{{y}^{2}}+{{z}^{2}}$,则$w={{e}^{2v}},u=f\left( \ln \sqrt{{{x}^{2}}+{{y}^{2}}+{{z}^{2}}} \right)=f\left( \frac{\ln w}{2} \right)$,于是
	\[\frac{\partial u}{\partial x}=\frac{v}{w}{f}'\left( v \right),\frac{{{\partial }^{2}}u}{\partial {{x}^{2}}}=\frac{{{x}^{2}}}{{{w}^{2}}}{f}'\left( v \right)+\frac{w-2{{x}^{2}}}{{{w}^{2}}}{f}''\left( v \right)\],
同理可得
	\[\frac{{{\partial }^{2}}u}{\partial {{y}^{2}}}=\frac{{{y}^{2}}}{{{w}^{2}}}{f}'\left( v \right)+\frac{w-2{{y}^{2}}}{{{w}^{2}}}{f}''\left( v \right),\frac{{{\partial }^{2}}u}{\partial {{z}^{2}}}=\frac{{{z}^{2}}}{{{w}^{2}}}{f}'\left( v \right)+\frac{w-2{{z}^{2}}}{{{w}^{2}}}{f}''\left( v \right)\],
代入$\frac{{{\partial }^{2}}u}{\partial {{x}^{2}}}+\frac{{{\partial }^{2}}u}{\partial {{y}^{2}}}+\frac{{{\partial }^{2}}u}{\partial {{z}^{2}}}={{\left( {{x}^{2}}+{{y}^{2}}+{{z}^{2}} \right)}^{-\frac{3}{2}}}$,得\[\frac{{f}''\left( v \right)}{w}+\frac{{f}'\left( v \right)}{w}={{w}^{-\frac{3}{2}}}\]. 化简得${f}''\left( v \right)+{f}'\left( v \right)={{e}^{-v}}$,即\[{u}''+{u}'={{e}^{-v}}\].
则该方程通解为$u={{e}^{-v}}\left( {{C}_{1}}-v \right)+{{C}_{2}}$.
