% !TEX root = ../HTNotes-Demo.tex
\begin{flushright}
  \color{zhanqing!80}
  \ding{43} 习题见\autopageref{cha:10}
\end{flushright}
\section{微分方程的基本概念}
\begin{flushright}
  \color{zhanqing!80}
  \ding{43} 教材见348页
\end{flushright}
  2.~(1)解:将$y=C_1 \cos \omega x$代入左边得:$y''+\omega ^2y=-C_1\omega ^2\cos \omega x+C_1\omega ^2\cos \omega x=0$

  符合原方程,故$y=C_1 \cos \omega x$是所给方程$y''+\omega ^2y=0$.

  (3)解:将$y=3s \ln x-4\cos x$代入方程左边得$y''+y=-3s \ln x+4\cos x+3s \ln x-4\cos x=0$

  符合原方程,故$y=3s \ln x-4\cos x$是所给方程$y''+y=0$的解.

  (5)解:$y=x^2e^x$求导得$y'=2xe^x+x^2e^x$,再次求导得$y''=2e^x+2xe^x+2xe^x+x^2e^x=2e^x+4xe^x+x^2e^x$

  代入左边得$y''-2y'+y=2e^x+4xe^x+x^2e^x-4xe^x-2x^2e^x+x^2e^x=2e^x \ne 0$

  故$y=x^2e^x$不是所给方程$y''-2y'+y=0$的解.

  3.~(1)解:设所求曲线方程为$y=y(x)$

  根据题意可知函数应满足如下关系:$\dfrac{{\rd y}}{{\rd x}}=x^2$,即$y'=x^2$.

  (3)解:设所求曲线方程为$y=y(x)$,该曲线在$x$处斜率为$y'$,可得切线方程为$y=y'x+s$

  又根据题意,$s=x$,故$y=y'x+x$,也即$xy'=y-x$.

  (5)解:设所求曲线方程为$y=y(x)$,该曲线在$x$处斜率为$y'$

  设其在$(x,y)$处切线的截距为$b$,则切线方程为$y=y'x+b$

  又根据题意,$b=\dfrac{x+y}{2}$,故$y=y'x+\dfrac{x+y}{2}$,也即$y-xy'=\dfrac{x+y}{2}$.

  4.解:设物体与空气的温差是$T$,在冷却过程中所需时间$t$是温差$T$的函数$t=T(t)$

  设其冷却速度为$v$,根据题意,设比例系数为$k$,则$v=kT$

  又因为冷却表示为温度的下降,即$v$是表示$T$递减速率的函数,故$v=-\dfrac{\rd T}{\rd t}$

  联立以上各式可得$-\dfrac{\rd T}{\rd t}=kT$

  解此微分方程可得:$t=-\dfrac{ \ln kT}{k}+C$($C$为常数)

  根据题意,物体在20分钟内由100度冷却到60度,温差$T$应为物体温度减去空气温度

  开始时$\begin{cases}t=0\\T=100-20=80\end{cases}$, 结束时$\begin{cases}t=20\\T=60-20=40\end{cases}$

  代入$t$与$T$的函数关系式,得$\begin{cases}-\dfrac{ \ln 80k}{k}+C=0\\-\dfrac{ \ln 40k}{k}+C=20\end{cases}$,解方程得$\begin{cases}k=\dfrac{ \ln 2}{20}\\C=-\dfrac{20 \ln (4 \ln 2)}{ \ln 2}\end{cases}$

  故$t=\dfrac{20}{ \ln 2} \ln \dfrac{80}{T}~$(单位:分钟), 则物体达到30度时,需要60分钟.


\section{可变量分离的微分方程}
\begin{flushright}
  \color{zhanqing!80}
  \ding{43} 教材见354页
\end{flushright}
  1.~(2)解:分离变量得$\dfrac{ \tan y }{ \ln ( \cos y )}\rd y = -\dfrac{1}{x}\rd x$
  $\Rightarrow -\dfrac{s \ln y}{ \cos y  \ln ( \cos y )}\rd y =\dfrac{1}{x}\rd x$

  两边积分得$\ln [ \ln ( \cos y )] = \ln x+C$
  $\Rightarrow \ln ( \cos y ) =xe^C$
  $\Rightarrow \cos y =e^{xe^C}$

  令$C_1=e^C, \cos y =e^{xC_1},~y=\arccos e^{C_1x}$,即$y=\arccos e^{Cx}$.

  (4)解:分离变量得$\dfrac{y}{\sqrt {1{ + }y^2} }\rd y=-\dfrac{x}{\sqrt {1{ + }x^2} }\rd x$

  两边积分得$\sqrt {1{ + }y^2} =-\sqrt {1{ + }x^2} +C$,即$\sqrt {1{ + }x^2} +\sqrt {1{ + }y^2} =C$.

  (6)解:分离变量得$\dfrac{5e^x}{1-e^x}\rd x=-\dfrac{\sec^2y}{ \tan y }\rd y$,
  两边积分得$-5 \ln (1-e^x)+C = - \ln ( \tan y )$

  $\Rightarrow (1-e^x)^5e^{-C} = \tan y$,
  令$C_1=e^{-C}$

  故$(1-e^x)^5C_1= \tan y ~~,y=\arctan C_1(1-e^x)^5$,即$y=\arctan C(1-e^x)^5$.

  (8)解:变换得$e^x(e^y-1)\rd x+e^y(e^x+1)\rd y=0$,
  分离变量得$\dfrac{e^x}{e^x+1}\rd x=-\dfrac{e^y}{e^y-1}\rd y$

  两边积分得
  $$\ln (e^x+1)+C =- \ln (e^y-1) \Rightarrow (e^x+1)e^C =\dfrac{1}{e^y-1}$$
  令$C_1=\dfrac{1}{e^C}$,
  $C_1=(e^x+1)(e^y-1)$,即$(e^x+1)(e^y-1)=C$.

  (10)解:变换得$y'+s \ln x \cos y + \cos x s \ln y =s \ln x \cos y - \cos x s \ln y$
  $$\Rightarrow y'+2\cos x s \ln y =0 \Rightarrow \dfrac{\rd y}{\rd x}+2\cos x \ln y=0$$
  两边积分得$ \ln \left| {\tan \dfrac{y}{2}} \right|=-2s \ln x+C$,故$2s \ln x+ \ln \left| {\tan \dfrac{y}{2}} \right|=C$.

  2.~(1)解:$\dfrac{\rd y}{\rd x}s \ln x=y \ln y$,分离变量得
  $$\dfrac{\rd y}{y \ln y}=\dfrac{\rd x}{s \ln x}$$
  两边积分得
  $\ln(\ln y) =\ln(\tan\dfrac{x}{2})+C$
  $\Rightarrow \ln y =\tan\dfrac{x}{2}e^C$

  代入初始条件得$C=0$,故$\ln y=\tan\dfrac{x}{2}$.

  (3)解:$\dfrac{\rd y}{\rd x}\cos x=\dfrac{y}{ \ln y}$,
  分离变量得$\dfrac{\rd x}{\cos x}=-\dfrac{ \ln y}{y}\rd y$,
  两边积分得
  $$\ln \left| {\tan (\dfrac{x}{2}+\frac{\pi }{4}}) \right|+C=\dfrac{1}{2} \ln ^2y$$

  代入初始条件得$C=0$,故$ \ln \left| \tan \left(\dfrac{x}{2}+\dfrac{\pi }{4}\right) \right|=\dfrac{1}{2} \ln ^2y$.

  (5)解:分离变量得$\dfrac{3e^x\rd x}{1+e^x}=-\dfrac{\sec^2y}{ \tan y }\rd y$,
  两边积分得$3 \ln (1+e^x)+C=- \ln ( \tan y )$

  变换得$(1+e^x)^3 \tan y e^C=0$,
  代入初始条件得$C=-3 \ln 2$,故$(1+e^x)^3 \tan y =8$.

  (7)解:$x\dfrac{\rd y}{\rd x}+x+s \ln (x+y) = 0$,
  变换得
  $$x(\dfrac{\rd y}{\rd x}+1)+s \ln (x+y) = 0 \Rightarrow x\dfrac{\rd y+\rd x}{\rd x}+s \ln (x+y)=0$$

  注意到$\dfrac{\rd y+\rd x}{\rd x}=\dfrac{\rd (x+y)}{\rd x}$,
  因此令$u=x+y$,故$x\dfrac{\rd u}{\rd x}+s \ln u=0$

  分离变量得$\dfrac{\rd x}{x}=-\dfrac{\rd u}{s \ln u}$,
  两边积分得
  $$\ln x+C=- \ln \left| {\tan \dfrac{u}{2}} \right|\Rightarrow xe^C=\dfrac{1}{\tan \frac{u}{2}}$$
  把初始条件代入得$C= \ln \dfrac{2}{\pi }$,故$x\tan\dfrac{x+y}{2}=\dfrac{2}{\pi }$.

  4.~(2)解:原方程可写为$\dfrac{\rd y}{\rd x}=\dfrac{y}{x}-\frac{1}{s \ln \dfrac{y}{x}}$,~~令$u=\dfrac{y}{x},$得$\dfrac{\rd y}{\rd x}=\dfrac{x\rd u}{\rd x}+u$

  故$\dfrac{x\rd u}{\rd x}+u=-\dfrac{1}{s \ln u}$,~~分离变量得$s \ln \rd u=-\dfrac{\rd x}{x}$

  两边积分得$\cos u+C_1= \ln x,$变形得$e^{\cos u}e^{C_1}=x$,即$x=Ce^{\cos\dfrac{y}{x}} $

  (4)解:原方程可写为$\dfrac{\rd y}{\rd x}=\dfrac{y}{x}+tan\dfrac{y}{x}$,~~令$u=\dfrac{y}{x},$得$\dfrac{\rd y}{\rd x}=\dfrac{x\rd u}{\rd x}+u$

  故$\dfrac{x\rd u}{\rd x}+u=\tan u$, 分离变量得$\dfrac{1}{tanu}\rd u=-\dfrac{\rd x}{x}$

  两边积分得$s \ln x=xe^C$,故$s \ln \dfrac{y}{x}=xC$.

  (6)解:原方程化为$\dfrac{\rd y}{\rd x}=\dfrac{2x-5y+3}{2x+4y-6}$,注意到$\Delta {\text{ = 18}} \ne {\text{0}}$

  解方程组$\begin{cases}2x-5y+3=0\\2x+4y-6=0\end{cases}$得到交点$\begin{cases}x_0=\alpha=1\\y_0=\beta=1\end{cases}$, 令$\begin{cases}x=X+\alpha=X+1\\y=Y+\beta=Y+1\end{cases}$

  原方程化为$\dfrac{\rd Y}{\rd X}=\dfrac{2X-5Y}{2X+4Y}=\dfrac{2-5\frac{Y}{X}}{2+4\frac{Y}{X}}$
  令$\dfrac{\rd Y}{\rd X}=u$

  $\dfrac{\rd Y}{\rd X}=\dfrac{X\rd u}{\rd X}+u$,代入原方程得$\dfrac{X\rd u}{\rd X}+u=\dfrac{2-5u}{2+4u}$,
  变形得$$\dfrac{1}{\frac{2-5u}{2+4u}-u}\rd u=\dfrac{\rd X}{X} \Rightarrow (\dfrac{4u+\frac{7}{2}}{2-7u-4u^2}+\dfrac{\frac{2}{3}}{4(u+2)}-\dfrac{\frac{2}{3}}{4u-1})\rd u=\dfrac{\rd X}{X}$$

  两边积分得$-\dfrac{1}{2} \ln (2-7u-4u^2)+\dfrac{1}{6} \ln (u+2)-\dfrac{1}{6} \ln (4u-1)= \ln X+C'$

  整理得$(4y-x-3)(y+2x-3)^2=C$.

  (8)解:原方程化为$\dfrac{\rd y}{\rd x}=\dfrac{x^2y^2+1}{2x^2}$,~~令$u=xy,~~\dfrac{\rd y}{\rd x}=\dfrac{\rd u}{x\rd x}+\dfrac{u}{x^2}$,代入原方程得
  $$\dfrac{\rd u}{x\rd x}+\dfrac{u}{x^2}=\dfrac{u^2+1}{2x^2} \Rightarrow \dfrac{\rd x}{x}=\dfrac{2\rd u}{(u-1)^2}$$
  两边积分,得$x=Ce^{\frac{2}{xy-1}}$.

  5.~(2)解:$\dfrac{\rd y}{\rd x}=\dfrac{y}{x}+s \ln \dfrac{y}{x}$,~~令$u=\dfrac{y}{x},\dfrac{\rd y}{\rd x}=\dfrac{x\rd u}{\rd x}+u$,则$\dfrac{\rd u}{s \ln u}=\dfrac{\rd x}{x}$

  两边积分得
  \[y=2x \arctan(xe^{C'})\]
  代入初始条件得$y=2x\arctan x$.

  (4)解:$\dfrac{\rd y}{\rd x}=-\dfrac{1+6(\frac{y}{x})^2+(\frac{y}{x})^4}{4\frac{y}{x}+4\frac{y}{x}^3}$,~~令$u=\dfrac{x}{y},\dfrac{\rd y}{\rd x}=\dfrac{x\rd u}{\rd x}+u$,~~则$\dfrac{4u+4u^3}{1+10u^2+5u^4}\rd u=-\dfrac{\rd x}{x}$

  两边积分得$(1+10u^2+5u^4)^{\frac{1}{5}}=\dfrac{e^{C'}}{x}$,代入初始条件得$x^5+10x^3y^2+5xy^4=1$.


\section{一阶线性微分方程与常数变易法}
\begin{flushright}
  \color{zhanqing!80}
  \ding{43} 教材见359页
\end{flushright}
  1.解:(1)将方程改写为$\dfrac{\rd y}{\rd x}+y \cos x=e^{-s \ln x}$,对应齐次方程$\dfrac{\rd y}{\rd x}+y \cos x=0$的通解

  齐次方程通解为$y=Ce^{-s \ln x}$,应用常数变易法,令$y=C(x)e^{-s \ln x}$,代入原方程得$C'(x)=1$

  积分后得$C(x)=x+C_1$,代入得$y=(x+C)e^{-s \ln x}$.

  (3)解:将方程改写为$\dfrac{\rd y}{y\rd x}=\dfrac{1}{x+ \ln y}$,令$u= \ln y,\dfrac{\rd u}{\rd x}=\dfrac{\rd y}{y\rd x}  $,则$\dfrac{\rd u}{\rd x}=\dfrac{1}{x+u}$

  令$t=x+u$,则$\dfrac{t\rd t}{t+1}=\rd x$,两边积分得
  $$t- \ln (t+1)=x+C' \Rightarrow x=Cy- \ln y-1.$$

  (5)解:将方程改写为$\dfrac{\rd x}{\rd y}-\dfrac{2x}{y}=-y,~~P(y)=-\dfrac{2}{y},Q(y)=-y$,
  则$x=y^2(C- \ln y)$.

  (6)解:将方程改写为$\dfrac{\rd y}{\rd x}+xy=x^3y^3$,令$z=y^{-2},~~\dfrac{\rd z}{\rd x}-2xz=-2x^3$

  代入通解得$z=Ce^{x^2}+x^2+1$,
  原方程为
  $$\dfrac{1}{y^2}=Ce^{x^2}+x^2+1.$$

  (7)解:将方程改写为$\dfrac{\rd y}{\rd x}+\dfrac{y}{x}=ay^2 \ln x$,令$z=xy,\dfrac{\rd z}{\rd x}=\dfrac{az^2 \ln x}{x}$,
  得
  $$xy\left[C-\dfrac{1}{2}( \ln x)^2\right]=1.$$

  2.~(1)解:原方程化为$\dfrac{\rd y}{\rd x}+\dfrac{y}{x}=\dfrac{e^x}{x}$,令$z=xy,\dfrac{\rd z}{\rd x}=e^x$,
  则$xy=e^x+C$,代入初始条件得$xy=e^x$.

  (3)解:$\dfrac{\rd y}{\rd x}+y \cot x=5e^{ \cos x}$,对应齐次方程$\dfrac{\rd y}{\rd x}=-y\cot x$通解为$y=\dfrac{C}{s \ln x}$

  应用常数变易法,令$y=\dfrac{C(x)}{s \ln x}$,代入得
  $$y=\dfrac{-5e^{ \cos x}+C_1}{s \ln x} \Rightarrow y=\dfrac{-5e^{ \cos x}+1}{s \ln x}.$$

  (5)解:原方程化为$\dfrac{\rd y}{\sqrt y \rd x}+\sqrt y =e^{\frac{x}{2}}$,令$z=\sqrt y,\dfrac{\rd z}{\rd x}+\dfrac{z}{2}=\dfrac{e^{\frac{x}{2}}}{2}$

  对应齐次方程$\dfrac{\rd z}{\rd x}+\dfrac{z}{2}=0$通解为$z=Ce^{-\frac{x}{2}}$,
  应用常数变易法,得$\sqrt{y} =(\dfrac{e^x}{2}+C_1)e^{-\frac{x}{2}}$

  $\Rightarrow \sqrt{y} = \dfrac{1}{2}e^{\frac{x}{2}}+e^{-\frac{x}{2}}$.


\section{全微分方程}
\begin{flushright}
  \color{zhanqing!80}
  \ding{43} 教材见364页
\end{flushright}
  1.~(2)解:$M=\dfrac{y^2}{(x-y)^2}-\dfrac{1}{x}~~~N=\dfrac{1}{y}-\dfrac{x^2}{(x-y)^2}$,
  $\dfrac{{\partial M}}{{\partial y}} = \dfrac{{\partial N}}{{\partial x}} = \dfrac{{2xy}}{{(x - y)^3}}$

  原方程化为$\dfrac{y^2\rd x-x^2\rd y}{(x-y)^2}=\rd (\dfrac{xy}{x-y})$,
  则$\dfrac{xy}{x-y}= \ln \left| {\dfrac{x}{y}} \right|+C$.

  (4)解:$M=y(x-2y),N=-x^2,~~~\dfrac{{\partial M}}{{\partial y}}  \ne  \dfrac{{\partial N}}{{\partial x}} $,方程不是全微分方程.

  (6)解:$M=x^2+y^2,N=xy,~~~\dfrac{{\partial M}}{{\partial y}}  \ne  \dfrac{{\partial N}}{{\partial x}} $,方程不是全微分方程.

  (8)解:$M=xy+\dfrac{1}{4}y^4~~N=\dfrac{1}{2}x^2-xy^3$,
  $\dfrac{{\partial M}}{{\partial y}} = \dfrac{{\partial N}}{{\partial x}} $

  原方程化为$2\rd (x^2y)+\rd (xy^4)=0$,
  则$2x^2y+xy^4+C=0$.

  2.~(1)解:原方程变形为$\rd (\dfrac{x^2}{y^3})-\rd (\dfrac{1}{y})=0$,积分得$\dfrac{x^2}{y^3}-\dfrac{1}{y}=C$.

  (3)解:$M=x+y^2~~N=-2xy$,
  $\dfrac{{\partial M}}{{\partial y}} = \dfrac{{\partial N}}{{\partial x}}$,
  原方程化为$\rd ( \ln x)-\rd (\dfrac{y^2}{x})=0$,
  积分得$x=Ce^{\frac{y^2}{x}}$.

  (5)解:$\dfrac{\frac{{\partial M}}{{\partial y}} - \frac{{\partial N}}{{\partial x}}}{-M}=-\dfrac{4}{y}$,
  $\left(\dfrac{3x^2}{y^3}-\dfrac{a}{y^2}\right)\rd x-\left(\dfrac{3x^2}{y^4}-\dfrac{2ax}{y^3}\right)\rd y=0$

  变形为$\rd (\dfrac{x^3}{y^3})+\rd (-\dfrac{ax}{y^2})=0$,得$x^3-axy=Cy^3$.

  (7)解:$\dfrac{\frac{{\partial M}}{{\partial y}} - \frac{{\partial N}}{{\partial x}}}{N}=-\dfrac{2}{x} \Rightarrow (x^2e^x+3x^2y^2)\rd x+2x^3y\rd y=0$

  $x^2e^x\rd x+\rd (x^3y^2)=0 \Rightarrow (x^2-zx+2)e^x+x^3y^2=C$.

  3.~(1)解:设所求曲线方程为$y=y(x)$,根据题意可知未知函数应满足如下关系:$\dfrac{\rd y}{\rd x}=x^2$,即$y'=x^2$.

  (3)解:设所求曲线方程为$y=y(x)$,切线方程为$y=y'x+s$,根据题意$s=x$,则$xy'=y-x$.

  (5)解:设所求曲线方程为$y=y(x)$,切线方程为$y=y'x+b$,根据题意$b=\dfrac{x+y}{2}$,故$y-xy'=\dfrac{x+y}{2}$.


\section{某些特殊类型的高阶方程}
\begin{flushright}
  \color{zhanqing!80}
  \ding{43} 教材见369页
\end{flushright}
  1.~(1)解:记$y'=p\left( y \right)$, $\therefore y''=p\dfrac\rd p\rd y$,
  原方程化为\[yp\dfrac\rd p\rd y+{{p}^{2}}=0 \Rightarrow p=\dfrac{{{C}_{0}}}{y}\]
  \[\therefore \dfrac{\rd y}{\rd x}=\dfrac{{{C}_{0}}}{y}\Rightarrow y\rd y={{C}_{0}}\rd x\]
  两侧积分得\[{{y}^{2}}={{C}_{1}}x+{{C}_{2}}\left( {{C}_{1}}=2{{C}_{0}} \right).\]

  (3)解:记$ y' = p\left( x \right) \Rightarrow y'' = p'$,
  故原式化为\[p' = x + p\]
  即\[\dfrac {\rd p}{\rd x} - p = x \]
  设积分因子$\mu  = {e^{ - x}}$,则在上式两边同乘积分因子$ \mu $得
  \[\dfrac{{\rd p}}{{{e^x}\rd x}} - \dfrac{p}{{{e^x}}} = \dfrac{x}{{{e^x}}}\]
  凑微分得\[\rd \left( {p{e^{ - x}}} \right) = x{e^{ - x}}\]
  两侧积分得\[\dfrac{{\rd y}}{{\rd x}} = p =  - x + {C_1}{e^x} - 1\]
  再次积分,得\[y =  - \dfrac{1}{2}{x^2} - x + {C_1}{e^x} + {C_2}.\]

  (5)解:记$y' = p\left( y \right)$,则$y'' = p\dfrac{{\rd p}}{{\rd y}}$,
  代入原式得
  \[y\left( {1 + y} \right) \times p\dfrac{{\rd p}}{{\rd y}} =  - \left( {1 + \ln y} \right){p^2} \Rightarrow \dfrac{{\rd p}}{p} =  - \dfrac{{1 + \ln y}}{{y(1 - \ln y)}}\rd y\]
  两侧积分得\[\ln p = \ln y + 2\ln (1 - \ln y) + \ln {C_0}\]
  整理得\[ p = \dfrac{{\rd y}}{{\rd x}} = {C_0}y{(1 - \ln y)^2} \Rightarrow \dfrac{{\rd y}}{{y{{(1 - \ln y)}^2}}} = {C_0}x\]
  再次积分得\[\dfrac{1}{{1 - \ln y}} = {C_0}x + C\]
  整理得\[ - \ln y = \dfrac{{ - {C_0}x - C + 1}}{{{C_0}x + C}} \Rightarrow y = {e^{\frac{{x + {C_2}}}{{x + {C_1}}}}} \quad ({C_1} = \dfrac{C}{{{C_0}}},{C_2} = \dfrac{{C - 1}}{{{C_0}}}).\]

  (7)解:记$y'' = p(x) \Rightarrow y''' = p'$,
  代入原式得\[2xpp' = {p^2} - {a^2} \Rightarrow 2p\dfrac{{\rd p}}{{{p^2} - {a^2}}} = \dfrac{{\rd x}}{x}\]
  两侧积分并整理得\[{p^2} = Cx + {a^2}\]
  \[\therefore y = \int {\left( {\int {p\rd x} } \right)} \rd x = {C_2}x + {C_3} \pm \dfrac{{4{{\left( {{C_1}x + {C_1}^2} \right)}^{\frac{5}{2}}}}}{{15C_1^2}}.\]

  2.~(1)解:记$ y' = p\left( x \right) \Rightarrow y'' = p'$,
  代入原式得\[p' + {p^2} = 1 \Rightarrow \dfrac{{\rd p}}{{\rd x}} =  - {p^2} + 1 \Rightarrow \dfrac{{\rd p}}{{1 - {p^2}}} = \rd x\]
  两侧积分并整理得\[\dfrac{{1 + p}}{{1 - p}} = {e^{2(x + C)}} \Rightarrow p = \dfrac{{{e^{2(x + C)}} - 1}}{{{e^{2(x + C)}} + 1}}\]
  亦即\[\dfrac{{\rd y}}{{\rd x}} = \dfrac{{{e^{2(x + C)}} - 1}}{{{e^{2(x + C)}} + 1}}\]
  两侧再次积分并整理得\[y = \ln \left[ {{e^{2(C + x)}} + 1} \right] - x + {C_0}\]
  将条件$y\left( 0 \right) = 0,y'\left( 0 \right) = 0$代入,得
  \[C = 0,{C_0} =  - \ln 2\]
  综上,特解为\[y = \ln \dfrac{{{e^x} + {e^{ - x}}}}{2} = \ln \mathrm{ch} x.\]

  (2)解:记$y' = p\left( y \right)$,则$y'' = p\dfrac{{\rd p}}{{\rd y}}$,
  代入原式,得\[yp\dfrac{{\rd p}}{{\rd y}} = 2({p^2} - p) \Rightarrow \dfrac{{\rd p}}{{p - 1}} = \dfrac{{2\rd y}}{y}\]
  两侧积分并整理得\[p = \dfrac{{dy}}{{dx}} = C{y^2} + 1 \Rightarrow \dfrac{{\rd y}}{{C{y^2} + 1}} = \rd x\]
  两侧积分并整理得\[y = \dfrac{{\tan \left( {\sqrt C x + {C_2}} \right)}}{{\sqrt C }}\]
  将条件$y(0) = 1,y'\left( 0 \right) = 2$代入,得
  \[\sqrt C  = 1,{C_2} = \dfrac{\pi }{4}\]
  综上,特解为\[y = \tan \left( {x + \dfrac{\pi }{4}} \right).\]

  (5)解:记$ y' = p\left( x \right) \Rightarrow y'' = p'$,
  代入原式得\[p' + {p^2} + 1 = 0 \Rightarrow \dfrac{{\rd p}}{{{p^2} + 1}} =  - \rd x\]
  两侧积分并整理得\[p = \tan (C - x)\]
  亦即\[\dfrac{{\rd y}}{{\rd x}} = \tan (C - x)\]
  \[ \Rightarrow \rd y =  - \dfrac{{\sin (C - x)}}{{\cos (C - x)}}\rd \left( {C - x} \right)\]
  两侧积分并整理得
  \[y = \ln \left| {\cos (C - x)} \right| + {C_1}\]

  将条件$y(0{\text{) = 0,y'(0) = 1}}$代入,得\[C = \dfrac{\pi }{4}{\text{,}}{{\text{C}}_1} = 1{\text{ + }}\dfrac{1}{2}\ln 2\]
  综上,特解为
  \[y = \ln \left| {\cos \left( {\dfrac{\pi }{4} - x} \right)} \right| + 1 + \dfrac{1}{2}\ln 2.\]

  3.解:$s = \displaystyle\int {v\rd t} $,
  对$v$,有$\dfrac{{\rd v}}{{\rd t}} = g - \dfrac{{{{(cv)}^2}}}{m}$,
  而$v = s'$,
  $\therefore$根据题意得关于s的微分方程为 \[s'' = g - \dfrac{{{c^2}{{(s')}^2}}}{m}\]
  记$p = s' \Rightarrow s'' = p\dfrac{{\rd p}}{{\rd s}}$,
  代入原式得
  \[p\dfrac{{\rd p}}{{\rd s}} = g - \dfrac{{{c^2}{p^2}}}{m} \Rightarrow \dfrac{p}{{mg - {c^2}{p^2}}}\rd p = \rd s\]

  两侧积分并整理得\[s + \ln {C_1} =  - \dfrac{1}{{2{c^2}}}\ln \left( {mg - {c^2}{p^2}} \right)\]
  \[ \Rightarrow p =  \pm \dfrac{1}{{{c^2}}}\sqrt {mg - C{e^{ - 2{c^2}s}}} ({C_2} = {C_1}{e^{ - 2{c^2}}})\]

  将条件$p(0) = 0,s(0) = 0$代入得\[{C_2} = mg\]

  所以\[\dfrac{{\rd p}}{{\rd s}} =  \pm \dfrac{{\sqrt {mg} }}{{{c^2}}}\sqrt {1 - {e^{ - 2{c^2}s}}} \]

  分离变量得\[ \pm \sqrt {\dfrac{{{e^{2{c^2}s}}}}{{{e^{2{c^2}s}} - 1}}} ds = \dfrac{{\sqrt {mg} }}{{{c^2}}}\rd t\]

  两侧积分并整理得\[s = \dfrac{m}{{{c^2}}}\ln ch\left( {\dfrac{{ct}}{{\sqrt {mg} }}} \right).\]

\section{高阶线性微分方程}
\begin{flushright}
  \color{zhanqing!80}
  \ding{43} 教材见373页
\end{flushright}
  1.由朗斯基行列式

  (1)解:$\begin{vmatrix}
    {{e^{{x^2}}}}&{x{e^{{x^2}}}} \\
    {2x{e^{{x^2}}}}&{{e^{{x^2}}} + 2{x^2}{e^{{x^2}}}}
  \end{vmatrix} = {e^{2{x^2}}} \ne 0$$\therefore $线性无关. \qquad
  (3)解:$\begin{vmatrix}
    {{e^{{x^2}}}}&{x{e^{{x^2}}}} \\
    {2x{e^{{x^2}}}}&{{e^{{x^2}}} + 2{x^2}{e^{{x^2}}}}
  \end{vmatrix} = {e^{2{x^2}}} \ne 0$$\therefore $线性无关.

  (5)解:$\begin{vmatrix}
    {2{x^2} + 1}&{{x^2} - 1}&{x + 2} \\
    {4x}&{2x}&1 \\
    4&2&0
  \end{vmatrix} =  - 6 \ne 0$$\therefore $线性无关.

  2.证明:假设${y_1}(x)$与${y_2}(x)$线性相关,则朗斯基行列式

  $\begin{vmatrix}
    {{y_1}\left( x \right)}&{{y_2}\left( x \right)} \\
    {{y_1}^\prime \left( x \right)}&{{y_2}^\prime \left( x \right)}
  \end{vmatrix} = {y_1}\left( x \right){y_2}^\prime \left( x \right) - {y_2}\left( x \right){y_1}^\prime \left( x \right) = 0$ \quad
  $\because \dfrac{{{y_2}\left( x \right)}}{{{y_1}\left( x \right)}} \ne $常数

  $\therefore {y_2}\left( x \right) \ne 0$故$\dfrac{{{y_1}\left( x \right)}}{{{y_2}\left( x \right)}} = \dfrac{{{y_1}^\prime \left( x \right)}}{{{y_2}^\prime \left( x \right)}}$与题设矛盾\quad
  $\therefore {y_1}(x)$与${y_2}(x)$线性无关.

  3.解:对于零次时,有
  \[{a_n}\left( x \right){y_1}\left( x \right) + {a_n}\left( x \right){y_2}\left( x \right) = {a_n}\left( x \right)[{y_1}\left( x \right) + {y_2}\left( x \right)]\]

  对于一次时,有
  \[{a_{n - 1}}\left( x \right)\frac{{\rd {f_1}\left( x \right)}}{{\rd x}} + {a_{n - 1}}\left( x \right)\frac{{\rd {f_2}\left( x \right)}}{{\rd x}}\rd x = {a_{n - 1}}\left( x \right)\frac{{\rd \left[ {{f_1}\left( x \right) + {f_2}\left( x \right)} \right]}}{{\rd x}}\]

  对于二次,有
  \[ {a_{n - 2}}\left( x \right)\dfrac{{{\rd ^2}[{y_1}\left( x \right)]}}{{\rd {x^2}}} + {a_{n - 2}}\left( x \right)\dfrac{{{\rd ^2}[{y_2}\left( x \right)]}}{{\rd {x^2}}} = {a_{n - 2}}\left( x \right)\dfrac{{{\rd ^2}\left[ {{y_1}\left( x \right)} \right] + {\rd ^2}\left[ {{y_2}\left( x \right)} \right]}}{{\rd {x^2}}} \]
  \[ {a_{n - 2}}\left( x \right)\dfrac{{{\rd ^2}\left[ {{y_1}\left( x \right) + {y_2}\left( x \right)} \right]}}{{\rd {x^2}}} = {a_{n - 2}}\left( x \right)\dfrac{{\rd \left[ {\frac{{\rd {y_1}\left( x \right)}}{{\rd x}} + \frac{{\rd {y_2}\left( x \right)}}{{\rd x}}} \right]}}{{\rd x}} = {a_{n - 2}}\left( x \right)\dfrac{{{\rd ^2}[{y_2}\left( x \right)] + {\rd ^2}\left[ {{y_1}\left( x \right)} \right]}}{{\rd {x^2}}} \]

  \[\text{即~}{a_{n - 2}}\left( x \right)\dfrac{{{\rd ^2}\left[ {{y_1}\left( x \right) + {y_2}\left( x \right)} \right]}}{{\rd {x^2}}} = {a_{n - 2}}\left( x \right)\dfrac{{{\rd ^2}[{y_2}\left( x \right)] + {\rd ^2}\left[ {{y_1}\left( x \right)} \right]}}{{\rd {x^2}}}; \]

  同理可证\[{a_{n - k}}\left( x \right)\dfrac{{{\rd ^k}\left[ {{y_1}\left( x \right)} \right]}}{{\rd {x^k}}} + {a_{n - k}}\left( x \right)\dfrac{{{\rd ^k}\left[ {{y_2}\left( x \right)} \right]}}{{\rd {x^k}}} = {a_{n - k}}\left( x \right)\dfrac{{{\rd ^k}\left[ {{y_1}\left( x \right) + {y_2}\left( x \right)} \right]}}{{\rd {x^k}}}k .\in \left[3, + \infty \right).\]

\section{常系数线性微分方程}
\begin{flushright}
  \color{zhanqing!80}
  \ding{43} 教材见380页
\end{flushright}
  1.~(1)解:特征方程为\[{r^4} - 1 = 0 \Rightarrow {r_1},{r_2} =  \pm 1,{r_3},{r_4} =  \pm i\]

  $\therefore $通解为
  \[y = {C_1}{e^x} + {C_2}{e^{ - x}} + {C_3}\cos x + {C_4}\sin x.\]
  \begin{flalign*}
    \begin{split}
      \text {(3)特征方程为}\displaystyle
      &{r^4} - 5{r^3} + 6{r^2} + 4r - 8 = 0
      \Rightarrow \left( {r - 2} \right)\left( {{r^3} - 3{r^2} + 4} \right) = 0\\
      &\Rightarrow \left( {r + 1} \right)\left( {r - 2} \right)\left( {{r^2} - 4r + 4} \right) = 0
      \Rightarrow {r_1} =  - 1,{r_2} = {r_3} = {r_4} = 2.
   \end{split}&
  \end{flalign*}
  $\therefore $通解为
  \[y = {C_1}{e^{ - x}}{\text{ + }}\left( {{C_2} + {C_3}x + {C_4}{x^2}} \right){e^{2x}}.\]

  (5)解:特征方程为
  \[{r^4} - 13{r^2} + 36 = 0 \Rightarrow \left( {{r^2} - 4} \right)\left( {{r^2} - 9} \right) = 0\]
  得特征方程的根为\[{r_1},{r_2} =  \pm 3,{r_3},{r_4} =  \pm 2\]

  故通解为\[y = {C_1}{e^{3x}} + {C_2}{e^{ - 3x}} + {C_3}{e^{2x}} + {C_4}{e^{ - 2x}}.\]

  2.~(2)解:对应的齐次方程为\[{r^3} + 3{r^2} + 3r + 1 = 0 \Rightarrow {\left( {r + 1} \right)^3} = 1\]
  故对应的齐次方程通解为\[\tilde y = \left( {{C_1} + {C_2}x + {C_3}{x^2}} \right){e^{ - x}}\]
  设一特解
  \[{y^*} = {x^3}\left( {{b_0}x + {b_1}} \right){e^{ - x}}\]
    \begin{flalign*}
      \begin{split}
      \text {于是}\displaystyle
      &{\left( {{y^*}} \right)^\prime } = \left[ { - {b_0}{x^4} + \left( {4{b_0} - {b_1}} \right){x^3} + 3{b_1}{x^2}} \right]{e^{ - x}}\\
      &{\left( {{y^*}} \right)^{\prime \prime }} = \left[ {{b_0}{x^4} - \left( {8{b_0} - {b_1}} \right){x^3} + \left( {12{b_0} - 6{b_1}} \right){x^2}} + 6{b_1}x \right]{e^{ - x}}\\
      &{\left( {{y^*}} \right)^{\prime \prime \prime }} = x\left[ { - {b_0}{x^4} + \left( {12{b_0} - {b_1}} \right){x^3} + \left( { - 36{b_0} + 9{b_1}} \right){x^2} + \left( {24{b_0} - 18{b_1}} \right)x} \right]{e^{ - x}}
      \end{split}&
    \end{flalign*}
  代入原方程得$24{b_0}x + 6{b_1} = x - 5$

  解得$e {b_0} = \dfrac{1}{{24}},{b_1} =  - \dfrac{5}{6}$

  故有一特解\[{y^*} = \dfrac{1}{{24}}{x^3}\left( {x - 20} \right){e^{ - x}}\]
  故而原方程通解为\[ y = \left( {{C_1} + {C_2}x + {C_3}{x^2}} \right){e^{ - x}} + \dfrac{1}{{24}}{x^3}\left( {x - 20} \right){e^{ - x}}.\]

  (4)解:特征方程为${r^2} - 3r + 2 = 0$

  故通解为\[\tilde y = {C_1}{e^{2x}} + {C_2}{e^x}\]
  设特解\[{y^*} = {a_0} + {a_1}x + {a_2}{x^2} + {a_3}{x^3} + {b_1}\sin x + {b_2}\cos x\]

  $\therefore {\left( {{y^*}} \right)^\prime } = {a_1} + 2{a_2}x + 3{a_3}{x^2} + {b_1}\cos x - {b_2}\sin x \quad {\left( {{y^*}} \right)^{\prime \prime }} = 2{a_2} + 6{a_3}x - {b_1}\sin x - {b_2}\cos x$

  代回原方程得\[2{a_3}{x^3} + \left( {2{a_2} - 9{a_3}} \right){x^2} + \left( {2{a_1} - 6{a_2} + 6{a_3}} \right)x + \left( {2{a_0} - 3{a_1} + 2{a_2}} \right) + \left( {{b_1} + 3{b_2}} \right)\sin x + \left( {{b_2} - 3{b_1}} \right)\cos x = {x^3} + \sin x\]
  解得${a_3} = \dfrac{1}{2},{a_2} = \dfrac{9}{4},{a_1} = \dfrac{{21}}{4},{a_0} = \dfrac{{45}}{8},{b_1} = \dfrac{1}{{10}},{b_2} = \dfrac{3}{{10}}$

  $\therefore$原方程通解为\[ y = {C_1}{e^{2x}} + {C_2}{e^x} + \dfrac{3}{{10}}\cos x + \dfrac{1}{{10}}\sin x + \dfrac{1}{2}{x^3} + \dfrac{9}{4}{x^2} + \dfrac{{21}}{4}x + \dfrac{{45}}{8}.\]

  (6)解:特征方程为${r^2} - 2r + 3 = 0 \Rightarrow {r_{1,2}} = 1 \pm \sqrt 2 i$

  $\therefore$对应的齐次方程通解为\[\tilde y = {e^x}\left( {{C_1}\sin \sqrt 2 x + {C_2}\cos \sqrt 2 x} \right)\]
  设一个特解为${y^*} = \left( {a\cos x + b\sin x} \right){e^{ - x}}$

  则${\left( {{y^*}} \right)^\prime } = \left[ {\left( {b - a} \right)\cos x - \left( {b + a} \right)\sin x} \right]{e^{ - x}}$

  ${\left( {{y^*}} \right)^{\prime \prime }} = \left( { - 2b\cos x + 2a\sin x} \right){e^{ - x}}$

  代入原方程并解得$a = \dfrac{5}{{41}},b =  - \dfrac{4}{{41}}$
  \[\therefore {y^*} = \dfrac{1}{{41}}\left( {5\cos x - 4\sin x} \right){e^{ - x}}\]
  $\therefore$通解为
  \[y = {e^x}\left( {{C_1}\sin \sqrt 2 x + {C_2}\cos \sqrt 2 x} \right) + \dfrac{1}{{41}}\left( {5\cos x - 4\sin x} \right){e^{ - x}}.\]

  (8)解:对应的特征方程为${r^2} + 2ar + {a^2} = 0 \Rightarrow {r_1} = {r_2} =  - a$

  故对应通解为\[\tilde y = \left( {{C_1} + {C_2}x} \right){e^{ - ax}}\]
  $\because  r =  - a $为二重根,
  $\therefore $设特解为${y^*} = {e^x}\left( {{b_0} + {b_1}x + {b_2}{x^2}} \right)$
  $\therefore {\left( {{y^*}} \right)^\prime } = {e^x}\left[ {\left( {{b_0} + {b_1}} \right) + \left( {{b_1} + 2{b_2}} \right)x + {b_2}{x^2}} \right]$
  ${\left( {{y^*}} \right)^{\prime \prime }} = {e^x}\left[ {\left( {{b_0} + {b_1} + 2{b_2}} \right) + \left( {{b_1} + 4{b_2}} \right)x + {b_2}{x^2}} \right]$
  解得$\begin{cases}
    {{a_0} =  - 1} \\
    {{b_0} = 0} \\
    {{b_1} = 0} \\
    {{b_2} = \dfrac{1}{2}}
  \end{cases}$
  或
  $\begin{cases}
    {{a_0} \ne  - 1} \\
    {{b_0} = {{\left( {\dfrac{1}{{a + 1}}} \right)}^2}} \\
    {{b_1} = 0} \\
    {{b_2} = 0}
  \end{cases}$

  $\therefore a =  - 1$时,方程的通解为\[y = {e^x}\left( {{C_1} + {C_2}x + \frac{1}{2}{x^2}} \right)\]

  $ a \ne 1 $时,方程的通解为\[y = {e^{ - ax}}\left( {{C_1} + {C_2}x} \right) + \frac{{{e^x}}}{{{{\left( {a + 1} \right)}^2}}}.\]

  3.~(1)解:特征方程为${r^2} - 4r + 13 = 0 \Rightarrow {r_{1,2}} = 2 \pm 3i$

  $\therefore $通解为\[y = {e^{2x}}\left( {{C_1}\sin 3x + {C_2}\cos 3x} \right)\]
  而\[y' = {e^{2x}}\left[ {\left( {{C_1} - 3{C_2}} \right)\sin 3x + \left( {3{C_1} + {C_2}} \right)\cos 3x} \right]\]
  将$\begin{cases}
    {y\left( 0 \right) = 0} \\
    {y'\left( 0 \right) = 3}
  \end{cases}$代入,得
  $\begin{cases}
    {{C_2} = 0} \\
    {{C_1} = 1}
  \end{cases}$

  $\therefore $特解为\[y = {e^{2x}} \cdot \sin 3x.\]

  (3)解:对应特征方程为${r^2} + r - 2 = 0 \Rightarrow {r_1} = 1,{r_2} =  - 2$

  $\therefore $对应通解为$\tilde y = {C_1}{e^x} + {C_2}{e^{ - 2x}}$

  设特解为${y^*} = a\sin x + b\cos x$

  $\therefore {\left( {{y^*}} \right)^\prime } = a\cos x - b\sin x,{\left( {{y^*}} \right)^{\prime \prime }} =  - a\sin x - b\cos x$

  代入得$\left( { - 3a + b} \right)\sin x + \left( {a - 3b} \right)\cos x = \cos x - 3\sin x$

  $\therefore \begin{cases}
    { - 3a + b =  - 3} \\
    {a - 3b = 1}
  \end{cases} \Rightarrow \begin{cases}
    {a = 1} \\
    {b = 0}
  \end{cases}$

  故特解为${y^*} = \sin x$

  $\therefore $通解为\[y = {C_1}{e^x} + {C_2}{e^{ - 2x}} + \sin x.\]
  又$y\left( 0 \right) = 0,y'\left( 0 \right) = 2,\therefore {C_1} = 1,{C_2} = 0$

  故特解为\[y = {e^x} + \sin x.\]

  (5)解:对应特征方程为${r^2} - 4r + 3 = 0$,
  特征根为${r_1} = 1,{r_2} = 3$

  $\therefore $对应通解为\[\tilde y = {C_1}{e^{3x}} + {C_2}{e^x}\]
  设一个特解为${y^*} = a{e^{5x}} \Rightarrow {\left( {{y^*}} \right)^\prime } = 5a{e^{5x}},{\left( {{y^*}} \right)^{\prime \prime }} = 25a{e^{5x}}$

  代入原式得$25a - 20a + 3a = 1 \Rightarrow a = \dfrac{1}{8}$

  $\therefore $特解为\[{y^*} = \dfrac{1}{8}{e^{5x}}\]
  $\therefore $通解为\[ y = {C_1}{e^{3x}} + {C_2}{e^x} + \dfrac{1}{8}{e^{5x}}\]
  $y' = 3{C_1}{e^{3x}} + {C_2}{e^x} + \dfrac{5}{8}{e^{5x}}$,又
  $\begin{cases}
    {y\left( 0 \right) = 3} \\
    {y'\left( 0 \right) = 9}
  \end{cases} \Rightarrow \begin{cases}
    {{C_1} = \dfrac{{11}}{4}} \\
    {{C_2} = \dfrac{1}{8}}
  \end{cases}$

  $\therefore $特解为\[\tilde y = \dfrac{{11}}{4}{e^{3x}} + \dfrac{1}{8}{e^x} + \dfrac{1}{8}{e^{5x}}.\]

  (7)解:对应特征方程为${r^2} - 1 = 0$

  特征根为${r_1} = 1,{r_2} =  - 1$

  $\therefore $对应通解为\[\tilde y = {C_1}{e^x} + {C_2}{e^{ - x}}\]
  设特解为${y^*} = \left( {{a_0} + {a_1}x + {a_2}{x^2}} \right){e^x}$
  \begin{flalign*}
      \begin{split}
      \Rightarrow \displaystyle
       &{\left( {{y^*}} \right)^\prime } = \left[ {{a_0} + {a_1} + \left( {{a_1} + 2{a_2}} \right)x + {a_2}{x^2}} \right]{e^x}\\
       &{\left( {{y^*}} \right)^{\prime \prime }} = \left[ {{a_0} + {a_1} + 2{a_2} + \left( {{a_1} + 4{a_2}} \right)x + {a_2}{x^2}} \right]{e^x}
  \end{split}&
    \end{flalign*}
  代入原方程得$2{a_1} + 2{a_2} + 4{a_2}x = 4x$
  $ \Rightarrow \begin{cases}
    {{a_0} = 0} \\
    {{a_1} =  - 1} \\
    {{a_2} = 1}
  \end{cases} \therefore {y^*} = \left( { - x + {x^2}} \right){e^x}$

  $\therefore $通解为\[y = {C_1}{e^x} + {C_2}{e^{ - x}} + \left( { - x + {x^2}} \right){e^x}\]
  由$y\left( 0 \right) = 0,y'\left( 0 \right) = 1$得${C_1} = 1,{C_2} =  - 1$

  $\therefore $特解为\[y = {e^x} - {e^{ - x}} + \left( { - x + {x^2}} \right){e^x}.\]

  (9)解:原方程对应的特征方程为${r^2} + 4r = 0 \Rightarrow {r_1} = 0,{r_2} =  - 4$,
  $\therefore $通解为\[\tilde y = {C_2}{e^{ - 4x}} + {C_1}\]
  记${y^*} = {a_0} + {a_1}x + {b_0}\cos 2x + {b_1}\sin 2x + {b_2}x\cos 2x + {b_3}x\sin 2x$
    \begin{flalign*}
      \begin{split}
      \Rightarrow \displaystyle
         &{\left( {{y^*}} \right)^\prime } = {a_0} + \left( {{b_3} - 2{b_0}} \right)\sin 2x + \left( {{b_2} + 2{b_1}} \right)\cos 2x - 2{b_2}x\sin 2x + 2{b_3}x\cos 2x \\
         &{\left( {{y^*}} \right)^{\prime \prime }} = \left( { - 4{b_2} - 4{b_1}} \right)\sin 2x + \left( {4{b_3} - 4{b_0}} \right)\cos 2x - 4{b_2}x\cos 2x - 4{b_3}x\sin 2x
      \end{split}&
    \end{flalign*}
  代入原方程得
  $\begin{cases}
    {{b_0} = 0} \\
    {{b_1} =  - \dfrac{1}{{16}}} \\
    {{b_2} = 0} \\
    {{b_3} = \dfrac{1}{8}}
  \end{cases} \Rightarrow
  \begin{cases}
    {{a_0} = 0} \\
    {{a_1} = \dfrac{1}{8}}
  \end{cases}$,
  $\therefore $原方程的一个特解为${y^*} = \dfrac{1}{8}x - \dfrac{1}{{16}}\sin 2x + \dfrac{1}{8}x\sin x$

  将$y\left( 0 \right) = 0,y'\left( 0 \right) = 0$代入得${C_1} = {C_2} = 0$

  $\therefore $原方程特解为
  \[y = \dfrac{1}{8}x - \dfrac{1}{{16}}\sin 2x + \dfrac{1}{8}x\sin x.\]

  4.解:设$\alpha x = t$,$\therefore \displaystyle\int_0^1 {y\left( {\alpha x} \right)} \rd \alpha  = \int_0^x {y\left( t \right)}\rd \dfrac{t}{x} = \dfrac{1}{x}\int_0^x {y\left( t \right)} \rd t $,
  故而$2x\displaystyle\int_0^1 {y\left( {\alpha x} \right)} \rd \alpha  = 2\int_0^x {y\left( t \right)} \rd t$

  方程两侧对$x$求导得$y'' + 3y' + 2y = {e^{ - x}}$,
  对应特征方程为${r^2} + 3r + 2 = 0 \Rightarrow {r_1} =  - 1,{r_2} =  - 2$

  $\therefore $对应通解为\[\tilde y = {C_1}{e^{ - x}} + {C_2}{e^{ - 2x}}\]
  设特解为${y^*} = \left( {{a_0} + {a_1}x} \right){e^{ - x}}$

  $\therefore {\left( {{y^*}} \right)^\prime } = \left( {{a_1} - {a_0} + {a_1}x} \right){e^{ - x}}$,
  $\therefore {\left( {{y^*}} \right)^{\prime \prime }} = \left( {{a_0} - 2{a_1} + {a_1}x} \right){e^{ - x}}$,
  代入原方程得${a_1} = 1,{a_0} = 0$

  $\therefore $特解为${y^*} = x{e^{ - x}}$,
  $\therefore $原方程的通解为\[y = x{e^{ - x}} + {C_1}{e^{ - x}} + {C_2}{e^{ - 2x}}\]
  代入${y_0} = 1$得${C_1} + {C_2} = 1$,
  $x = 0$代入原方程得${y'_0}= - 1$,
  代入通解,得$ - 2{C_1} - {C_1} = 1$

  $\therefore {C_2} = 0 , {C_1} = 0$

  $\therefore $特解为\[y = x{e^{ - x}} + {e^{ - 2x}}.\]

  6.解:$v = s' = \dfrac{{\rd s}}{{\rd t}}, a = v' = s''$,
  $v' = \dfrac{{F - W}}{P} = \dfrac{{F - a - bv}}{P}$,
  $\therefore s'' = \dfrac{{F - a - bs'}}{P} \Rightarrow Ps'' + bs' = F - a$

  易得原方程的特征方程为$P{\lambda ^2} + b\lambda  = 0 \Rightarrow {\lambda _1} = 0,{\lambda _2} =  - \dfrac{b}{P}$

  $\therefore $通解为\[\tilde s = {C_1} + {C_2}{e^{ - \frac{b}{P}t}}\]
  易知一个特解为$\dfrac{{F - a}}{b}t$,
  $\therefore $通解为$s = {C_1} + {C_2}{e^{ - \frac{b}{P}t}} + \dfrac{{F - a}}{b}t$

  将条件$s\left( 0 \right) = 0,s'\left( 0 \right) = 0$代入得
  $\begin{cases}
    {{C_1} = \dfrac{{\left( {F - a} \right)}}{{{b^2}}}P} \\
    {{C_2} = \dfrac{{\left( {a - F} \right)}}{{{b^2}}}P}
  \end{cases}$
  \[\therefore s = \dfrac{{\left( {F - a} \right)}}{{{b^2}}}P - \dfrac{{\left( {F - a} \right)}}{{{b^2}}}{e^{ - \frac{b}{P}t}} + \dfrac{{F - a}}{b}t.\]


\section*{总习题十}
\addcontentsline{toc}{section}{总习题十}
\begin{flushright}
  \color{zhanqing!80}
  \ding{43} 教材见393页
\end{flushright}
  1.~(1)解:设积分因子$\mu  = {e^x}$
  原式$\Rightarrow {e^{ - y}}\left( {{e^y} + {e^y}\dfrac{{dy}}{{dx}} - 4\sin x} \right) = 0
  \Rightarrow {e^y} + {e^y}\dfrac{{\rd y}}{{\rd x}} - 4\sin x = 0$

  $\Rightarrow {e^x}\left( {{e^y} + {e^y}\dfrac{{\rd y}}{{\rd x}} - 4\sin x} \right) = 0$,
  设$P\left( {x,y} \right) = {e^{x + y}} - 4\sin x{e^x}$,

  $Q\left( {x,y} \right) = {e^{x + y}}$
  $\because\dfrac{{\partial P}}{{\partial y}} = {e^{x + y}} = \dfrac{   {\partial Q}}{{\partial x}}$
  $\therefore $该式子可以构成全微分

  $\therefore \rd \left( {{e^{x + y}} - 2{e^x}\sin x + 2{e^x}\cos x} \right) = 0$,
  即所求方程为\[{{e^{x + y}} - 2{e^x}\sin x + 2{e^x}\cos x} = C.\]

  (3)解:令$y = ux \therefore \dfrac{{\rd y}}{{\rd x}} = u + \dfrac{{\rd u}}{{\rd x}}$
  $\therefore \text {原式} \Rightarrow x\dfrac{{\rd u}}{{\rd x}} = {u^2}
    \Rightarrow \dfrac{{\rd u}}{{{u^2}}} = \dfrac{{\rd x}}{x}
    \Rightarrow - \dfrac{1}{u} = \ln x + {C_0}$

  $\therefore y =  - \dfrac{x}{{\left( {C - \ln \left| x \right|} \right)}}\left( {y \ne 0} \right)$,
  $y = 0$时符合题意.

  (5)解:特征方程为${r^2} + 2r + 5 = 0$,
  特征根为${r_{1,2}} =  - 1 \pm 2i$,
  $\therefore $通解为\[\tilde y = {e^{ - x}}\left( {{C_1}\cos 2x + {C_2}\sin 2x} \right)\]

  设一个特解为${y^*} = a\cos 2x + b\sin 2x$

  ${\left( {{y^*}} \right)^\prime } =  - 2a\sin 2x + 2b\cos 2x$
  $\Rightarrow {\left( {{y^*}} \right)^{\prime \prime }} =  - 4a\cos 2x - 4b\sin 2x$

  代入原方程得$\left( {b - 4a} \right)\sin 2x + \left( {4b + a} \right)\cos 2x = \sin 2x$

  解得$a =  - \dfrac{4}{{17}},b = \dfrac{1}{{17}}$,
  $\therefore $特解为${y^*} =  - \dfrac{4}{{17}}\cos 2x + \dfrac{1}{{17}}\sin 2x$

  故原方程的通解为\[y = {e^{ - x}}\left( {{C_1}\cos 2x + {C_2}\sin 2x} \right) - \dfrac{4}{{17}}\cos 2x + \dfrac{1}{{17}}\sin 2x.\]

  (7)解:方程两侧同乘积分因子$\mu \left( x \right) = {e^x}$得

  $2xy{e^x}\rd x + {x^2}y{e^x}\rd x + \dfrac{{{y^3}}}{3}{e^x}\rd x + {x^2}{e^x}\rd y + {y^2}{e^x}\rd y = 0$,
  $ \Rightarrow d\left( {{x^2}y{e^x}} \right) + d\left( {\dfrac{{{y^3}}}{3}{e^x}} \right) = 0$

  故所得的方程为\[{x^2}y{e^x} + \dfrac{{{y^3}}}{3}{e^x} = C.\]

  (9)解:$\dfrac {{\rd y}}{{\rd x}} = \dfrac{{1 + x{y^3}}}{{1 + {x^3}y}} = 0 \Rightarrow \left( {\rd x + \rd y} \right) + xy\left( {{y^2}\rd x + {x^2}\rd y} \right) = 0$

  令$u = x + y,v = x - y$,
  $\therefore \begin{cases}
    {\rd y = \dfrac{{y\rd u - \rd v}}{{y - x}}} \\
    {\rd x = \dfrac{{\rd v - y\rd u}}{{y - x}}}
  \end{cases}$,
  代入整理得$\left( {{v^2} - 1} \right)\rd u + uv\rd v = 0$
  \[ \Rightarrow \dfrac{v}{{{v^2} - 1}}\rd v =  - \dfrac{{\rd u}}{u}\left( {u = x + y \ne 0,v = xy \ne 1} \right)\]

  两侧积分并整理得\[\dfrac{1}{2}\ln \left( {{v^2} - 1} \right) = \ln u + C' \Rightarrow \sqrt {{x^2}{y^2} - 1}  = C\left( {x + y} \right).\]

  当$xy = 1$时不符合题意;当$x+y = 0$时,符合题意.
  故通解为
  \[\sqrt {{x^2}{y^2} - 1}  = C\left( {x + y} \right),x + y = 0.\]

  2.~(1)解:特征方程为${r^2} + 2r + 1 = 0 \Rightarrow {r_1} = {r_2} = 1$,
  $\therefore $对应通解为
  \[\tilde y = \left( {{C_1}{\text{ + }}{C_2}x} \right){e^{ - x}}\]

  设一个特解为${y^*} = a\sin x + b\cos x$

  $\therefore {\left( {{y^*}} \right)^\prime } = a\cos x - b\sin x$,
  $\therefore {\left( {{y^*}} \right)^{\prime \prime }} =  - a\sin x - b\cos x$,
  代入整理得$- 2b\sin x + 2a\cos x = \cos x$

  解得$b = 0,a = \dfrac{1}{2}$,
  $\therefore $特解为${y^*} = \dfrac{1}{2}\sin x$,
  $\therefore $通解为
  \[y = \left( {{C_1} + {C_2}x} \right){e^{ - x}} + \dfrac{1}{2}\sin x\]
  将 $y\left( 0 \right) = 0,y'\left( 0 \right) = \dfrac{3}{2}$代入得${C_1} = 2,{C_2 = - 1}$,
  $\therefore $原方程特解为
  \[y = \dfrac{1}{2}\sin x + x{e^{ - x}}.\]

  (3)解:对应的特征方程为${r^3} + 6{r^2} + 11r + 6 = 0 \Rightarrow \left( {r + 1} \right)\left( {r + 2} \right)\left( {r + 3} \right) = 0$

  $\therefore $特征解为$ {r_1} = - 1,{r_2} = - 2,{r_3} = - 3$,
  $\therefore $通解为
  \[\tilde y = {C_1}{e^{ - x}} + {C_2}{e^{ - 2x}} + {C_3}{e^{ - 3x}}\]

  设一个特解为${y^*} = {a_0} + {a_1}{e^{ - 4x}}$
  $ \Rightarrow {\left( {{y^*}} \right)^\prime } =  - 4{a_1}{e^{ - 4x}},{\left( {{y^*}} \right)^{\prime \prime }} = 16{a_1}{e^{ - 4x}}, \Rightarrow {\left( {{y^*}} \right)^{\prime \prime \prime }} =  - 64{a_1}{e^{ - 4x}}$

  代入原方程并解得${a_0} = {a_1} = \dfrac {1}{6}$,
  $\therefore $特解为${y^*} = \dfrac{1}{6}{e^{ - 4x}} + \dfrac{1}{6}$,
  $\therefore $原方程通解为
  \[y = {C_1}{e^{ - x}} + {C_2}{e^{ - 2x}} + {C_3}{e^{ - 3x}} + \dfrac{1}{6}{e^{ - 4x}} + \dfrac{1}{6}\]

  将$y\left( 0 \right) = 5,y'\left( 0 \right) = 0$代入得${C_1} = \dfrac {43}{3}, {C_2} = - 14, {C_3} = \dfrac {13}{3}$,
  $\therefore $原方程特解为
  \[y = \dfrac{{43}}{3}{e^{ - x}} - 14{e^{ - 2x}} + \dfrac{{13}}{3}{e^{ - 3x}} + \dfrac{1}{6}{e^{ - 4x}} + \dfrac{1}{6}.\]

  (5)解:设$y'' = p \Rightarrow y''' = p'$,
  $\therefore \left( {x - 1} \right)p' - p = 0 \Rightarrow \dfrac {\rd p}{\rd x} = \dfrac {\rd x}{x - 1}$,
  整理并解得$p = C\left( {x - 1} \right)$

  $\because \begin{cases}
    {x = 2} \\
    {y'' = 1}
  \end{cases} \Rightarrow C = 1$,
  $\therefore \dfrac{{\rd \left( {y'} \right)}}{{\rd x}} = x - 1$
  两侧积分得$y' = \dfrac{1}{2}{x^2} - x + {C_0}$

  $\because \begin{cases}
    {x = 2} \\
    {y' = 1}
  \end{cases} \Rightarrow {C_0} = 1$,
  两侧积分得
  $y' = \dfrac{1}{2}{x^2} - x + 1 \Rightarrow y = \dfrac{1}{6}{x^3} - \dfrac{1}{2}{x^2} + x + {C_1}$

  $\because \begin{cases}
    {x = 2} \\
    {y = 2}
  \end{cases} \Rightarrow {C_1} = \dfrac {2}{3}$
  $\Rightarrow y = \dfrac{1}{6}{x^3} - \dfrac{1}{2}{x^2} + x + \dfrac{2}{3}$.