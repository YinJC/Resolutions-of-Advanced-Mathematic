% !TEX root = ../HTNotes-Demo.tex
\begin{flushright}
  \color{zhanqing!80}
  \ding{43} 习题见\autopageref{cha:8}
\end{flushright}
\section{二重积分的概念及性质}
\begin{flushright}
  \color{zhanqing!80}
  \ding{43} 教材见184页
\end{flushright}

  2.解:(1) 该积分表示的是以$R$为半径的上半球的体积, 则$$\displaystyle\iint \limits_D \sqrt{R^2-x^2-y^2} \rd \sigma =  \dfrac{1}{2} \times  \dfrac{4}{3} \times \pi R^3 =  \dfrac{2}{3}\pi R^3.$$

  (2) 该积分表示的是以$R$为底面半径,$H$为高的圆锥的体积, 则$$\displaystyle\iint \limits_D H- \dfrac{H}{R}\sqrt{x^2+y^2} \rd \sigma =  \dfrac{1}{3} \times \pi R^2 \times H =  \dfrac{1}{3}\pi R^2 H.$$

  3.解: (1) $\sin^2 x \le x^2$,  则$\sin^2\left( x+y \right) \le \left( x+y \right)^2$, 则$$\displaystyle\iint \limits_D \sin^2\left( x+y \right) \rd \sigma < \displaystyle\iint \limits_D \left( x+y \right)^2 \rd \sigma.$$

  (2) $1 \le x+y \le 2$,  则$0 \le \ln\left( x+y \right) \le 1$, 则$\ln \left( x+y \right) \ge \ln^2 \left( x+y \right)$, 则$$\displaystyle\iint \limits_D \ln \left( x+y \right) \rd \sigma > \displaystyle\iint \limits_D \ln^2 \left( x+y \right) \rd \sigma.$$

  (3) $3 \le x+y \le 6$,  则$\ln\left( x+y \right) > 1$, 则$\ln \left( x+y \right) < \ln^2 \left( x+y \right)$, 则$$\displaystyle\iint \limits_D \ln \left( x+y \right) \rd \sigma < \displaystyle\iint \limits_D \ln^2 \left( x+y \right) \rd \sigma.$$

  (4) $e^x \ge 1+x$,  则$e^{x^2+y^2} \ge 1+{x^2+y^2}$, 则$$\displaystyle\iint \limits_D e^{x^2+y^2} \rd \sigma > \displaystyle\iint \limits_D 1+{x^2+y^2} \rd \sigma.$$

  4.解: (1) $0 \le \sin^2 x \sin^2 y \le 1$, 则$\displaystyle\displaystyle\iint \limits_D 0 \rd \sigma \le I \le \displaystyle\displaystyle\iint \limits_D 1 \rd \sigma$, 则$$0 \le I \le \pi^2.$$

  (2) $1 \le x+y+1 \le 4$, 则$\displaystyle\displaystyle\iint \limits_D 1 \rd \sigma \le I \le \displaystyle\displaystyle\iint \limits_D 4 \rd \sigma$, 则$$2 \le I \le 8.$$

  (3) $0 \le \sqrt[4]{xy(x+y)} \le 2$, 则$\displaystyle\displaystyle\iint \limits_D 0 \rd \sigma \le I \le \displaystyle\displaystyle\iint \limits_D 2 \rd \sigma$, 则$$0 \le I \le 8.$$

  (4) $1 \le x^2+y^2+1 \le 17$, 则$\displaystyle\displaystyle\iint \limits_D 1 \rd \sigma \le I \le \displaystyle\displaystyle\iint \limits_D 17 \rd \sigma$, 则$$12 \pi \le I \le 204 \pi.$$

\section{二重积分的计算}
\begin{flushright}
  \color{zhanqing!80}
  \ding{43} 教材见201页
\end{flushright}

  \begin{flalign*}
    \begin{split}
      \text{1.解: }(1) \displaystyle\iint \limits_D \left( x^2+y^2 \right) \rd \sigma
      & = 4 \int_0^1 \rd x \int_0^1 \left( x^2+y^2 \right) \rd y \\
      & = 4 \int_0^1 \left( \left. x^2+\dfrac{1}{3} y^3 \right|_0^1 \right) \rd x
      = 4 \int_0^1 \left( x^2+\dfrac{1}{3} \right) \rd x \\
      & = 4 \times \left( \dfrac{1}{3}+\dfrac{1}{3} \right) = \dfrac{8}{3}.
    \end{split}&
  \end{flalign*}

  \begin{flalign*}
    \begin{split}
      (2) \displaystyle\iint \limits_D x\cos(x+y) \rd \sigma
      & = 4 \int_0^\pi \rd x \int_0^x x\cos(x+y) \rd y
      = 4 \int_0^\pi x \left. \sin(x+y) \right|_0^x \rd x  \\
      & = \left( \left. -\dfrac{x}{2} \cos2x + \dfrac{1}{4}\sin2x +x\cos x -\sin x \right) \right|_0^\pi \\
      & = - \dfrac{\pi}{2} - \pi = - \dfrac{3\pi}{2}.
    \end{split}&
  \end{flalign*}

  \begin{flalign*}
    \begin{split}
      (3) \displaystyle\iint \limits_D xe^{x^2+y} \rd \sigma
      & = \int_0^4 \rd x \int_1^3 xe^{x^2+y} \rd y
      = \int_0^4 \left( x \left. e^{x^2+y} \right|_1^3 \right) \rd x \\
      & = \dfrac{1}{2} \int_0^4 e^{x^2+3}-e^{x^2+1} \rd (x^2)
      = \dfrac{1}{2} \left( e^{19} - e^{17} - e^3 +e \right).
    \end{split}&
    \end{flalign*}

  \begin{flalign*}
    \begin{split}
      (4) \displaystyle\iint \limits_D x^2y\sin\left( xy^2 \right) \rd \sigma
      & = - \dfrac{1}{2} \int_0^\frac{\pi}{2} \rd x \int_0^2 x^2y\sin\left( xy^2 \right) \rd y
      = 4 \int_0^\frac{\pi}{2} x \left. \cos(xy^2) \right|_0^2 \rd x \\
      & = -\dfrac{1}{2} \left( -\dfrac{x}{4} \sin4x - \dfrac{1}{16}\cos4x - \left. \dfrac{1}{2} x^2  \right) \right|_0^\frac{\pi}{2} = \dfrac{\pi^2}{16}.
    \end{split}&
  \end{flalign*}

  \begin{flalign*}
     \begin{split}
      \text{2.解: }(1) \displaystyle\iint \limits_D xy \rd \sigma
      & = \int_0^1 \rd x \int_{\sqrt{x}}^{2-x} xy \rd y
      = \dfrac{1}{2} \int_0^1 \left( x \left. y^2 \right|_{\sqrt{x}}^{2-x} \right) \rd x  \\
      & = \dfrac{1}{2} \int_0^1 \left( x^3-5x^2+4x \right) \rd x
      = \dfrac{1}{2} \left( \left. \dfrac{1}{4} x^4 - \dfrac{5}{3} x^3 + 2x^2 \right) \right|_0^1 \\
      & = \dfrac{1}{2} \times\left( \dfrac{1  }{4}-\dfrac{5}{3}+2 \right)  = \dfrac{7}{24}.
    \end{split}&
  \end{flalign*}

  \begin{flalign*}
    \begin{split}
      (2) \displaystyle\iint \limits_D x\sqrt{y} \rd \sigma
      & = \int_0^1 \rd y \int_{y^2}^{\sqrt{y}} x\sqrt{y} \rd y
      = \dfrac{1}{2} \int_0^1 \left( \left. x^2 \sqrt{y} \right|_{y^2}^{\sqrt{y}} \right) \rd y
      = \dfrac{1}{2} \int_0^1 \left( y^{\frac{3}{2}}-y^{\frac{9}{2}} \right) \rd y \\
      & = \dfrac{1}{2} \times \left( \dfrac{2}{5}-\dfrac{2}{11} \right) = \dfrac{6}{55}.
    \end{split}&
  \end{flalign*}

  \begin{flalign*}
    \begin{split}
      (3) \displaystyle\iint \limits_D \left( x^2+y^2-y \right) \rd \sigma
      & = \int_0^2 \rd y \int_y^{2y} \left( x^2+y^2-y \right) \rd x
      = \int_0^2 \left( \dfrac{10}{3}y^3-y^2 \right) \rd y \\
      & = \dfrac{5}{3} \times 8 - \dfrac{8}{3}
      = \dfrac{32}{3}.
    \end{split}&
  \end{flalign*}

  \begin{flalign*}
    \begin{split}
      (4) \displaystyle\iint \limits_D xy^2 \rd \sigma
      & = 2 \int_0^2 \rd y \int_0^{\sqrt{4-y^2}} xy^2 \rd x
      = 2 \int_0^2 \left. \dfrac{1}{2}x^2y^2 \right|_0^{\sqrt{4-y^2}} \rd y \\
      & = \left( \left. \dfrac{4}{3} y^3 - \dfrac{1}{5} y^5 \right) \right|_0^2
      = \dfrac{4}{3} \times 8 - \dfrac{32}{5}
      = \dfrac{64}{15}.
    \end{split}&
  \end{flalign*}

  % 3.画图即可得解, 略.

  \begin{flalign*}
     \begin{split}
      \text{4.解: }(1) \int_0^{\sqrt{\pi}} \rd x \int_x^{\sqrt{\pi}} \sin \left( y^2 \right) \rd y
      & = \int_0^{\sqrt{\pi}} \rd y \int_0^y \sin \left( y^2 \right) \rd x
      = \int_0^{\sqrt{\pi}} y \sin \left( y^2 \right) \rd y \\
      & = -\dfrac{1}{2} \left. \cos \left( y^2 \right) \right|_0^{\sqrt{\pi}}
      = - \dfrac{1}{2} \times (-2)
      = 1.
    \end{split}&
  \end{flalign*}

  \begin{flalign*}
    \begin{split}
      (2) \int_2^4 \rd y \int_{\frac{y}{2}}^2 e^{x^2-2x} \rd x
      & = \int_1^2 \rd x \int_2^{2x} e^{x^2-2x} \rd y
      = 2 \int_1^2 \left( x-1 \right) e^{x^2-2x} \rd x
      = \int_1^2 e^{x^2-2x} \rd \left( x^2-2x \right) \\
      & = \left. e^{x^2-2x} \right|_1^2
      = 1-e^{-1}.
    \end{split}&
  \end{flalign*}

  \begin{flalign*}
    \begin{split}
      (3) \int_1^2 \rd x \int_{\sqrt{x}}^x \sin \left( \dfrac{\pi x}{2y} \right) \rd y + \int_2^4 \rd x \int_{\sqrt{x}}^2 \sin \left( \dfrac{\pi x}{2y} \right) \rd y
      & = \int_1^2 \rd y \int_y^{y^2} \sin \left( \dfrac{\pi x}{2y} \right) \rd x \\
      & = -\dfrac{2}{\pi} \int_1^2 \left( y \cos \left( \dfrac{\pi}{2} y \right) \right) \rd y \\
      & = -\dfrac{2}{\pi} \times \left( -\dfrac{2}{\pi}-\dfrac{4}{\pi^2} \right)
      = \dfrac{4}{\pi^3}\left( \pi+2 \right).
    \end{split}&
    \end{flalign*}

  \begin{flalign*}
    \begin{split}
      (4) \int_{\frac{1}{4}}^{\frac{1}{2}} \rd y \int_{\frac{1}{2}}^{\sqrt{y}} e^{\frac{y}{x}} \rd x + \int_{\frac{1}{2}}^{\sqrt{y}} \rd y \int_y^{\sqrt{y}} e^{\frac{y}{x}} \rd x
      & = \int_{\frac{1}{2}}^1 \rd x \int_{x^2}^x e^{\frac{y}{x}} \rd y \\
      & = \int_{\frac{1}{2}}^1 \left. e^{\frac{y}{x}} \right|_{x^2}^x \rd x
      = \int_{\frac{1}{2}}^1 \left( ex-xe^x \right) \rd x
      = \dfrac{3}{8}e-\dfrac{\sqrt{e}}{2}.
    \end{split}&
  \end{flalign*}

  5.证明:
    \begin{align*}
      \text{ }\displaystyle\iint \limits_D f_1 (x) \cdot f_2 (y) \rd x \rd y
      & = \int_a^b \rd x \int_c^d f_1 (x) \cdot f_2 (y) \rd y \\
      & = \int_a^b f_1 (x) \rd x \int_c^d f_2 (y) \rd y \\
      & = \left[ \int_a^b f_1 (x) \rd x \right] \cdot \left[ \int_c^d f_2 (y) \rd y \right].
    \end{align*}

  \begin{flalign*}
     \begin{split}
      \text{6.解: }(1)~V & = \displaystyle\iint \limits_D \left( 6-2x-3y-0 \right) \rd x \rd y
      = \int_0^1 \rd x \int_0^1 \left( 6-2x-3y \right) \rd y
      = \int_0^1 \left( 6-2x-\dfrac{3}{2} \right) \rd x
      =  \dfrac{7}{2}.
    \end{split}&
  \end{flalign*}

  \begin{flalign*}
    \begin{split}
      (2)~V & = \displaystyle\iint \limits_D \left( 6-x^2-y^2-0 \right) \rd x \rd y
      = \int_0^1 \rd y \int_0^{1-x} \left( 6-x^2-y^2 \right) \rd y \\
      & = \int_0^1 \left( 6-6x-x^2+x^3+\dfrac{1}{3}\left( x-1 \right)^3 \right) \rd x
      = 3-\dfrac{1}{3}+\dfrac{1}{4}-\dfrac{1}{12}
      = \dfrac{17}{6}.
    \end{split}&
  \end{flalign*}

  (3) 由$x^2+2y^2=6-2x^2-y^2$可知$D$为: $x^2+y^2 \le 2$.
  \begin{flalign*}
    \begin{split}
      V & = \displaystyle\iint \limits_D \left( 6-2x^2-y^2-x^2-2y^2 \right) \rd x \rd y
      = 3 \displaystyle\iint \limits_D \left( 2-x^2-y^2 \right) \rd x \rd y \\
      & = 3 \int_0^{2\pi} \rd \theta \int_0^{\sqrt{2}} \left( 2-\rho^2 \right) \rho \rd \rho
      = 6 \pi \times (2-1)
      = 6 \pi.
    \end{split}&
  \end{flalign*}

  % 7.画图即可得解, 略.

  \begin{flalign*}
     \begin{split}
      \text{8.解: }(1) \int_0^2 \rd x \int_0^{\sqrt{2x-x^2}} \left( x^2+y^2 \right) \rd y
      & = \int_0^{\frac{\pi}{2}} \rd \theta \int_0^{2\cos\theta} \rho^3 \rd \rho \\
      & = 4 \int_0^{\frac{\pi}{2}} \cos^4 \theta \rd \theta
      = 4 \times \dfrac{3}{4\times2}\times{\frac{\pi}{2}}
      = \dfrac{3\pi}{4}.
    \end{split}&
  \end{flalign*}

  \begin{flalign*}
    \begin{split}
      (2) \int_0^a \rd x \int_0^x \sqrt{x^2+y^2} \rd y
      & = \int_0^{\frac{\pi}{4}} \rd \theta \int_0^{\frac{a}{\cos\theta}} \rho^2 \rd \rho \\
      & = \dfrac{a^3}{3} \int_1^{\frac{\pi}{4}} \sec^3 \theta \rd \theta
      = \dfrac{a^3}{3} \left( \left. \sec\theta\tan\theta-\dfrac{1}{2}\sec^2\theta \right) \right|_0^{\frac{\pi}{4}}
      = \dfrac{a^3}{3}\left( \sqrt{2} - \dfrac{1}{2} \right).
    \end{split}&
  \end{flalign*}
  {\small 注: 该题结果与答案不一样, 经检验, 未发现错误, 如果发现请联系本节作者.}

  \begin{flalign*}
    \begin{split}
      (3) \int_0^1 \rd x \int_{x^2}^x \left( x^2+y^2 \right)^\frac{1}{2} \rd y
      & = \int_0^{\frac{\pi}{4}} \rd \theta \int_0^{\frac{\sin\theta}{\cos^2\theta}} \rd \rho \\
      & = \int_0^{\frac{\pi}{4}} \frac{\sin\theta}{\cos^2\theta} \rd \theta
      = - \int_0^{\frac{\pi}{4}} \frac{1}{\cos^2\theta} \rd \left( \cos\theta \right)
      = \sqrt{2}-1.
    \end{split}&
    \end{flalign*}

  \begin{flalign*}
    \begin{split}
      (4) \int_0^a \rd y \int_0^{\sqrt{a^2-y^2}} \left( x^2+y^2 \right) \rd x
      & = \int_0^{\frac{\pi}{2}} \rd \theta \int_0^a \rho^3 \rd \rho
      = \dfrac{\pi}{2}\times\dfrac{a^3}{4}=\dfrac{\pi}{8} a^4.
    \end{split}&
  \end{flalign*}

  9.解: (1)由题可知
  \begin{flalign*}
    \begin{split}
      \displaystyle\iint \limits_D \ln \left( 1+x^2+y^2 \right) \rd \sigma
      & = \int_0^{\frac{\pi}{2}} \rd \theta \int_0^1 \rho \ln \left( 1+\rho^2 \right) \rd \rho
      = \dfrac{\pi}{4} \int_0^1 \ln \left( 1+x \right) \rd x
      = \dfrac{\pi}{4}\left( 2\ln2-1 \right).
    \end{split}&
  \end{flalign*}

  (2)由题可知
  \begin{flalign*}
    \begin{split}
      \displaystyle\iint \limits_D \arctan \left( \dfrac{y}{x} \right) \rd \sigma
      = \int_0^{\frac{\pi}{4}} \rd \theta \int_1^2 \rho \arctan\left( \dfrac{\rho\sin\theta}{\rho\cos\theta} \right) \rd \rho
      = \int_0^{\frac{\pi}{4}} \theta \rd \theta \int_1^2 \rho \rd \rho
      = \dfrac{\pi^2}{32}\times\dfrac{3}{2}
      = \dfrac{3}{64}\pi^2.
    \end{split}&
  \end{flalign*}

  (3)由题可知
  \begin{flalign*}
    \begin{split}
      \displaystyle\iint \limits_D \dfrac{x+y}{x^2+y^2} \rd \sigma
      = \int_0^{\frac{\pi}{2}} \rd \theta \int_{\frac{1}{\cos\theta+\sin\theta}}^1 \left( \cos\theta+\sin\theta \right) \rd \rho
      = \int_0^{\frac{\pi}{2}} \left( \cos\theta+\sin\theta-1 \right) \rd \theta
      = 2-\dfrac{\pi}{2}.
    \end{split}&
  \end{flalign*}

  (4)由题可知
  \begin{flalign*}
    \begin{split}
      \displaystyle\iint \limits_D \left( \dfrac{1-x^2-y^2}{1+x^2+y^2} \right) ^\frac{1}{2} \rd \sigma
      & = \int_0^{\frac{\pi}{2}} \rd \theta \int_0^1 \sqrt{\dfrac{1-\rho^2}{1-\rho^2}} \rho \rd \rho
      = \dfrac{\pi}{4} \int_0^1 \sqrt{\dfrac{1-t}{1+t}} \rd t \\
      & \xlongequal[\text{再三角换元}]{\text{分子分母同时乘以}1-t} \dfrac{\pi}{4} \times \int_0^{\frac{\pi}{2}} \dfrac{1-\sin\alpha}{\cos\alpha} \rd (\sin\alpha)
      = \dfrac{\pi}{4} \left( \dfrac{\pi}{2}-1 \right).
    \end{split}&
  \end{flalign*}

  10.解: (1)由题可知$D$: $\begin{cases} x^2+y^2 \le ax \\ z=0 \end{cases}$
  \begin{flalign*}
    \begin{split}
      V = \displaystyle\iint \limits_D  \left( x^2+y^2-0 \right) \rd \sigma
      = 2 \int_0^{\frac{\pi}{2}} \rd \theta \int_0^{a\cos\theta} \rho^3 \rd \rho
      = \dfrac{a^4}{2} \int _0^{\frac{\pi}{2}} \cos^4 \theta \rd \theta
      = \dfrac{a^4}{2} \times \dfrac{3}{4\times2} \times \dfrac{\pi}{2}
      = \dfrac{3\pi}{64} a^4.
    \end{split}&
  \end{flalign*}

  (2)由题可知$D$: $\begin{cases} x^2+y^2 \le \dfrac{1}{2} \\ z=0 \end{cases}$
  \begin{flalign*}
    \begin{split}
      V & = \displaystyle\iint \limits_D  \sqrt{1-x^2-y^2}-\sqrt{x^2+y^2} \rd \sigma
      = 4 \int_0^{\frac{\pi}{2}} \rd \theta \int_0^{\frac{\sqrt{2}}{2}} \left( \rho\sqrt{1-\rho^2}-\rho^2 \right) \rd \rho \\
      & = 2\pi \int _0^{\frac{1}{2}} \sqrt{1-t} \rd t - 2\pi \int_0^{\frac{\sqrt{2}}{2}} \rho^2 \rd \rho
      = \dfrac{2\pi}{3} \times \left( 1-\dfrac{\sqrt{2}}{4} \right) - \dfrac{2\pi}{3} \times \dfrac{\sqrt{2}}{4}
      = \dfrac{\pi}{3} \left( 2-\sqrt{2} \right).
    \end{split}&
  \end{flalign*}

\section{三重积分}
\begin{flushright}
  \color{zhanqing!80}
  \ding{43} 教材见220页
\end{flushright}
  \begin{flalign*}
    \begin{split}
      \text{3.解: }(1) \displaystyle\iiint \limits_\Omega xz \rd x \rd y \rd z
      = \int_{-1}^1 \rd x \int_{x^2}^1 \rd y \int_0^y xz \rd z
      = \dfrac{1}{2} \int_0^1 \rd x \int_{x^2}^1 xy^2 \rd y
      = \dfrac{1}{6} \int_{-1}^1 \left( x-x^7 \right) \rd x
      = 0.
    \end{split}&
  \end{flalign*}

  法二: 由于积分区间关于平面$yOz$对称, 同时被积函数是关于x的奇函数, 则根据“偶倍奇零”可直接得出答案$0$.

  \begin{flalign*}
    \begin{split}
      (2) & \displaystyle\iiint \limits_\Omega \dfrac{1}{\left( 1+x+y \right)^3} \rd x \rd y \rd z
      = \int_0^1 \rd x \int_0^{1-x} \rd y \int_0^{1-x-y} \dfrac{1}{\left( 1+x+y \right)^3} \rd z \\
      & = \int_0^1 \rd x \int_0^{1-x} \left( -\dfrac{1}{8}+\dfrac{1}{2(1+x+y)^2} \right) \rd y
      = - \int_0^1 \left( \dfrac{1-x}{8}+\dfrac{1}{4}-\dfrac{1}{2(1+x)} \right) \rd x
      = \dfrac{1}{2}\left( \ln2-\dfrac{5}{8} \right).
    \end{split}&
  \end{flalign*}

  \begin{flalign*}
    \begin{split}
      (3) \displaystyle\iiint \limits_\Omega xyz \rd x \rd y \rd z
      & = \int_0^1 \rd x \int_0^{\sqrt{1-x^2}} \rd y \int_0^{\sqrt{1-x^2-y^2}} xyz \rd z
      = \dfrac{1}{2} \int_0^1 \rd x \int_0^{\sqrt{1-x^2}} xy \left( 1-x^2-y^2 \right) \rd y \\
      & = \dfrac{1}{8} \int_0^1 \left( x-2x^3+x^5 \right) \rd x
      = \dfrac{1}{48}.
    \end{split}&
  \end{flalign*}

  \begin{flalign*}
    \begin{split}
      (4) \displaystyle\iiint \limits_\Omega z \rd x \rd y \rd z
      = \int_0^h z \rd z \displaystyle\iint \limits_D \rd x \rd y
      = \dfrac{\pi R^2}{h^2} \int_0^h z^3 \rd z
      = \dfrac{\pi}{4}RH.
    \end{split}&
  \end{flalign*}

  (5) 令$u=\dfrac{x}{a}$, $v=\dfrac{y}{b}$, $w=\dfrac{z}{c}$, 则$\rd x \rd y \rd z = abc \rd u \rd v \rd w$, $\Omega$为$u^2+v^2+w^2 \le 1$
  \begin{flalign*}
    \begin{split}
      \displaystyle\iiint \limits_\Omega \left( \dfrac{x^2}{a^2}+\dfrac{y^2}{b^2}+\dfrac{z^2}{c^2} \right) \rd x \rd y \rd z
      & = \displaystyle\iiint \limits_\Omega \left( u^2+v^2+w^2 \right) abc \rd u \rd v \rd w \\
      & = abc \int_0^1 \rd u \int_0^{1-u^2} \rd v \int_0^{1-u^2-v^2} \left( u^2+v^2+w^2 \right) \rd w \\
      & \xlongequal[\text{采用球坐标系求解}]{\text{鉴于计算复杂}} abc \int_0^{2\pi} \rd \theta \int_0^\pi \sin\varphi \rd \varphi \int_0^1 r^4 \rd r \\
      & = abc\times2\pi\times2\times\dfrac{1}{5}
      = \dfrac{4\pi}{5}abc.
    \end{split}&
  \end{flalign*}

  \begin{flalign*}
    \begin{split}
      \text{4.解: }(1) \displaystyle\iiint \limits_\Omega \left( x^2+y^2 \right) \rd x \rd y \rd z
      & = \int_0^{2\pi} \rd \theta \int_0^3 \rho^3 \rd \rho \int_0^{9-\rho^2} \rd z \\
      & = 2 \pi \int_0^3 \left( 9\rho^3-\rho^5 \right) \rd \rho
      = 2 \pi \left( \dfrac{9}{4}\times81-\dfrac{243}{2} \right)
      = \dfrac{243}{2}\pi.
    \end{split}&
  \end{flalign*}

  \begin{flalign*}
    \begin{split}
      (2) \displaystyle\iiint \limits_\Omega z \rd x \rd y \rd z
      & = \int_0^{2\pi} \rd \theta \int_0^1 \rho \rd \rho \int_{\rho^2}^{\sqrt{2-\rho^2}} z \rd z
      = \pi \int_0^1 \left( 2\rho-\rho^3-\rho^5 \right) \rd \rho
      = \pi \left( 1-\dfrac{1}{4}-\dfrac{1}{6} \right)
      = \dfrac{7}{12}\pi.
    \end{split}&
  \end{flalign*}

  \begin{flalign*}
    \begin{split}
      (3) \displaystyle\iiint \limits_\Omega \sqrt{x^2+y^2} \rd x \rd y \rd z
      & = \int_0^{2\pi} \rd \theta \int_0^4 \rho^2 \rd \rho \int_0^{4-\rho\sin\theta} \rd z
      = \int_0^{2\pi} \rd \theta \int_0^4 \left( 4-\rho\sin\theta \right) \rd \rho \\
      & = \dfrac{8\pi}{3}\times64
      = \dfrac{512}{3}\pi.
    \end{split}&
  \end{flalign*}

  \begin{flalign*}
    \begin{split}
      \text{5.解: }(1) \displaystyle\iiint \limits_\Omega xe^{\frac{x^2+y^2+z^2}{a^2}} \rd x \rd y \rd z
      & = a^4 \int_0^{\frac{\pi}{2}} \rd \theta \int_0^{\frac{\pi}{2}} \rd \varphi \int_0^a r^3 e^{r^2} \sin^2\varphi \cos\theta \rd r \\
      & = \int_0^{\frac{\pi}{2}} \cos\theta \rd \theta \int_0^{\frac{\pi}{2}} \sin^2\varphi \rd \varphi \int_0^a r^3e^{r^2} \rd r \\
      & = a^4 \times 1 \times \dfrac{\pi}{4} \times \dfrac{1}{2}\left( a^2e^{a^2}-a^{a^2}+1 \right)
      = \dfrac{\pi a^4}{8}\left( a^2e^{a^2}-a^{a^2}+1 \right).
    \end{split}&
  \end{flalign*}
  {\small 注: 该题结果与答案不一样, 经检验, 未发现错误, 如果发现请联系本节作者.}

  \begin{flalign*}
    \begin{split}
      (2) \displaystyle\iiint \limits_\Omega \sqrt{x^2+y^2+z^2} \rd x \rd y \rd z
      & = \int_0^{2\pi} \rd \theta \int_0^{\frac{\pi}{2}} \sin\varphi \rd \varphi \int_0^{\cos\varphi} r^3 \rd r \\
      & = -\dfrac{\pi}{2} \int_0^{\frac{\pi}{2}} \cos^4 \varphi \rd (\cos \varphi)
      = \dfrac{\pi}{2} \times \dfrac{1}{5}
      = \dfrac{\pi}{10}.
    \end{split}&
  \end{flalign*}

  \begin{flalign*}
    \begin{split}
      (3) \displaystyle\iiint \limits_\Omega z \rd x \rd y \rd z
      & = \int_0^{2\pi} \rd \theta \int_0^{\frac{\pi}{2}} \sin\varphi \cos\varphi \rd \varphi \int_0^{2a\cos\varphi} r^3 \rd r \\
      & = -8\pi a^4 \int_0^{\frac{\pi}{4}} \cos^5\varphi \rd (\cos\varphi)
      = \dfrac{4}{3}\pi a^4 \times \left( 1-\dfrac{1}{8} \right)
      = \dfrac{7}{6}\pi a^4.
    \end{split}&
  \end{flalign*}

  \begin{flalign*}
    \begin{split}
      \text{6.解: } M & = 2 \times \dfrac{1}{2} \displaystyle\iiint \limits_\Omega \left( x^2+y^2+z^2 \right) \rd x \rd y \rd z
      = \int_0^{2\pi} \rd \theta \int_0^{\frac{\pi}{4}} \sin\varphi \rd \varphi \int_0^{\sqrt{2}} r^4 \rd r \\
      & = \pi \left( 1-\dfrac{\sqrt{2}}{2} \right) \times \dfrac{1}{5} \times 4\sqrt{2}
      = \dfrac{4}{5}\pi\left( \sqrt{2}-1 \right).
    \end{split}&
  \end{flalign*}

  \begin{flalign*}
    \begin{split}
      \text{7.解: }(1) \displaystyle\iiint \limits_\Omega \sin z \rd x \rd y \rd z
      = \int_0^\pi \sin z \rd z \displaystyle\iint \limits_D \rd x \rd y
      = \pi \int_0^\pi z^2 \sin z \rd z
      = \pi^3-4\pi.
    \end{split}&
  \end{flalign*}

  \begin{flalign*}
    \begin{split}
      (2) \displaystyle\iiint \limits_\Omega e^{\sqrt{x^2+y^2+z^2}} \rd x \rd y \rd z
      & = \int_0^{2\pi} \rd \theta \int_0^{\frac{\pi}{4}} \sin\varphi \rd \varphi \int_0^1 r^2 e^r \rd r \\
      & = 2 \pi \left( e-2e+2e-2 \right) \times \dfrac{1}{5} \times 4\sqrt{2}
      = \pi \left( 2-\sqrt{2} \right)(e-2).
    \end{split}&
  \end{flalign*}

  \begin{flalign*}
    \begin{split}
      (3) \displaystyle\iiint \limits_\Omega \left( x^2+y^2 \right) \rd x \rd y \rd z
      = \int_0^{2\pi} \rd \theta \int_0^2 \rho^3 \rd \rho \int_{\frac{5}{2}\rho}^{5} \rd z
      = 10 \pi \int_0^2 \left( \rho^3-\dfrac{1}{2} \rho^4 \right) \rd \rho
      = 40 \pi - 32 \pi
      = 8 \pi.
    \end{split}&
  \end{flalign*}

  \begin{flalign*}
    \begin{split}
      (4) \displaystyle\iiint \limits_\Omega \left( x^2+y^2 \right) \rd x \rd y \rd z
      & = \int_0^{2\pi} \rd \theta \int_0^{\frac{\pi}{2}} \sin^3 \varphi \rd \varphi \int_a^A r^4 \rd r
      = 2 \pi \times \dfrac{2}{3} \times \dfrac{1}{5} \left( A^5-a^5 \right)
      = \dfrac{4\pi}{15} \left( A^5-a^5 \right).
    \end{split}&
  \end{flalign*}

  \begin{flalign*}
    \begin{split}
      \text{8.解: }(1)~V & = \displaystyle\iiint \limits_\Omega \rd x \rd y \rd z
      = \int_0^{2\pi} \rd \theta \int_0^2 \rho \rd \rho \int_\rho^{6-\rho^2} \rd z
      = 2 \pi \int_0^2 \left( 6 \rho - \rho^3 - \rho^2 \right) \rd \rho \\
      & = 2 \pi \times \left( 12-4-\dfrac{8}{3} \right)
      = \dfrac{32}{3}\pi.
    \end{split}&
  \end{flalign*}

  \begin{flalign*}
    \begin{split}
      (2)~V = \displaystyle\iiint \limits_\Omega \rd x \rd y \rd z
      = \int_0^{2\pi} \rd \theta \int_0^{\frac{\pi}{4}} \sin\varphi \rd \varphi \int_0^{2a\cos\varphi} r^2 \rd r
      = -\dfrac{4}{3} \pi a^3 \times \left( 1-\dfrac{1}{4} \right)
      = \pi a^3.
    \end{split}&
  \end{flalign*}

  \begin{flalign*}
    \begin{split}
      (3)~V & = \displaystyle\iiint \limits_\Omega \rd x \rd y \rd z
      = \int_0^{2\pi} \rd \theta \int_0^2 \rho \rd \rho \int_{\frac{\rho^2}{4}}^{\sqrt{5-\rho^2}} \rd z \\
      & = \pi \int_0^2 \left( \sqrt{5-\rho^2}-\frac{\rho^2}{4} \right) \rd \left( \rho^2 \right) \\
      & = \pi \times \left( \dfrac{2}{3} \left( 5\sqrt{5}-1 \right) - 2 \right)
      = \dfrac{2\pi}{3}\left( 5\sqrt{5}-4 \right).
    \end{split}&
  \end{flalign*}

\section{重积分的应用}
\begin{flushright}
  \color{zhanqing!80}
  \ding{43} 教材见234页
\end{flushright}

  \begin{flalign*}
    \begin{split}
      \text{1.解: }(1)~S & = \displaystyle\iint \limits_{\left( x-1 \right)^2 + y^2 \le 1} \sqrt{1 + \left( \dfrac{\partial z}{\partial x} \right)^2 + \left( \dfrac{\partial z}{\partial y}\right)^2}  \rd x \rd y \\
      & = \displaystyle\iint \limits_{D_{xy}} \sqrt{1 + \left( \dfrac{x}{\sqrt{x^2+y^2}} \right)^2 + \left( \dfrac{y}{x^2+y^2}\right)^2}  \rd x \rd y
      = \displaystyle\iint \limits_{D_{xy}} \sqrt{2} \rd x \rd y \\
      & = \sqrt{2} \displaystyle\iint \limits_{D_{xy}}  \rd \sigma
      = \sqrt{2} \pi \times 1^2
      = \sqrt{2} \pi.
    \end{split}&
  \end{flalign*}

  \begin{flalign*}
    \begin{split}
      (2)~S & = \displaystyle\iint \limits_{\left( x-\frac{a}{2} \right)^2 + y^2 \le \frac{a^2}{4}} \sqrt{1 + \left( \dfrac{\partial z}{\partial x} \right)^2 + \left( \dfrac{\partial z}{\partial y}\right)^2}  \rd x \rd y \\
      & = \displaystyle\iint \limits_{D_{xy}} \sqrt{1 + \left( \dfrac{-x}{\sqrt{a^2-x^2-y^2}} \right)^2 + \left( \dfrac{-y}{a^2-x^2-y^2}\right)^2}  \rd x \rd y \\
      & = \displaystyle\iint \limits_{D_{xy}} \sqrt{\dfrac{a^2}{a^2-x^2-y^2}} \rd x \rd y
      = a \int_{-\frac{\pi}{2}}^{\frac{\pi}{2}} \rd \theta \int_0^{a\cos\theta} \dfrac{1}{\sqrt{a^2-\rho^2}}\rho \rd \rho \\
      & = 2a^2\int_0^\frac{\pi}{2}\left( 1-\sin\theta \right) \rd \theta
      = \left( \pi-2 \right)a^2.
    \end{split}&
  \end{flalign*}

  \begin{flalign*}
    \begin{split}
      \text{2.解: }(1)~\bar{y} & = \dfrac{1}{A} \displaystyle\iint \limits_D y \rd x \rd y
      = \dfrac{2}{\pi ab} \displaystyle\iint \limits_D y \rd x \rd y
      = \dfrac{4}{\pi ab} \int_0^{\frac{\pi}{2}} \sin\theta \rd \theta \int_0^1 ab^2\rho^2 \rd \rho
      = \dfrac{2}{\pi ab} \cdot 2 \cdot ab^2 \dfrac{1}{3}
      = \dfrac{4b}{3\pi}.
    \end{split}&
  \end{flalign*}
  由“偶倍奇零”可得$\bar{x}=0$, 则质心为$\left( 0,\dfrac{4b}{3\pi} \right)$.

  \begin{flalign*}
    \begin{split}
      (2)~V & = \displaystyle\iiint \limits_\Omega \rd x \rd y \rd z
      = \int_0^a  \rd x \int_0^{a-x}  \rd y \int_0^{x^2+y^2} \rd z
      = \int_0^a  \rd x \int_0^{a-x} \left( x^2+y^2 \right) \rd y \\
      & = \int_0^a \left[x^2\left( a-x \right) + \dfrac{1}{3} \left( a-x \right)^3 \right] \rd x
      = \dfrac{1}{3} \times \dfrac{a^4}{2}
      = \dfrac{a^4}{6}
    \end{split}&
  \end{flalign*}
  \begin{flalign*}
    \begin{split}
      \bar{x} & = \dfrac{1}{V} \displaystyle\iiint \limits_\Omega x \rd x \rd y \rd z
      = \dfrac{1}{V} = \int_0^a x \rd x \int_0^{a-x}  \rd y \int_0^{x^2+y^2} \rd z
      = \dfrac{1}{V} \int_0^a  \rd x \int_0^{a-x} \left( x^2+y^2 \right) \rd y \\
      & = - \dfrac{1}{3V} \int_0^a \left( 4x^4-6ax^3+3a^2x^2-a^3x \right) \rd x
      = \dfrac{a^5}{15} \times \dfrac{6}{a^4}
      = \dfrac{2}{5}a
    \end{split}&
  \end{flalign*}
  \begin{flalign*}
    \begin{split}
      \bar{y} & = \dfrac{1}{V} \displaystyle\iiint \limits_\Omega y \rd x \rd y \rd z
      = \dfrac{1}{V} = \int_0^a \rd y \int_0^{a-y} y \rd x \int_0^{x^2+y^2} \rd z
      = \dfrac{1}{V} \int_0^a  \rd y \int_0^{a-y} \left( x^2+y^2 \right) \rd x \\
      & = - \dfrac{1}{3V} \int_0^a \left( 4y^4-6ay^3+3a^2y^2-a^3y \right) \rd y
      = \dfrac{a^5}{15} \times \dfrac{6}{a^4}
      = \dfrac{2}{5}a
    \end{split}&
  \end{flalign*}
  \begin{flalign*}
    \begin{split}
      \bar{z} & = \dfrac{1}{V} \displaystyle\iiint \limits_\Omega z \rd x \rd y \rd z
      = \dfrac{1}{V} = \int_0^a \rd x \int_0^{a-x} \rd y \int_0^{x^2+y^2} z \rd z
      = \dfrac{1}{2V} \int_0^a \rd x \int_0^{a-x} \left( x^2+y^2 \right)^2 \rd y \\
      & = \dfrac{1}{2V} \int_0^a \left[ x^4 \left( a-x \right)-\dfrac{2}{3} x^2 \left( a-x \right)^3+\dfrac{1}{5}\left( a-x \right)^5 \right] \rd x
      = \dfrac{1}{5a^4} \times \dfrac{-25+54-45+20+3}{6} a^6
      = \dfrac{7}{30}a^2
    \end{split}&
  \end{flalign*}
  则质心为$\left( \dfrac{2}{5}a,\dfrac{2}{5}a,\dfrac{7}{30}a^2 \right)$.

  \begin{flalign*}
    \begin{split}
      \text{4.解: }(1)~I_y & = \displaystyle\iint \limits_D x^2 \rd x \rd y
      = \int_0^{2\pi} \rd \theta \int_0^1 \rho^2 a^2 \cos^2\theta \rho ab \rd \rho \\
      & = a^3b \int_0^{2\pi} \cos^2 \theta \rd \theta \int_0^1 \rho^3 \rd \rho
      = a^b \times \pi \times \dfrac{1}{4}
      = \dfrac{1}{4}\pi a^3 b.
    \end{split}&
  \end{flalign*}

  \begin{flalign*}
    \begin{split}
      (2)~I & = \rho_0 \displaystyle\iiint \limits_\Omega \left( x^2+y^2 \right) \rd x \rd y \rd z
      = \rho_0 \int_0^{2\pi} \rd \theta \int_0^R \rho^3 \rd \rho \int_{\frac{H}{R}\rho}^{H} \rd z \\
      & = \dfrac{3M}{\pi R^2H} \times 2\pi H \times \int_0^R \left( 1-\dfrac{\rho}{R} \right) \rho^3 \rd \rho
      = \dfrac{6M}{R^2} \times \dfrac{1}{20} R^4
      = \dfrac{3}{10}MR^2.
    \end{split}&
  \end{flalign*}

  % 5. 略.

\section*{总习题八}
\addcontentsline{toc}{section}{总习题八}
\begin{flushright}
  \color{zhanqing!80}
  \ding{43} 教材见256页
\end{flushright}

  \begin{flalign*}
    \begin{split}
      \text{2.解: }(1)~\displaystyle\iint \limits_D \left| x^2+y^2-4 \right| \rd x \rd y
      & = \int_0^{2\pi} \rd \theta \int_0^3 \left| \rho^2-4 \right| \rho \rd \rho
      = \pi \left[ \int_0^4 (4-t) \rd t + \int_4^9 (t-4) \rd t \right] \\
      & = \pi \left( 16 - 8 + \dfrac{65}{2} - 20 \right)
      = \dfrac{41}{2} \pi.
    \end{split}&
  \end{flalign*}

  \begin{flalign*}
    \begin{split}
      (2)~\iint\limits_D {\left| {\cos \left( {x + y} \right)} \right|{\rd}\sigma }
      &  = \int_0^{\frac{\pi }{4}} {{\rd}y} \int_y^{\frac{\pi }{2} - y} {\cos\left( {x + y} \right){\rd}x}  - \int_{\frac{\pi }{4}}^{\frac{\pi }{2}} {{\rd}x} \int_{\frac{\pi }{2} - x}^x {\cos\left( {x + y} \right){\rd}y} \\
      & = \int_0^{\frac{\pi }{4}} {\left[ {1 - \sin \left( {2y} \right)} \right]{\rd}y}  - \int_{\frac{\pi }{4}}^{\frac{\pi }{2}} {\left[ {\sin \left( {2x} \right) - 1} \right]{\rd}x}
      = \int_0^{\frac{\pi }{2}} {\left[ {1 - \sin \left( {2x} \right)} \right]{\rd}x} \\
      & = \frac{\pi }{2} + \frac{1}{2}\left( { - 1 - 1} \right) = \frac{\pi }{2} - 1.
    \end{split}&
  \end{flalign*}

  \begin{flalign*}
    \begin{split}
      (3)~\displaystyle\iint \limits_D x^2 \rd x \rd y
      & = \int_{-\pi}^{\pi} \rd \theta \int_0^{a(1-\cos\theta)} \rho^3 \cos^2\theta \rd \rho \\
      & = \dfrac{a^2}{4} \int_{-\pi}^{\pi} \cos^2\theta (1-\cos\theta)^4 \rd \theta
      = a^2 \int_0^{\frac{\pi}{2}} \left( \cos^6\theta + 6\cos^4\theta + \cos^2\theta \right) \rd \theta \\
      & = a^2\left( \dfrac{5\times3}{6\times4\times2} + 6\times \dfrac{3}{4\times2} + \dfrac{1}{2} \right) \times \dfrac{\pi}{2}
      = \dfrac{49}{32} \pi.
    \end{split}&
  \end{flalign*}

  (4) 积分区间关于$y$轴对称, 被积函数是关于$x$的奇函数, 由``偶倍奇零''可知积分值为0.

  3.证明: 对左边交换程序可得

  \begin{flalign*}
    \begin{split}
      \int_0^1 \rd y \int_y^1 f(y) (x-y)^{a-1} \rd x
      & = \int_0^1 \dfrac{1}{a} (1-y)^a f(y) \rd y
      = \dfrac{1}{a} \int_1^0 t^a f(1-t) \rd (1-t) \\
      & = \dfrac{1}{a} \int_0^1 t^a f(1-t) \rd t
      = \dfrac{1}{a} \int_0^1 y^a f(1-y) \rd y
    \end{split}&
  \end{flalign*}

  证毕.

  4.证明: 将$D$分为关于原点对称的两个部分, 记作$D_1$, $D_2$, 则有
  \[ \iint_D f(x,y) \rd \sigma = \iint_{D_1} f(x,y) + \iint_{D_2} f(x,y) \rd \sigma \]

  又$\iint_{D_2} = \iint_{D_1} f(-x,-y) \rd \sigma$

  则有$\iint_D f(x,y) \rd \sigma = \iint_{D_1} f(x,y) - \iint_{D_1} f(x,y) \rd \sigma = 0$

  $\displaystyle\iint \limits_D \left( x^2y+xy^2 \right) \rd x \rd y
  = \displaystyle\iint \limits_D 1 \rd x \rd y + 0
  = 2\times2
  = 4$.

  \begin{flalign*}
    \begin{split}
      \text{5.解: }(1)~& \displaystyle\iint \limits_D \left| z^2 \right| \rd x \rd y \rd z \\
      & = \int_0^{2\pi} \rd \theta \int_0^{\frac{\pi}{3}} \sin\varphi \rd \varphi \int_0^R r^4 \cos^2\varphi \rd r + \int_0^{2\pi} \rd \theta \int_{\frac{\pi}{3}}^{\frac{\pi}{2}} \sin\varphi \rd \varphi \int_{2R\cos\varphi}^R r^4 \cos^2\varphi \rd r \\
      & = \dfrac{2}{5} \pi R^5 \left[ \int_0^{\frac{\pi}{3}} \left( \sin\varphi \cos^2\varphi \right) \rd \varphi + \int_{\frac{\pi}{3}}^{\frac{\pi}{2}} \left( \sin\varphi \cos^2\varphi  \right) \rd \varphi - 32\int_{\frac{\pi}{3}}^{\frac{\pi}{2}} \left( \sin\varphi \cos^7\varphi \right) \rd \varphi \right] \\
      & = \dfrac{2}{5} \pi R^5 \left( \int_0^1 t^2 \rd t - 32\int_0^{\frac{1}{2}} t^7 \rd t \right)
      = \dfrac{2}{5} \pi R^5 \times \left( \dfrac{1}{3} - \dfrac{32}{8}\times\dfrac{1}{256} \right)
      = \dfrac{59}{480} \pi R^5.
    \end{split}&
  \end{flalign*}

  (2)积分区间关于面$xOy$对称, 被积函数是关于$z$的奇函数, 由``偶倍奇零''可知积分值为0.

  (3)~$\displaystyle\iint \limits_\Omega \left( y^2+z^2 \right) \rd x \rd y \rd z = \int_0^5 \rd x \int_0^{2\pi} \rd \theta \int_0^{\sqrt{2x}} \rho^3 \rd \rho = 2\pi \int_0^5 x^2 \rd x = \dfrac{250}{3} \pi$.

  6. 证明: 对左边交换次序可得

  \begin{flalign*}
    \begin{split}
      \int_0^1 \rd x \int_0^x \rd y \int_0^y f(z) \rd z
      & = \int_0^1 f(z) \rd z \int_z^1 \rd y \int_z^y \rd x
      = \int_0^1 f(z) \rd z \int_z^1 (y-z) \rd y \\
      & = \dfrac{1}{2} \int_0^1 (1-z)^2 f(z) \rd z
    \end{split}&
  \end{flalign*}

  \begin{flalign*}
    \begin{split}
      \text{8.解: }~S & = \dfrac{1}{a} \displaystyle\iint \limits_{x^2+y^2 \le a} \sqrt{4x^2+4y^2+a^2} \rd x \rd y + \displaystyle\iint \limits_{x^2+y^2 \le a} \sqrt{2} \rd x \rd y \\
      & = \dfrac{1}{2a} \int_0^{2\pi} \rd \theta \int_0^a \sqrt{4\rho^2+a^2} \rd \rho^2 + \sqrt{2} \pi a^2
      = \pi a^2 \left( \dfrac{5\sqrt{5}-1}{6} + \sqrt{2} \right).
    \end{split}&
  \end{flalign*}

  \begin{flalign*}
    \begin{split}
      \text{10.解: }~J & =\mu \displaystyle\iint \limits_D \left( y+1 \right)^2 \rd x \rd y
      = \int_0^1 \left( y+1 \right)^2 \rd y \int_{-\sqrt{y}}^{\sqrt{y}} \rd x \\
      & = 2\mu \int_0^1 \left( y+1 \right)^2\sqrt{y} \rd y
      = \dfrac{368}{105} \mu.
    \end{split}&
  \end{flalign*}

  \begin{flalign*}
    \begin{split}
      \text{12.解: }~F(t) & = \int_0^{2\pi} \rd \theta \int_0^t \rho \rd \rho \int_0^h \left( z^2 + f(\rho^2) \right) \rd z
      = 2\pi \int_0^t \left[ \dfrac{1}{3} \rho h^3 + h \rho f(\rho^2) \right] \rd \rho \\
      \Rightarrow F'(t) & =\dfrac{2\pi}{3}h^3t + 2\pi ht f(t^2).
    \end{split}&
  \end{flalign*}