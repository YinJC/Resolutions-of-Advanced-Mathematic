% !TEX program = xelatex
% !TEX root = ../HTNotes-Demo.tex
\begin{flushright}
  \color{zhanqing!80}
  \ding{43} 习题见\autopageref{cha:5}
\end{flushright}


\section{常数项级数的概念与性质}
\begin{flushright}
  \color{zhanqing!80}
  \ding{43} 教材见 387页 % 这里需要添加页数
\end{flushright}
1、  单项选择题

(1) B
解析:由例1.1知几何级数  $\sum\limits_{n = 1}^\infty  {a{q^n}} $收敛的条件是$ - 1 < q < 1 $。

(2) B
解析:由级数收敛的必要条件知该级数发散。

(3) B
解析:取k=0时,该级数收敛,取k=1时,该级数发散。

(4) D
解析:取${\mu _n} = \dfrac{1}{n}$时,级数发散但$\lim\limits_{n \to +\infty} {\mu _n} = 0$。取${\mu _n}$=n时,$\lim\limits_{n \to +\infty} {\mu _n} = \infty $。

(5) C
解析:由级数收敛的必要条件知道 ,所以级数发散。

(6) C
解析:
\begin{equation}\label{eq:5-1-1}
  {2 = \sum\limits_{n = 1}^\infty  {( - 1)} ^{n - 1}}{\mu _n} = {\mu _1} - {\mu _2} + {\mu _3} + \cdots
\end{equation}
\begin{equation}\label{eq5-1-2}
  5 = \sum\limits_{n = 1}^\infty  {{\mu _{2n - 1}}}  = {\mu _1} + {\mu _3} + {\mu _4} + \cdots
\end{equation}
  \autoref{eq:5-1-1} $-$ \autoref{eq5-1-2} = 3 = $\sum\limits_{n = 1}^\infty  {{\mu _{2n}}}  = {\mu _2} + {\mu _4} + {\mu _6} + \cdots$

  \autoref{eq:5-1-1} $+$ \autoref{eq5-1-2} = 8.

  2、  填空题

(1)1,$\dfrac{4}{3},\dfrac{{31}}{{21}};$ (2)1,$\dfrac{3}{2},\dfrac{{31}}{{18}};$(3)$\dfrac{1}{5}, - \dfrac{1}{{25}},\dfrac{1}{{125}}.$

3、解析

${s_n} = {\mu _1} + {\mu _2} + \cdots + {\mu _{n - 1}} + {\mu _n} {s_{n - 1}} = {\mu _1} + {\mu _2} + \cdots + {\mu _{n - 2}} + {\mu _{n - 1}}$

所以${\mu _n} = {s_n} - {s_{n - 1}}$

因为${s_n} = \dfrac{{{3^n} - 1}}{{{3^n}}}$所以解得${\mu _n} = \dfrac{2}{{{3^n}}}$

$\sum\limits_{n = 1}^\infty  {{\mu _n}}  = \lim\limits_{n \to +\infty} {s_n} = \lim\limits_{n \to +\infty} \frac{{{3^n} - 1}}{{{3^n}}} = 1$

4、解析:

(1)$\because \dfrac{n}{{(n + 1)!}} = \dfrac{1}{{n!}} - \dfrac{1}{{(n + 1)!}},$

${s_n} = \sum\limits_{n = 1}^n {(\dfrac{1}{{n!}} - \dfrac{1}{{(n + 1)!}})}  = 1 - \dfrac{1}{{2!}} + \dfrac{1}{{2!}} - \dfrac{1}{{3!}} + \cdots + \dfrac{1}{{n!}} - \dfrac{1}{{(n + 1)!}} = 1 - \dfrac{1}{{(n + 1)!}}$

$\therefore \sum\limits_{n = 1}^\infty  {\dfrac{n}{{(n + 1)!}}}  = \lim\limits_{n \to +\infty} {s_n} = 1$

(2)解析:$ln\dfrac{{{n^2} - 1}}{{{n^2}}} = \ln \dfrac{{n - 1}}{n} - \ln \dfrac{n}{{n + 1}},{s_n} = \sum\limits_{n = 2}^n {\ln \dfrac{{n - 1}}{n} - \ln \dfrac{n}{{n + 1}}}  = \ln \dfrac{1}{2} - \ln \dfrac{n}{{n + 1}}$

$\sum\limits_{n = 2}^\infty  {\ln \dfrac{{{n^2} - 1}}{{{n^2}}}}  = \lim\limits_{n \to +\infty} (\ln \dfrac{1}{2} - \ln \dfrac{n}{{n + 1}}) = \ln \dfrac{1}{2}$

\begin{flalign*}
  \begin{split}
    \text{(3)解析:}\displaystyle {a_n} + {a_{n + 2}}
    & = \displaystyle\int_0^{\frac{\pi }{4}} {{{\tan }^n}} x\rd x + \displaystyle\int_0^{\frac{\pi }{4}} {{{\tan }^{n + 2}}} x\rd x
    = \displaystyle\int_0^{\frac{\pi }{4}} {{{\tan }^n}} x(1 + {\tan ^2}x)\rd x \\
    &= \displaystyle\int_0^{\frac{\pi }{4}} {{{\tan }^n}} x\rd\tan x = \dfrac{1}{{n + 1}}
  \end{split}&
\end{flalign*}

(4)解析:在判断一个级数时必须先判断他的通项是否在n趋于无穷大时趋于0 ,又$\lim\limits_{n \to +\infty} \dfrac{{\sqrt[n]{n}}}{{{{(1 + \frac{1}{n})}^n}}} = \dfrac{1}{e} \ne 0$由收敛级数的必要条件知该级数发散。

(5)解析:$\lim\limits_{n \to +\infty} {n^2}\ln (1 + \frac{1}{{{n^2}}}) = \lim\limits_{n \to +\infty} \ln {(1 + \frac{1}{{{n^2}}})^{{n^2}}} = 1$,由收敛级数的必要条件可知该级数发散.

(6)$\dfrac{1}{5} - \dfrac{1}{2} + \dfrac{1}{{10}} - \dfrac{1}{{{2^2}}} + \dfrac{1}{{15}} - \dfrac{1}{{{2^3}}}\cdots + \dfrac{1}{{5n}} - \dfrac{1}{{{2^n}}} + \cdots = \sum\limits_{n = 1}^\infty  {(\dfrac{1}{{5n}} - \dfrac{1}{{{2^n}}})} $

因为级数$\sum\limits_{n = 1}^\infty  {\dfrac{1}{{5n}}} $ 发散,而级数$\sum\limits_{n = 1}^\infty  {\dfrac{1}{{{2^n}}}} $收敛,故原级数发散.

5、解析:级数$\sum\limits_{n = 1}^\infty  {{\mu _n}} $的前2n项和为:
${S_{2n}} = \sum\limits_{n = 1}^n {({\mu _{2n - 1}} + {\mu _{2n}})}  = {\mu _1} + {\mu _2} + {\mu _3} + \cdots + {\mu _{2n - 1}} + {\mu _{2n}}.$

级数$\sum\limits_{n = 1}^\infty  {{\mu _n}} $的前$\left( {2{\text{n + 1}}} \right)$项和为:
${S_{2n + 1}} = \sum\limits_{n = 1}^n {({\mu _{2n - 1}} + {\mu _{2n}})}  + {\mu _{2n + 1}}$所以 ${S_{2n + 1}} = {S_{2n}} + {\mu _{2n + 1}}$\ding {172}.

因为级数$\sum\limits_{n = 1}^\infty  {({\mu _{2n - 1}}}  + {\mu _{2n}})$收敛于s,所以$\lim\limits_{n \to +\infty} {S_{2n}} = s$\ding{173},

又因为$\lim\limits_{n \to +\infty} {\mu _n} = 0$ ,
对\ding{172}两侧求极限得$\lim\limits_{n \to +\infty} {S_{2n + 1}} = s$\ding{174}

综合\ding{173}\ding{174}得$\lim\limits_{n \to +\infty} {S_n} = s$.证毕.


\section{常数项级数的审敛法}

\begin{flushright}
  \color{zhanqing!80}
  \ding{43} 教材见403 页 % 这里需要添加页数
\end{flushright}
1.解析:

(1)$\dfrac{1}{{4n + 1}} > \dfrac{1}{{4n + 1 + 3}} = \dfrac{1}{{4(n + 1)}}$

由$\sum\limits_{n = 1}^\infty  {\dfrac{1}{n}} $发散,得$\sum\limits_{n = 1}^\infty  {\dfrac{1}{{4(n + 1)}}} $发散.由比较审敛法得原式发散.

(3)$\dfrac{n}{{(n + 1)(n + 2)}} = \dfrac{{n + 1 - 1}}{{(n + 1)(n + 2)}} = \dfrac{1}{{n + 2}} - (\dfrac{1}{{n + 1}} - \dfrac{1}{{n + 2}}) = \dfrac{2}{{n + 2}} - \dfrac{1}{{n + 1}} > \dfrac{2}{{n + 2}} - \dfrac{1}{{n + 2}} = \dfrac{1}{{n + 2}}$

由$\sum\limits_{n = 1}^\infty  {\dfrac{1}{n}} $发散,得$\sum\limits_{n = 1}^\infty  {\dfrac{1}{{n + 2}}} $发散.由比较审敛法得原式发散.

(5)$n \geqslant 1$时,${2^n} > n,$得$\leqslant \sqrt[n]{n} \leqslant 2$.当$n = 1$时,$\sin \dfrac{\pi }{{n\sqrt[n]{n}}} = 0$

当$n \geqslant 2$时,又正弦函数在$(0,\dfrac{\pi }{2})$上单调递增,$\sin \dfrac{\pi }{{n\sqrt[n]{n}}} \geqslant \sin \dfrac{\pi }{{2n}}$.由$\sum\limits_{n = 1}^\infty  {\sin \dfrac{\pi }{n}} \sum\limits_{n = 1}^\infty  {\sin \dfrac{\pi }{{2n}}} $发散,得原式发散.

(7)$(\sqrt n  + 1)\ln (1 + \dfrac{1}{{{n^2}}}\sum\limits_{n = 1}^\infty  {\dfrac{1}{{{n^2}}}} ) = \ln {(1 + \dfrac{1}{{{n^2}}})^{\sqrt n  + 1}} \leqslant {\left( {\frac{1}{{{n^2}}}} \right)^{\sqrt n  + 1}} \leqslant \dfrac{1}{{{n^2}}}$

由p级数易得$\sum\limits_{n = 1}^\infty  {\dfrac{1}{{{n^2}}}} $收敛,所以原式收敛.

注:$\ln (1 + x) = x - \dfrac{{{x^2}}}{2} + o({x^2})$

2.解析:(1)$\lim\limits_{n \to +\infty} \dfrac{{\frac{{{{(n + 1)}^3}}}{{{3^{n + 1}}}}}}{{\frac{{{n^3}}}{{{3^n}}}}} = \lim\limits_{n \to +\infty} \dfrac{{{{(n + 1)}^3}}}{{3{n^3}}} = \dfrac{1}{3} < 1$

由比值审敛法可知原级数收敛.

(3)$\lim\limits_{n \to +\infty} \dfrac{{(n + 1)\sin \frac{\pi }{{{2^{n + 1}}}}}}{{n\sin \frac{\pi }{{{2^n}}}}} = \lim\limits_{n \to +\infty} \dfrac{{(n + 1)\frac{\pi }{{{2^{n + 1}}}}}}{{n\frac{\pi }{{{2^n}}}}} = \frac{1}{2} < 1$

由比值审敛法可知原级数收敛.

(5)$\lim\limits_{n \to +\infty} \dfrac{{\frac{{(n + 1)!}}{{n + 1}}}}{{\frac{{n!}}{n}}} = \lim\limits_{n \to +\infty} \frac{{(n + 1)n}}{{n + 1}} = \infty  > 1$

由比值审敛法可知原级数收敛.

3.解析:

(1)$\lim\limits_{n \to +\infty} \sqrt[n]{{{{(\sqrt[n]{3} - 1)}^n}}} = \lim\limits_{n \to +\infty} \left( {\sqrt[n]{3} - 1} \right) = 0 < 1$

由根值审敛法可知原级数收敛.

(3)$\lim\limits_{n \to +\infty} \sqrt[n]{{\dfrac{a}{{{{[\ln (1 + n)]}^n}}}}} = \lim\limits_{n \to +\infty} \dfrac{{\sqrt[n]{a}}}{{\ln (1 + n)}} = 0 < 1$

由根值审敛法可知原级数收敛.

(5)$\lim\limits_{n \to +\infty} \sqrt[n]{{{{\left( {\dfrac{n}{{n + 1}}} \right)}^{{n^2}}}}} = \lim\limits_{n \to +\infty} {\left( {\dfrac{n}{{n + 1}}} \right)^n} = \lim\limits_{n \to +\infty} {\left( {1 + \dfrac{{ - 1}}{{n + 1}}} \right)^{\frac{{n + 1}}{{ - 1}} \cdot \frac{{ - n}}{{n + 1}}}} = {e^{ - 1}} < 1$

由根值审敛法可知原级数收敛.

4.解析:

(1)$\dfrac{1}{{an + b}} > \dfrac{1}{{an}}$由$\sum\limits_{n = 1}^\infty  {\dfrac{1}{n}} $发散,得$\sum\limits_{n = 1}^\infty  {\dfrac{1}{{an}}} $发散.由比较审敛法得原式发散.

(3)$\lim\limits_{n \to +\infty} \dfrac{{\frac{{{{(n + 1)}^3}}}{{(n + 1)!}}}}{{\frac{{{n^3}}}{{n!}}}} = \lim\limits_{n \to +\infty} \dfrac{1}{{n + 1}}{\left( {\dfrac{{n + 1}}{n}} \right)^3} = 0 < 1$

由比值审敛法可知原级数收敛.

(5)$\sum\limits_{n = 1}^\infty  {\dfrac{{1 + {a^n}}}{{1 + {b^n}}}}  = \sum\limits_{n = 1}^\infty  {\dfrac{1}{{1 + {b^n}}}}  + \sum\limits_{n = 1}^\infty  {\dfrac{{{a^n}}}{{1 + {b^n}}}} $

当$b \leqslant 1$时,${b^n} \leqslant 1,\dfrac{1}{{1 + {b^n}}} \geqslant \dfrac{1}{2},\sum\limits_{n = 1}^\infty  {\dfrac{1}{{1 + {b^n}}}} $发散.

当$b > 1$时,$\dfrac{1}{{1 + {b^n}}} < \frac{1}{{{b^n}}}\sum\limits_{n = 1}^\infty  {\dfrac{1}{{1 + {b^n}}}} $收敛.

(a)$\dfrac{{{a^n}}}{{1 + {b^n}}} < \dfrac{{{a^n}}}{{{b^n}}}$,易得$a < b$时,$\sum\limits_{n = 1}^\infty  {\dfrac{{{a^n}}}{{1 + {b^n}}}} $收敛

(b)$a \geqslant b$时,$\lim\limits_{n \to +\infty} \dfrac{{{a^n}}}{{1 + {b^n}}} \geqslant \lim\limits_{n \to +\infty} \dfrac{{{b^n}}}{{1 + {b^n}}} = 1,\sum\limits_{n = 1}^\infty  {\dfrac{{{a^n}}}{{1 + {b^n}}}} $发散.

综上可知当$b > 1$且$0 < a < b$时原式收敛,其余情况皆发散.

5.解析:

(1)$\dfrac{1}{{\sqrt n }} > \frac{1}{{\sqrt {n + 1} }},\lim\limits_{n \to +\infty} \dfrac{1}{{\sqrt n }} = 0$,由莱布尼兹审敛法得原式收敛.由p级数易得$\sum\limits_{n = 1}^\infty  {\dfrac{1}{{\sqrt n }}} $发散,所以原式条件收敛.

(3)$\sin \dfrac{1}{n} > \sin \dfrac{1}{{n + 1}},\lim\limits_{n \to +\infty} \sin \dfrac{1}{n} = 0$由莱布尼兹审敛法得原式收敛.$\sum\limits_{n = 1}^\infty  {\sin \dfrac{1}{n}} $发散易得原式条件收敛.

(5)$\sum\limits_{n = 1}^\infty  {\dfrac{1}{{{n^2} + 1}}} $收敛.

$\dfrac{1}{n} > \dfrac{1}{{n + 1}},\lim\limits_{n \to +\infty} \dfrac{1}{n} = 0$,由莱布尼兹审敛法得$\sum\limits_{n = 1}^\infty  {{{( - 1)}^{n + 1}}\dfrac{1}{n}} $收敛,则原式收敛.

$\dfrac{1}{n} - \dfrac{1}{{{n^2} + 1}} > \dfrac{1}{n}\sum\limits_{n = 1}^\infty  {\dfrac{1}{n}} $发散,得原级数条件发散.

6.解析:$\dfrac{{\sqrt {{a_n}} }}{n} = \sqrt {\dfrac{{{a_n}}}{{{n^2}}}}  \leqslant \dfrac{1}{2}({a_n} + \dfrac{1}{{{n^2}}})$

$\sum\limits_{n = 1}^\infty  {{a_n}} $收敛$,\sum\limits_{n = 1}^\infty  {\dfrac{1}{{{n^2}}}} $收敛,故$\sum\limits_{n = 1}^\infty  {\dfrac{1}{2}({a_n} + \dfrac{1}{{{n^2}}})} $收敛.$\therefore \sum\limits_{n = 1}^\infty  {\dfrac{{\sqrt {{a_n}} }}{n}} $收敛.

7.证明:

(1)必要性:$\sum\limits_{n = 1}^\infty  {{a_n}} $绝对收敛,则$\sum\limits_{n = 1}^\infty  {\left| {{a_n}} \right|} $收敛,且$ - \sum\limits_{n = 1}^\infty  {\left| {{a_n}} \right|} $也收敛.

又$ - \left| {{a_n}} \right| \leqslant {a_n}^ +  \leqslant \left| {{a_n}} \right|, - \left| {{a_n}} \right| \leqslant {a_n}^ -  \leqslant \left| {{a_n}} \right|$,故$\sum\limits_{n = 1}^\infty  {a_n^ + } $与$\sum\limits_{n = 1}^\infty  {a_n^ - } $同时收敛.

充分性:正部与负部同时收敛,即$\sum\limits_{n = 1}^\infty  {a_n^ + } ,\sum\limits_{n = 1}^\infty  {a_n^ - } $收敛.

$\therefore \sum\limits_{n = 1}^\infty  {a_n^ + }  + \sum\limits_{n = 1}^\infty  {a_n^ - }  = \sum\limits_{n = 1}^\infty  {{a_n}} $收敛.

且$\sum\limits_{n = 1}^\infty  {a_n^ + }  - \sum\limits_{n = 1}^\infty  {a_n^ - }  = \sum\limits_{n = 1}^\infty  {\left| {{a_n}} \right|} $收敛,故而$\sum\limits_{n = 1}^\infty  {{a_n}} $绝对收敛.

$\sum\limits_{n = 1}^\infty  {{a_n}} $绝对收敛的充要条件是其正部与负部同时收敛,得证.

(2)必要性:$\sum\limits_{n = 1}^\infty  {{a_n}} $条件收敛,即$\sum\limits_{n = 1}^\infty  {\left| {{a_n}} \right|} $发散,且$\sum\limits_{n = 1}^\infty  {{a_n}} $收敛.

反证法:如果其正部与负部不同时发散,有下列情况:

(a)同时收敛,此时$\sum\limits_{n = 1}^\infty  {{a_n}} $绝对收敛,这与$\sum\limits_{n = 1}^\infty  {{a_n}} $条件收敛矛盾.

(b)一格收敛一个发散,得正部与负部之和发散,即$\sum\limits_{n = 1}^\infty  {{a_n}} $发散,这与$\sum\limits_{n = 1}^\infty  {{a_n}} $条件收敛矛盾.

综上得证$\sum\limits_{n = 1}^\infty  {{a_n}} $条件收敛的必要条件是其正部与负部同时发散.

充分性:正部与负部同时发散,而$\sum\limits_{n = 1}^\infty  {{a_n}} $收敛,当且仅当

\[\sum\limits_{n = 1}^\infty  {{a_n}}  = \sum\limits_{n = 1}^\infty  {a_n^ + }  + \sum\limits_{n = 1}^\infty  {a_n^ - }  = 0\]

时成立,此时$\sum\limits_{n = 1}^\infty  {a_n^ + }  =  - \sum\limits_{n = 1}^\infty  {a_n^ - } $,故有

\[\sum\limits_{n = 1}^\infty  {\left| {{a_n}} \right|}  = \sum\limits_{n = 1}^\infty  {a_n^ + }  - \sum\limits_{n = 1}^\infty  {a_n^ - }  = 2\sum\limits_{n = 1}^\infty  {a_n^ + }  =  - 2\sum\limits_{n = 1}^\infty  {a_n^ - } \]

$\therefore \sum\limits_{n = 1}^\infty  {\left| {{a_n}} \right|} $发散,$\sum\limits_{n = 1}^\infty  {{a_n}} $条件收敛.$\sum\limits_{n = 1}^\infty  {{a_n}} $条件收敛的充要条件是其正部与负部同时发散.

得证.

\section{幂级数}
\begin{flushright}
  \color{zhanqing!80}
  \ding{43} 教材见415 页 % 这里需要添加页数
\end{flushright}
1.解析:

(1)原式在$x = 0$处收敛,且$\lim\limits_{n \to +\infty} \left| {\dfrac{{{n^2} + n}}{{{{(n + 1)}^2} + n + 1}}} \right| = 1$

在$x =  \pm 1$处$\left| {\dfrac{1}{{{n^2} + n}}} \right| < \dfrac{1}{{{n^2}}}$,所以收敛.因此,收敛域为$[ - 1,1]$.

(3)令$t = {(x + 1)^2}$,原式变为$\sum\limits_{n = 1}^\infty  {{2^n}{t^n}} $,该式在$(0,\dfrac{1}{2})$上收敛,所以原式在$( - 1 - \dfrac{{\sqrt 2 }}{2}, - 1 + \dfrac{{\sqrt 2 }}{2})$收敛.

(5)令$t = x + 1$,$\lim\limits_{n \to +\infty} \left| {\dfrac{{{u_{n + 1}}}}{{{u_n}}}} \right| = \dfrac{{{n^p}}}{{{{(n + 1)}^p}}} = 1$,当$1 \geqslant p > 0$
时,在$t = 1$处发散,$p < 1$时,在$t = 1$处收敛,$t = - 1$时函数都收敛.

因此原函数的收敛域为
\[\left\{ {\begin{array}{*{20}{c}}
  {\left[ { - 2,0} \right),0 < p \leqslant 1} \\
  {\left[ { - 2,0} \right],p > 1}
\end{array}} \right.\]
(7)设$t = {x^3}$$\lim\limits_{n \to +\infty} \left| {\dfrac{{{u_{n + 1}}}}{{{u_n}}}} \right| = \dfrac{1}{2}$,在$t = 2$时收敛,在$t = - 2$时发散,因此原式收敛域为$( - \sqrt[3]{2},\sqrt[3]{2}]$

2.解析:

(1)当$x = 0$时,$s(x) = 0$;当$x \ne 0$时,$s(x) = {x^2}\sum\limits_{n = 0}^\infty  {{e^{ - nx}}}  = \dfrac{{{x^2}}}{{1 - {e^{ - x}}}}$

(3)$s(x) = \sum\limits_{n = 1}^\infty  {{{( - 1)}^{n + 1}}\displaystyle\int {{x^{2n - 2}}} }  = \displaystyle\int {\sum\limits_{n = 1}^\infty  {{{( - 1)}^{n + 1}}{x^{2n - 2}}} }  = \displaystyle\int {\dfrac{1}{{1 + {x^2}}}}  = \arctan x $$x \in [ - 1,1]$

(5)当$x = 1$时,$s(x) = \sum\limits_{n = 1}^\infty  {\dfrac{1}{{n(n + 1)}}}  = 1$

$x \ne 1$时,$s(x) = \sum\limits_{n = 1}^\infty  {\displaystyle\int {\dfrac{{{x^n}}}{n}} }  = \sum\limits_{n = 1}^\infty  {\displaystyle\iint {{x^{n - 1}}}  }  = \displaystyle\int {\sum\limits_{n = 1}^\infty  {{x^{n - 1}}} }   = (1 - x)\ln (1 - x) + x,x \in [ - 1,1)$

3.解析:

(1)$x =  \pm 1$时,级数发散;

$x \ne  \pm 1$时,$rho  = \lim\limits_{n \to +\infty} \left| {\dfrac{{{u_{n + 1}}(x)}}{{{u_n}(x)}}} \right| = \left| {\dfrac{{x(1 + {x^{2n}})}}{{1 + {x^{2n + 2}}}}} \right|$

$\left| x \right| < 1$时,$\rho  = \left| x \right| < 1$$\left| x \right| > 1$时,$\rho  = 0 < 1$

因此,级数收敛域为$( - \infty , - 1) \cup ( - 1,1) \cup (1, + \infty )$

(2)$\rho  = \lim\limits_{n \to +\infty} \left| {\frac{{{u_{n + 1}}(x)}}{{{u_n}(x)}}} \right| = \left| {\dfrac{{{x^2} + {n^3}}}{{{x^2} + {{(n + 1)}^3}}}} \right| = 1$因此,收敛域为$( - \infty , + \infty )$

4.解析:$\sum\limits_{n = 0}^\infty  {\dfrac{{{n^2} + 1}}{{{2^n}n!}}{x^n}}  = \sum\limits_{n = 0}^\infty  {\dfrac{{{n^2}}}{{n!}}{{\left( {\frac{x}{2}} \right)}^n}}  + \sum\limits_{n = 0}^\infty  {\dfrac{1}{{n!}}{{\left( {\dfrac{x}{2}} \right)}^n}} $

由$\sum\limits_{n = 0}^\infty  {\dfrac{{{x^n}}}{{n!}}}  = {e^x}$得$\sum\limits_{n = 0}^\infty  {\dfrac{1}{{n!}}{{\left( {\dfrac{x}{2}} \right)}^n}}  = {e^{\frac{x}{2}}},\sum\limits_{n = 0}^\infty  {\dfrac{{{n^2}}}{{n!}}{{\left( {\dfrac{x}{2}} \right)}^n}}  = \dfrac{x}{2}\sum\limits_{n = 1}^\infty  {\frac{n}{{(n - 1)!}}{{\left( {\dfrac{x}{2}} \right)}^{n - 1}}} $

令$G(x) = \sum\limits_{n = 1}^\infty  {\dfrac{n}{{(n - 1)!}}{{\left( {\dfrac{x}{2}} \right)}^{n - 1}}} ,$
\begin{flalign*}
    \begin{split}
    \displaystyle \text {则}\displaystyle\int {G(x)\rd x}
    &= \dfrac{1}{{(n - 1)!}}\sum\limits_{n = 1}^\infty  {\displaystyle\int {n{{\left( {\dfrac{x}{2}} \right)}^{n - 1}}\rd x} } \\
    &= \dfrac{2}{{(n - 1)!}}\sum\limits_{n = 1}^\infty  {{{\left( {\frac{x}{2}} \right)}^n}}  + C = 2 \cdot \dfrac{x}{2} \cdot \sum\limits_{n = 1}^\infty  {\dfrac{1}{{(n - 1)!}}{{\left( {\dfrac{x}{2}} \right)}^{n - 1}} + C}  = x{e^{\frac{x}{2}}} + C\\
    \end{split}&
\end{flalign*}
$\therefore \sum\limits_{n = 0}^\infty  {\dfrac{{{n^2}}}{{n!}}{{\left( {\frac{x}{2}} \right)}^n}}  = \dfrac{x}{2} \cdot G'(x) = (\dfrac{{{x^2}}}{4} + \dfrac{x}{2}){e^{\frac{x}{2}}}$

$\therefore \sum\limits_{n = 0}^\infty  {\dfrac{{{n^2} + 1}}{{{2^n}n!}}{x^n}}  = (\dfrac{{{x^2}}}{4} + \dfrac{x}{2}){e^{\frac{x}{2}}} + {e^{\frac{x}{2}}} = (\dfrac{{{x^2}}}{4} + \dfrac{x}{2} + 1){e^{\frac{x}{2}}}.$

3.解析:

(1)$s(x){\text{ = }}\sum\limits_{n = 1}^\infty  {\dfrac{{{{( - 1)}^{n - 1}}}}{{n\left( {2n - 1} \right)}}} {x^n} $

$s(x) = \sum\limits_{n = 1}^\infty  {\dfrac{{{{( - 1)}^{n - 1}}}}{{n - \frac{1}{2}}}} {x^n} - \sum\limits_{n = 1}^\infty  {\dfrac{{{{( - 1)}^{n - 1}}}}{n}} {x^n}$

设$u(x) = \sum\limits_{n = 1}^\infty  {\dfrac{{{{( - 1)}^{n - 1}}}}{{n - \dfrac{1}{2}}}} {x^n}$,$v(x) = \sum\limits_{n = 1}^\infty  {\dfrac{{{{( - 1)}^{n - 1}}}}{n}} {x^n}$

设$x = {t^2}$

$u({t^2}) = \sum\limits_{n = 1}^\infty  {\dfrac{{2{{( - 1)}^{n - 1}}}}{{2n - 1}}} {t^{2n}} = \sum\limits_{n = 1}^\infty  {t\dfrac{{2{{( - 1)}^{n - 1}}}}{{2n - 1}}} {t^{2n - 1}} = 2t\sum\limits_{n = 1}^\infty  {\displaystyle\int {{{( - {t^2})}^{n - 1}}} }  = 2t\displaystyle\int {\dfrac{1}{{1 + {t^2}}}}\rd t  = 2t\arctan t$

$v(x) = \sum\limits_{n = 1}^\infty  {\dfrac{{{{( - 1)}^{n - 1}}}}{n}} {x^n} = \sum\limits_{n = 1}^\infty  {\displaystyle\int {{{\left( { - x} \right)}^{n - 1}}} }  = \ln (1 + x)$

$s(x) = 2\sqrt x \arctan \sqrt x  - \ln (1 + x),s(\dfrac{1}{3}) = \dfrac{\pi }{{3\sqrt 3 }} - \ln \dfrac{4}{3}$

(3)令$s(x) = \sum\limits_{n = 1}^\infty  {(2n - 1)} {(x)^n},x = {t^2}$

$s({t^2}) = \sum\limits_{n = 1}^\infty  {(2n - 1)} {t^{2n}} = {t^2}\sum\limits_{n = 1}^\infty  {(2n - 1)} {t^{2n - 2}} = {t^2}\sum\limits_{n = 1}^\infty  {{{({t^{2n - 1}})}'}}  = {t^2}(\dfrac{t}{{1 - {t^2}}}){t^{2n - 1}} = {t^2}\dfrac{{1 + {t^2}}}{{{{(1 - t)}^2}}}{t^2}$

\[s(\dfrac{1}{2}) = \dfrac{1}{2} + \dfrac{3}{4} + \dfrac{5}{8} + \dfrac{7}{{16}} +  \cdot  \cdot  \cdot  = 3\]

\section{函数展开成幂级数及其应用}
\begin{flushright}
  \color{zhanqing!80}
  \ding{43} 教材见433 页 % 这里需要添加页数
\end{flushright}
1.解析:

(1)${e^x} = 1 + x + \dfrac{{{x^2}}}{{2!}} + \dfrac{{{x^3}}}{{3!}} + \cdots \ding{173} $

${e^{ - x}} = 1 + x + \dfrac{{{{( - x)}^2}}}{{2!}} + \dfrac{{{{( - x)}^3}}}{{3!}} + \cdots \ding{174} ']]$

$\mathrm{sh} x = \sum\limits_{n = 1}^\infty  {\dfrac{{{x^{2n - 1}}}}{{(2n - 1)!}}} ,x \in ( - \infty , + \infty )$

(3)${\sin ^2}x = \dfrac{{1 - \cos 2x}}{2} = \sum\limits_{n = 1}^\infty  {{{( - 1)}^{n - 1}}\dfrac{{{{(2x)}^{2n}}}}{{2(2n)!}}} ,x \in ( - \infty , + \infty )$

(5)$\dfrac{1}{{{{(1 + x)}^2}}} =  - (\dfrac{1}{{1 + x}})' =  - (\sum\limits_{n = 0}^\infty  {{{( - 1)}^n}{x^n}} )' = \sum\limits_{n = 0}^\infty  {{{( - 1)}^n}(n + 1){x^n}} ,x \in ( - 1,1)$

(7)$((1 + x)\ln (1 + x))' =$
$\ln (1 + x) + 1 = 1 + \sum\limits_{n = 0}^\infty  {\dfrac{{{{( - 1)}^n}{x^{n + 1}}}}{{n + 1}}}$ \\
$(1 + x)\ln (1 + x)  = x + \sum\limits_{n = 0}^\infty  {\dfrac{{{{( - 1)}^n}{x^{n + 2}}}}{{(n + 1)(n + 2)}},} x \in ( - 1,1]$

2.解析'
'
\begin{flalign*}
    \begin{split}
    (1)\sqrt x
    &= {(1 + (x - 1))^{{\raise0.7ex\hbox{$1$} \!\mathord{\left/
   {\vphantom {1 2}}\right.\kern-\nulldelimiterspace}
   \!\lower0.7ex\hbox{$2$}}}} = 1 + \dfrac{{x - 1}}{2} + \sum\limits_{n = 2}^\infty  {\dfrac{{\frac{1}{2} \times ( - \frac{1}{2}) \times ( - \frac{3}{2}) \times \cdots( - \frac{{2n - 1}}{2})}}{{n!}}{{(x - 1)}^n}  } \\
   &=1 + \dfrac{{x - 1}}{2} + \sum\limits_{n = 2}^\infty  {{{( - 1)}^{n - 1}}\dfrac{{(2n - 3)!!}}{{(2n)!!}}{{(x - 1)}^n}} ,x \in [0,2]\\
   \end{split}&
\end{flalign*}
\begin{flalign*}
     \begin{split}
     (3)\cos x
     &= \cos (x + \dfrac{\pi }{3} - \dfrac{\pi }{3}) = \dfrac{1}{2}\cos (x + \dfrac{\pi }{3}) + \dfrac{{\sqrt 3 }}{2}\sin (x + \dfrac{\pi }{3})\\
     &= \dfrac{1}{2}\sum\limits_{n = 0}^\infty  {\left[ {\dfrac{{{{(x + \frac{\pi }{3})}^{2n}}}}{{(2n)!}} + \sqrt 3 \dfrac{{{{(x + \frac{\pi }{3})}^{2n + 1}}}}{{(2n + 1)!}}} \right]},x \in ( - \infty , + \infty ) \\
    \end{split}&
\end{flalign*}
(5)$\sin 2x =  - \sin 2(x - \dfrac{\pi }{2}) = \sum\limits_{n = 0}^\infty  {\dfrac{{{{( - 1)}^{n + 1}}{2^{2n + 1}}{{(x - \frac{\pi }{2})}^{2n + 1}}}}{{(2n + 1)!}}} ,x \in ( - \infty , + \infty )$
\section{傅里叶级数}
\begin{flushright}
  \color{zhanqing!80}
  \ding{43} 教材见 449页 % 这里需要添加页数
\end{flushright}
1.解析:由题意可知,$f(x)$在$x = k\pi (k = 0, \pm 1, \pm 2 \ldots )$处有间断点,满足收敛定理,

$\therefore f(\pi ) = \dfrac{1}{2}(f({\pi ^ + }) - f({\pi ^ - })) = \dfrac{1}{2}( - 1 + 1 + {\pi ^2}) = \dfrac{1}{2}{\pi ^2}$

同理可知$f(5\pi ) = \dfrac{1}{2}{\pi ^2}$

2.解析:

(1)${a_0} = 2\displaystyle\int_{ - \frac{1}{2}}^{\frac{1}{2}} {(1 - {x^2})} \rd x = 2 \times 2\displaystyle\int_0^{\frac{1}{2}} {(1 - {x^2})} \rd x = 4 \times (\frac{1}{2} - \dfrac{1}{{24}}) = \dfrac{{11}}{6}$
\begin{flalign*}
  \begin{split}
    {a_1}
    &= 2\displaystyle\int_{ - \frac{1}{2}}^{\frac{1}{2}} {(1 - {x^2})\cos 2n} \pi \rd x
    = 2 \times 2\displaystyle\int_0^{\frac{1}{2}} {(1 - {x^2})\cos 2n} \pi \rd x
    = 4(\displaystyle\int_0^{\frac{1}{2}} {\cos 2n} \pi \rd x - \displaystyle\int_0^{\frac{1}{2}} {{x^2}\cos 2n} \pi \rd x \\
    &= 4(0 - \dfrac{1}{{4{n^2}{\pi ^2}}}\cos n\pi ) =  - \dfrac{1}{{{n^2}{\pi ^2}}}\cos n\pi
    = \dfrac{{{{( - 1)}^{n + 1}}}}{{{n^2}{\pi ^2}}}\\
  \end{split}&
\end{flalign*}
${b_n} = 0$

\[\therefore f(x) = \dfrac{{11}}{{12}} + \dfrac{1}{{{\pi ^2}}}\sum\limits_{n = 1}^\infty  {\dfrac{{{{( - 1)}^{n + 1}}}}{{{n^2}{\pi ^2}}}} \cos 2n\pi x\]

(3)$f(x)$为偶函数,$\therefore {b_n} = 0$

${a_0} = \dfrac{1}{\pi }\displaystyle\int_{ - \pi }^\pi  {(3{x^2} + 1)} \rd x = \dfrac{1}{\pi }[2{\pi ^3} + 2\pi ] = 2{\pi ^2} + 2$

${a_n} = \dfrac{1}{\pi }\displaystyle\int_{ - \pi }^\pi  {(3{x^2} + 1)} \cos nx\rd x = \dfrac{2}{\pi }\displaystyle\int_0^\pi  {(3{x^2} + 1)\cos nx\rd x} = \dfrac{{12{{( - 1)}^n}}}{{{n^2}}}$

\[\therefore f(x) = {\pi ^2} + 1 + \sum\limits_{n = 1}^\infty  {\dfrac{{12{{( - 1)}^n}}}{{{n^2}}}\cos nx} \]

3.解析:$2l = 2 - 0 = 2,\therefore l = 1$

将$f(x)$在$[0,2]$外补充定义,将$f(x)$延拓为周期为2的周期函数

$\therefore {a_0} = \displaystyle\int_0^2 {f(x)} \rd x = \displaystyle\int_0^1 {x\rd x}  = \dfrac{1}{2}$

${a_n} = \displaystyle\int_0^2 {f(x)\cos n\pi x} \rd x = \displaystyle\int_0^1 {x\cos n\pi x\rd x}  = d\frac{{{{( - 1)}^n} - 1}}{{{n^2}{\pi ^2}}}$

${b_n} = \displaystyle\int_0^2 {f(x)\sin n\pi x} \rd x = \displaystyle\int_0^1 x \sin n\pi \rd x = \dfrac{{{{( - 1)}^{1 + n}}}}{{n\pi }}$

$\therefore f(x) = \dfrac{1}{4} + \sum\limits_{n = 1}^\infty  {\dfrac{1}{{n\pi }}} [\dfrac{{{{( - 1)}^n} - 1}}{{n\pi }}\cos n\pi x + {( - 1)^{n + 1}}\sin n\pi x]$

$f(0) = \dfrac{1}{4} + \sum\limits_{n = 1}^\infty  {\dfrac{1}{{n\pi }}}  \cdot \dfrac{{{{( - 1)}^n} - 1}}{{n\pi }} = \dfrac{1}{4} - \dfrac{2}{{{\pi ^2}}}\sum\limits_{n = 0}^\infty  {\dfrac{1}{{{{(2n + 1)}^2}}}}  = 0$

$\therefore \sum\limits_{n = 0}^\infty  {\dfrac{1}{{{{(2n + 1)}^2}}}}  = \dfrac{{{\pi ^2}}}{8}.$

5.解析:

(1)将$f(x)$偶延拓得
${b_n} = 0$, ${a_0} = \dfrac{2}{\pi }\displaystyle\int_0^\pi  {(\dfrac{\pi }{2} - x)} \rd x = \dfrac{2}{\pi }(\dfrac{{{\pi ^2}}}{2} - \dfrac{{{\pi ^2}}}{2})= 0$
\begin{flalign*}
  \begin{split}
    {a_n}
    &= \dfrac{2}{\pi }\displaystyle\int_0^\pi  {(\dfrac{\pi }{2} - x)} \cos nx\rd x
    = \dfrac{2}{\pi }[0 - \dfrac{1}{{{n^2}}}(\cos n\pi  - 1)]
    = \dfrac{2}{\pi }[\displaystyle\int_0^\pi  {\dfrac{\pi }{2}} \cos nx\rd x - \displaystyle\int_0^\pi  x \cos nx\rd x] \\
    & =  - \frac{2}{{\pi {n^2}}}(\cos n\pi  - 1)
    = \left\{ \begin{gathered}
  \frac{4}{{\pi {n^2}}}(n = 2k) \hfill \\
  0(n = 2k) \hfill \\
  \end{gathered}  \right.(k =  \pm 1, \pm 2 \ldots )\\
    \end{split}&
\end{flalign*}
$\therefore f(x) = \dfrac{4}{\pi }\sum\limits_{k = 1}^\infty  {\dfrac{1}{{{{(2k - 1)}^2}}}} \cos (2k - 1)x$.

(3)a.正弦级数
\begin{flalign*}
    \begin{split}
    {b_n}
    &= \dfrac{2}{l}\displaystyle\int0^l {f(x)} \sin \dfrac{{n\pi x}}{l}\rd x
    = \dfrac{2}{l}(\displaystyle\int0^{\frac{l}{2}} x \sin \dfrac{{n\pi x}}{l}\rd x + \displaystyle\int{\frac{l}{2}}^l {(l - x)} \sin \dfrac{{n\pi x}}{l}\rd x)\\
    & = \int \dfrac{{4l}}{{{n^2}{\pi ^2}}}\sin \dfrac{{n\pi }}{2}\rd x
    = \dfrac{{4l{{( - 1)}^{k - 1}}}}{{{\pi ^2}(2k - 1)}} \\
    & \therefore f(x) = \frac{{4l}}{{{\pi ^2}}}
    \end{split}&
\end{flalign*}
$\therefore f(x) = \dfrac{{4l}}{{{\pi ^2}}}\sum\limits_{k = 1}^\infty  {\dfrac{{{{( - 1)}^{k - 1}}}}{{{{(2k - 1)}^2}}}} \sin \dfrac{{(2k - 1)\pi x}}{l}$.

b.余弦级数
  \[ {a_0} = \dfrac{2}{l}\displaystyle\int0^l {f(x)} \rd x
    = \dfrac{2}{l}(\displaystyle\int0^{\frac{l}{2}} {x\rd x + \displaystyle\int{\frac{l}{2}}^l {(l - x)} } \rd x)
    = \dfrac{l}{2} \]
  \[ {a_n} = \dfrac{2}{l}\displaystyle\int_0^l {f(x)} \cos \dfrac{{n\pi x}}{l}\rd x
    = \dfrac{2}{l}(\displaystyle\int_0^{\frac{l}{2}} x \cos \dfrac{{n\pi x}}{l}\rd x + \displaystyle\int_{\frac{l}{2}}^l {(l - x)} \cos \dfrac{{n\pi x}}{l}dx)
    = \dfrac{{{{( - 1)}^{k - 1}}l}}{{{\pi ^2}{{(2k - 1)}^2}}} \]
$$\therefore f(x) = \dfrac{l}{4} + \dfrac{l}{{{\pi ^2}}}\sum\limits_{k = 1}^\infty  {\dfrac{{{{( - 1)}^{k - 1}}}}{{{{(2k - 1)}^2}}}} \cos \dfrac{{(2k - 1)\pi x}}{l}$$

7.解析:令$f(x) = x(\pi  - x)$,对其进行奇延拓,得:
\begin{flalign*}
    \begin{split}
    {b_n} &= \dfrac{2}{\pi }\displaystyle\int_0^\pi  {x(\pi  - x)} \sin nx\rd x
    = \dfrac{2}{\pi }[\displaystyle\int_0^\pi  {x\pi } \sin nx\rd x - \displaystyle\int_0^\pi  {{x^2}} \sin nx\rd x]\\
    & = \dfrac{2}{\pi }[ - \dfrac{{{\pi ^2}}}{n}\cos n\pi  + \dfrac{{{\pi ^2}}}{n}\cos n\pi  - \dfrac{2}{{{n^3}}}\cos n\pi  + \dfrac{2}{{{n^3}}}]
    = \dfrac{2}{\pi }[\dfrac{2}{{{n^3}}} - \dfrac{2}{{{n^3}}}\cos n\pi ]\\
    & = \dfrac{4}{{{n^3}\pi }}[1 - {( - 1)^n}] = \left\{ \begin{gathered}
  \dfrac{8}{{{n^3}\pi }}(n = 2k - 1) \hfill \\
  0(n = 2k) \hfill \\
\end{gathered}  \right.k = 0, \pm 1, \pm 2 \ldots \\
    \end{split}
\end{flalign*}
$\therefore f(x) = \dfrac{8}{\pi }\sum\limits_{k = 1}^\infty  {\dfrac{{\sin (2k - 1)x}}{{{{(2k - 1)}^3}}}} $

8.解析:将$f(x)$偶延拓得:

$f(x) = \dfrac{{{a_0}}}{2} + \sum\limits_{n = 1}^\infty  {{a_n}\cos nx} $

$f(x + \dfrac{\pi }{2}) = \dfrac{{{a_0}}}{2} + \sum\limits_{n = 1}^\infty  {{a_n}\cos n(x + \dfrac{\pi }{2})}  = \dfrac{{{a_0}}}{2} + \sum\limits_{n = 1}^\infty  {{a_n}\cos (nx + \dfrac{\pi }{2}n)} $

$f(x - \dfrac{\pi }{2}) = \dfrac{{{a_0}}}{2} + \sum\limits_{n = 1}^\infty  {{a_n}\cos n(x - \dfrac{\pi }{2})} = \dfrac{{{a_0}}}{2} + \sum\limits_{n = 1}^\infty  {{a_n}\cos (nx - \dfrac{\pi }{2}n)} $

$ - f(x - \dfrac{\pi }{2}) =  - \dfrac{{{a_0}}}{2} + \sum\limits_{n = 1}^\infty  {{a_n}\cos (nx - \dfrac{\pi }{2}n) \times ( - 1)} $

$\because f(x + \dfrac{\pi }{2}) =  - f(x - \dfrac{\pi }{2}),\therefore {a_0} = 0$

$\therefore \cos (nx + \dfrac{\pi }{2}n) =  - \cos (nx - \dfrac{\pi }{2}n) \Rightarrow \cos (nx + \dfrac{\pi }{2}n) + \cos (nx - \dfrac{\pi }{2}n) = 0$

$\therefore 2\cos nx\cos \dfrac{{n\pi }}{2} = 0 \Rightarrow \cos \dfrac{{n\pi }}{2} = 0$

$\therefore n = 2k + 1(k = 0, \pm 1, \pm 2 \ldots )$

即$f(x)$得所有偶数项为0,即${a_{2k}} = 0$

\section{总习题五}
\begin{flushright}
  \color{zhanqing!80}
  \ding{43} 教材见 459页 % 这里需要添加页数
\end{flushright}

2.解析:

(1)$\lim\limits_{n \to +\infty} \dfrac{{\sqrt[n]{3} - 1}}{{\frac{1}{n}}} = \lim\limits_{n \to +\infty} n\left( {\sqrt[n]{3} - 1} \right) = \lim\limits_{n \to +\infty} n\dfrac{1}{n}\ln 3 = \ln 3 > 0$

所以级数$\sum\limits_{n = 1}^\infty  {\dfrac{1}{n}} $与级数$\sum\limits_{n = 1}^\infty  {\left( {\sqrt[n]{3} - 1} \right)} $有相同的敛散性

又因为调和级数$\sum\limits_{n = 1}^\infty  {\dfrac{1}{n}} $发散,所以级数$\sum\limits_{n = 1}^\infty  {\left( {\sqrt[n]{3} - 1} \right)} $发散.

(3)$\lim\limits_{n \to +\infty} \dfrac{{\frac{1}{{{{\ln }^5}n}}}}{{\frac{1}{n}}} = \lim\limits_{n \to +\infty} \frac{n}{{{{\ln }^5}n}} =  + \infty $(由洛必达法则易知)

又因为调和级数$\sum\limits_{n = 1}^\infty  {\dfrac{1}{n}} $,所以级数$\sum\limits_{n = 1}^\infty  {\dfrac{1}{n}} $

(5)$\lim\limits_{n \to +\infty} \dfrac{{{u_{n + 1}}}}{{{u_n}}} = \lim\limits_{n \to +\infty} \dfrac{{{a^{_{n + 1}}}{n^s}}}{{{a^n}{{(n + 1)}^s}}} = a
\left\{ \begin{gathered}
   > 1, \text {级数发散}\hfill  \\
   = 1 \hfill \\
   < 1, \text {级数收敛}\hfill \\
\end{gathered}  \right.$

当$a = 1$时,级数为$\sum\limits_{n = 1}^\infty  {\dfrac{1}{{{n^s}}}} $p级数,当$s > 1$时,级数收敛;当$s \leqslant 1$时,级数发散.

3.解析:

(1)因为$\left| {{u_n}} \right| = \dfrac{{\sin (n + 2)}}{{{\pi ^n}}} \leqslant \dfrac{1}{{{\pi ^n}}}$,级数$\sum\limits_{n = 1}^\infty  {\frac{1}{{{\pi ^n}}}}$收敛

所以级数 $\sum\limits_{n = 1}^\infty  {\left| {{u_n}} \right|}$收敛,级数$ \sum\limits_{n = 1}^\infty  {{u_n}} $绝对收敛.

(3)因为$\lim\limits_{n \to +\infty} \left| {\dfrac{{\cos \frac{1}{n} - n\sin \frac{1}{n}}}{{\dfrac{1}{{{n^2}}}}}} \right| = \lim\limits_{n \to +\infty} \left| {{n^2}\cos \frac{1}{n} - {n^3}\sin \dfrac{1}{n}} \right|
= \lim\limits_{n \to +\infty} \left| {{n^2}\cos \dfrac{1}{n} - {n^3}\sin \dfrac{1}{n}} \right|$

$\xrightarrow{{x = \dfrac{1}{n}}}\mathop {\lim }\limits_{x \to 0} \left| {\dfrac{{x\cos x - \sin x}}{{{x^3}}}} \right| = \dfrac{2}{3} > 0$

所以级数$\sum\limits_{n = 1}^\infty  {\dfrac{1}{{{n^2}}}} $和级数$\sum\limits_{n = 1}^\infty  {\left| {\cos \dfrac{1}{n} - n\sin \dfrac{1}{n}} \right|} $有相同的敛散性;

又因为级数$\sum\limits_{n = 1}^\infty  {\dfrac{1}{{{n^2}}}} $收敛,所以级数$\sum\limits_{n = 1}^\infty  {\left| {\cos \dfrac{1}{n} - n\sin \dfrac{1}{n}} \right|} $收敛

所以级数$\sum\limits_{n = 1}^\infty  {\left( {\cos \dfrac{1}{n} - n\sin \dfrac{1}{n}} \right)} $
绝对收敛.

(5)因为${u_n} = {( - 1)^n}\ln \dfrac{{n + 1}}{n},\left| {{u_n}} \right| = \ln \dfrac{{n + 1}}{n} = \ln (n + 1) - \ln n$

级数$\sum\limits_{n = 1}^\infty  {\left| {{u_n}} \right|} $的部分和为
$\lim\limits_{n \to +\infty} {s_n} = \lim\limits_{n \to +\infty} [\ln (n + 1) - \ln n + \ln n - \ln (n - 1) +  \ldots  + \ln 2 - \ln 1] = \lim\limits_{n \to +\infty} \ln (n + 1) =  + \infty $

所以级数$\sum\limits_{n = 1}^\infty  {\left| {{u_n}} \right|} $发散;

又$\left| {{u_n}} \right| \geqslant \left| {{u_{n + 1}}} \right|$,
且$\lim\limits_{n \to +\infty} \left| {{u_n}} \right| = 0$,所以交错级数$\sum\limits_{n = 1}^\infty  {{u_n}} $收敛;

综上所述,级数$\sum\limits_{n = 1}^\infty  {{u_n}} $条件收敛.

