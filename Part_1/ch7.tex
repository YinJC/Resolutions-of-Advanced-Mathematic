% !TEX root = ../HTNotes-Demo.tex
\begin{flushright}
  \color{zhanqing!80}
  \ding{43} 习题见\autopageref{cha:6}
\end{flushright}
\section{平面点集与多元函数}
\begin{flushright}
  \color{zhanqing!80}
  \ding{43} 教材见86页
\end{flushright}

  2.解:$f\left( \frac{x}{y}, \sqrt{xy} \right) = \dfrac{x^3 - 2xy^2\sqrt{xy} + 3xy^4}{y^3}=\dfrac{x^3}{y^3} - 2 \times \dfrac{x}{y} \times \sqrt{xy} + 3xy$, 则

  $$f(x,y) = x^3 - 2xy + 3y^2$$

  则有
  $$f\left( \frac{1}{x},\sqrt{xy} \right) = \dfrac{1}{x^3} - \dfrac{4}{xy} + \dfrac{12}{y^2}.$$

  4.解:  套定义式即可得出答案$f(x + y, x-y, xy) = (x + y)^{xy} +(xy)^{2x}$.

  6.解:  (1) $\begin{cases}
  4x - {y^2} \ge 0\\
  1 - {x^2} - {y^2} > 0
  \end{cases}$
  $\Rightarrow ~ \left\{ (x,y)\left|0< x^2 + y^2 <1, y^2 \leq 4x \right. \right\}$;

  (2)$\begin{cases}
  y \ge 0\\
  x - \sqrt y  \ge 0
  \end{cases}$
  $\Rightarrow ~ \left\{ (x,y)\left|x \geq 0, y \geq 0, x^2 \geq y \right. \right\}$;

  \smallskip
  (3) $\left( x^2 + y^2 - a^2 \right) \left( 2a^2 - x^2 - y^2 \right) \geq 0$
  $\Rightarrow ~ \left\{ (x,y)\left|a^2 \leq\ x^2 + y^2 \leq 2a^2 \right. \right\}$;

  (4) $\begin{cases}
  -1 \leq \dfrac{x}{y^2} \leq 1 \\
  y \geq 0 \\
  1 - \sqrt{y} > 0
  \end{cases}$
  $\Rightarrow ~ \left\{ (x,y)\left|-y^2 \leq x \leq y^2, 0<y<1 \right. \right\}$;

  (5) $ \left\{ {\begin{array}{*{20}{l}}
    {\sqrt {{x^2} + {y^2}}  > 0}\\
    { - 1 \le \dfrac{z}{{\sqrt {{x^2} + {y^2}} }} \le 1}
    \end{array}} \right.$
  $\Rightarrow ~ \left\{ (x,y)\left|x^2 + y^2 -z^2 \geq 0, x^2 + y^2 \neq 0 \right. \right\}$.

\section{多元函数的极限与连续性}
\begin{flushright}
  \color{zhanqing!80}
  \ding{43} 教材见92页
\end{flushright}

  1.解:(1)原式=$ \dfrac{{1 \times 2}}{{{1^2} + 1}} = 1.$

  (2)令$xy = u$,
  则原式=$\mathop {\lim }\limits_{u \to 0} \dfrac{{\sqrt {u + 1}  - 1}}{u} = \dfrac{1}{2}.$

  (3)令${x^2} + {y^2} = u$,
  则原式=$\mathop {\lim }\limits_{u \to 0} \dfrac{{1 - \cos u}}{{{u^2}}} = \mathop {\lim }\limits_{u \to 0} \dfrac{{\frac{{{u^2}}}{2}}}{{{u^2}}} = \dfrac{1}{2}.$

  (4)原式=$\mathop {\lim }\limits_{(x,y) \to (1,0)} {(1 + \dfrac{y}{x})^{\frac{x}{y} \times \frac{2}{x}}} = \mathop {\lim }\limits_{x \to 0} {e^{\frac{2}{x}}} = {e^2}$.

  2.解: (1)先令$y = x$,
  则$\mathop {\lim }\limits_{(x,y) \to (0,0)}  \dfrac{{xy}}{{x + y}} =  \mathop {\lim }\limits_{x \to 0} \dfrac{x}{2} = 0$

  再令$y = {x^2} - x$,
  则$\mathop {\lim }\limits_{(x,y) \to (0,0)} \dfrac{{xy}}{{x + y}} = \mathop {\lim }\limits_{x \to 0} \dfrac{{{x^2} - x}}{{{x^2} + x - x}} = 1$

  因为两条路径上所得的极限不相等,所以极限不存在.

  (2)令$y = kx,(k \in R)$
  则$ \mathop {\lim }\limits_{(x,y) \to (0,0)} \dfrac{{{x^2} - {y^2}}}{{{x^2} + {y^2}}} = \mathop {\lim }\limits_{x \to 0} \dfrac{{{x^2}(1 - {k^2})}}{{{x^2}(1 + {k^2})}} = \dfrac{{1 - {k^2}}}{{1 + {k^2}}}$

  当$k$值发生变化时, 极限值会发生相应的变化,所以极限不存在.

  4.解: $ \left\{ (x,y)\left|{x^2} + {y^2} = k\pi  + \dfrac{1}{2}\pi (k \in N) \right.\right\}$.

  5.解: 先令$ y = 0$,
  则$ \mathop {\lim }\limits_{(x,y) \to (0,0)} \dfrac{{2xy}}{{{x^2} + {y^2}}} \Leftrightarrow \mathop {\lim }\limits_{x \to 0} \dfrac{{2x \times 0}}{{{x^2} + 0}} = 0$

  再令$ y = x$,
  则$ \mathop {\lim }\limits_{(x,y) \to (0,0)} \dfrac{{2xy}}{{{x^2} + {y^2}}} \Leftrightarrow \dfrac{{2{x^2}}}{{{x^2} + {x^2}}} = 1$

  当$ x \ne 0 $时, 函数极限不存在,
  所以$f(x,y)$在$x=0$处不连续.

  7.解: (1)原式=$ \mathop {\lim }\limits_{x \to 0} \dfrac{{{e^0} - 1}}{1} = 0$.

  (3) 因为$ |\sin \dfrac{1}{{{x^2}}}\cos \dfrac{1}{{{y^2}}}| \le 1$,
  所以原式=$ \mathop {\lim }\limits_{(x,y) \to (0,0)} 0 \times \sin \dfrac{1}{{{x^2}}}\cos \dfrac{1}{{{y^2}}} = 0$.

  (5)原式=$ \mathop {\lim }\limits_{(x,y) \to (0,0)} {(1 + {x^2}{y^2})^{\frac{1}{{{x^2}{y^2}}} \times \frac{{{x^2}{y^2}}}{{{x^2} + {y^2}}}}} = {e^{\mathop {\lim }\limits_{(x,y) \to (0,0)} \frac{{{x^2}{y^2}}}{{{x^2} + {y^2}}}}} = {e^0} = 1$.

\section{全微分与偏导数}
\begin{flushright}
  \color{zhanqing!80}
  \ding{43} 教材见106页
\end{flushright}

  3.解: (1) $ \dfrac{{\partial z}}{{\partial x}} = \dfrac{{\frac{x}{{\sqrt {{x^2} + {y^2}} }} + 1}}{{x + \sqrt {{x^2} + {y^2}} }} = \dfrac{1}{{\sqrt {{x^2} + {y^2}} }}$; \qquad
  $ \dfrac{{\partial z}}{{\partial y}} = \dfrac{{\frac{y}{{\sqrt {{x^2} + {y^2}} }}}}{{x + \sqrt {{x^2} + {y^2}} }} = \dfrac{y}{{\sqrt {{x^2} + {y^2}} (x + \sqrt {{x^2} + {y^2}} }}. $

  (2) $ \dfrac{{\partial z}}{{\partial x}} = y \times xy{(1 + xy)^{y - 1}} = {y^2}{(1 + xy)^{y - 1}}$; 所以$\ln z = y\ln \left( {1 + xy} \right).$

  且 $ \dfrac{{\partial \ln z}}{{\partial z}} = z\dfrac{{\partial z}}{{\partial y}}, $
  所以 $ \dfrac{{\partial z}}{{\partial y}} = {(1 + xy)^y}\left[\ln (1 + xy) + \dfrac{{xy}}{{1 + xy}}\right]. $

  (3) $ \dfrac{{\partial z}}{{\partial x}}
  = \dfrac{{\cos \left(\frac{{x + a}}{{\sqrt y }}\right)}}{{\sin \frac{{x + a}}{{\sqrt y }}}} \times \dfrac{1}{{\sqrt y }}
  = \dfrac{{\cos \left(\frac{{x + a}}{{\sqrt y }}\right)}}{{\sqrt y }}$;

  $ \dfrac{{\partial z}}{{\partial y}}
  = \dfrac{{\cos \left(\frac{{x + a}}{{\sqrt y }}\right) \times \left( - \frac{{x + a}}{2} \right) \times y \times \frac{3}{2}}}{{\sin \left(\frac{{x + a}}{{\sqrt y }} \right)}}
  = \cos \dfrac{{x + a}}{{\sqrt y }} \times (\dfrac{{x + a}}{2}){y^{ - \frac{3}{2}}}$.

  (4) $ \dfrac{{\partial u}}{{\partial x}} = yz{x^{yz - 1}}; \qquad
  \dfrac{{\partial u}}{{\partial y}} = z{x^{yz}}\ln x; \qquad \dfrac{{\partial u}}{{\partial z}} = y{x^{yz}}\ln x. $

  (5) $ \mathop u\nolimits_x  = {x^2}, $
  所以 $ \mathop u\nolimits_{x(x,1)}  = \mathop {2x|}\nolimits_{x = 1}  = 2$.

  5.解: $ \dfrac{{\partial u}}{{\partial x}} = (y - z)\left[ - 2x + (y - z)\right]; $ \qquad
  $ \dfrac{{\partial u}}{{\partial y}} = (z - x)\left[ - 2y + (x - z)\right]; $ \qquad
  $ \dfrac{{\partial u}}{{\partial z}} = (x - y)\left[ - 2z + (x - y)\right]$.

  6.解: (1)$ \rd u = {y^2}{x^{{y^2} - 1}} \rd x + 2y{x^{{y^2}}}\ln x \rd y; $

  (2)$ \rd u = yz{x^{yz - 1}} \rd x + z{x^{yz}}\ln x \rd y - y{x^{yz}}\ln x \rd z; $

  (3)因为 $ \rd \ln u = \dfrac{{\rd u}}{u} = \dfrac{y}{{xz}} \rd x + \dfrac{1}{z}(\ln \dfrac{x}{y} - 1) \rd y - \dfrac{y}{{{z^2}}}\ln \dfrac{x}{y}\rd z, $

  所以 $ \rd u = {\left( \dfrac{x}{y} \right)^{\frac{y}{z}}} \left[\dfrac{y}{{xz}} \rd x + \dfrac{1}{z} \left(\ln \dfrac{x}{y} - 1 \right) \rd y - \dfrac{y}{{{z^2}}}\ln \left(\dfrac{x}{y} \right) \rd z \right]. $

  9.解: $ \dfrac{{{\partial ^2}z}}{{\partial {x^2}}} = \dfrac{\partial }{{\partial x}}\dfrac{{\partial z}}{{\partial x}} = \dfrac{\partial }{{\partial x}}(\dfrac{{2x}}{{{x^2} + y}}) = \dfrac{{ - 2{x^2} + 2y}}{{{{({x^2} + y)}^2}}}; $\quad
  $ \dfrac{{{\partial ^2}z}}{{\partial x\partial y}} = \dfrac{\partial }{{\partial y}}(\dfrac{{\partial z}}{{\partial x}}) = \dfrac{\partial }{{\partial y}}(\dfrac{{2x}}{{{x^2} + y}}) = \dfrac{{ - 2x}}{{{{(y + {x^2})}^2}}}; $

  $ \dfrac{{{\partial ^2}z}}{{\partial {y^2}}} = \dfrac{\partial }{{\partial y}}(\dfrac{{\partial z}}{{\partial y}}) = \dfrac{\partial }{{\partial y}}(\dfrac{1}{{{x^2} + y}}) =  - \dfrac{1}{{({x^2} + y)}}. $

  11.解: $ \dfrac{{{\partial ^2}u}}{{\partial {x^2}}} = \dfrac{\partial }{{\partial z}}(\dfrac{{yz}}{{{x^2} + {y^2}}}) =  - \dfrac{{2yzx}}{{{{({x^2} + {y^2})}^2}}}; $\qquad
  $ \dfrac{{{\partial ^2}u}}{{\partial {y^2}}} = \dfrac{\partial }{{\partial y}}(\dfrac{{ - zx}}{{{x^2} + {y^2}}}) = \dfrac{{2yzx}}{{({x^2} + {y^{2{)^2}}}}}; $\qquad
  $ \dfrac{{{\partial ^2}u}}{{\partial {z^2}}} = 0; $

  所以$ \dfrac{{{\partial ^2}u}}{{\partial {x^2}}} + \dfrac{{{\partial ^2}u}}{{\partial {y^2}}} + \dfrac{{{\partial ^2}u}}{{\partial {z^2}}} = 0. $

  13.解: 曲线的参数方程为
  $\begin{cases}
  x = 1 \\
  y = y \\
  z = \sqrt{2 + y^2}
  \end{cases}$,
  因为 $ \dfrac{{\rd x}}{{\rd y}} = 0,\dfrac{{\rd y}}{{\rd y}} = 1,\dfrac{{\rd z}}{{\rd y}} = \dfrac{y}{{\sqrt {2 + {y^2}} }} $

  所以在点$ (1,1,\sqrt 3 ) $ 处的切线方向向量为 $ \mathop m\limits^ \to   = (\dfrac{{\partial x}}{{\partial y}},\dfrac{{\partial y}}{{\partial y}},\dfrac{{\partial z}}{{\partial y}}) = (0,1,\dfrac{1}{{\sqrt 3 }})$

  所以$ \mathop m\limits^ \to   $
  与y轴正向的夹角余弦$ \cos \alpha  = \dfrac{{\sqrt 3 }}{2}, $即$ \alpha  = {30^ \circ }. $

\section{多元复合函数的微分法}
\begin{flushright}
  \color{zhanqing!80}
  \ding{43} 教材见118页
\end{flushright}

  1.解:  易知\(\begin{cases}
  \dfrac{{\partial z}}{{\partial x}} = 2u \times u' + 2v \times v' = 4x
  \dfrac{{\partial z}}{{\partial y}} = 2u \times u' + 2v \times v' = 4y
  \end{cases}.\)

  3.解:  因为 $ z = \arctan \left( {3t - 4{t^3}} \right), $
  所以 $ \dfrac{{\rd z}}{{\rd t}} = \dfrac{{{{\left( {3t - 4{t^3}} \right)}^\prime }}}{{1 + \left( {3t - 4{t^3}} \right)}} = \dfrac{{3(1 - 4{t^2})}}{{1 + {{\left( {3t - 4{t^3}} \right)}^2}}}. $

  5.解:  $ z = y{e^{{y^2}}} \times x{e^x} + \sin \dfrac{x}{{{y^2}}}$,
  所以 $ \dfrac{{\rd z}}{{\rd x}} = y{e^{{y^2}}} \times (x + 1){e^x} + \dfrac{1}{y}\cos \dfrac{x}{{{y^2}}} = x\left( {y + 1} \right){e^{x + {y^2}}} + \dfrac{1}{{{y^2}}}\cos \dfrac{x}{{{y^2}}}. $

  7.解:  $ {f_x} = {e^{ - {{(x + ay)}^2}}} - {e^{ - {x^2}}}, $
  所以
  $\begin{cases}
  {f_{xx}} =  - 2\left( {x + ay} \right){e^{ - {{\left( {x + ay} \right)}^2}}} + 2x{e^{ - {x^2}}} \\
  {f_{xx(1,1)}} = (2a - 2){e^{ - {{(1 - a)}^2}}} + \dfrac{2}{e}
  \end{cases}$.

  9.解:  $ \dfrac{{{\partial ^2}z}}{{\partial {x^2}}}
  = \dfrac{\partial }{{\partial x}}\left( {\dfrac{{\partial z}}{{\partial x}}} \right)
  = \dfrac{\partial }{{\partial x}}\left( {{f_1}^\prime  + y{f_2}^\prime } \right)
  = {f_{11}}^{\prime \prime } + 2y{f_{12}}^{\prime \prime } + {y^2}{f_{22}}^{\prime \prime }. $

  11.解:  $ \dfrac{{\partial z}}{{\partial x}} = f'\left( {x + \varphi \left( y \right)} \right), \qquad
  \dfrac{{{\partial ^2}z}}{{\partial {x^2}}} = f''\left( {x + \varphi \left( y \right)} \right), $

  同理可得$ \dfrac{{\partial z}}{{\partial y}} = \phi'\left( y \right)f'\left( {x + \phi \left( y \right)} \right), \qquad
  \dfrac{{{\partial ^2}z}}{{\partial x\partial y}} = \phi'(y)f''\left( {x + \phi \left( y \right)} \right). $

  所以 $ \dfrac{{\partial z}}{{\partial x}} \times \dfrac{{{\partial ^2}z}}{{\partial x\partial y}} = f'\left( {x + \phi \left( y \right)} \right)f''\left( {x + \phi \left( y \right)} \right)\phi'y = \dfrac{{\partial z}}{{\partial y}} \times \dfrac{{{\partial ^2}z}}{{\partial {x^2}}}$.

  \begin{flalign*} \indent
  \begin{split}
  \text{13.解:  } u & = \int {\rd u}
  = \int {\left[ {\varphi (x + ay)\rd x + a\varphi (x + ay)\rd y - \varphi (x + ay)\rd x + a\varphi (x - ay)\rd y} \right]} \\
  & = a\left[ {\varphi \left( {x + ay} \right) + \varphi \left( {x - ay} \right)} \right] + \varphi \left( {x + ay} \right) - \varphi \left( {x - ay} \right) + c
  \end{split}&
  \end{flalign*}

  故而易得
  $\begin{cases}
  {u_{xx}} = a\left[ {{\varphi ^{''}}\left( {x + ay} \right) + {\varphi ^{''}}\left( {x - ay} \right)} \right] + {\varphi ^{''}}\left( {x + ay} \right) - {\varphi ^{''}}\left( {x - ay} \right)] \\
  \left[ {{\varphi ^{''}}\left( {x + ay} \right) + {\varphi ^{''}}\left( {x - ay} \right)} \right] + {\varphi ^{''}}\left( {x + ay} \right) - {\varphi ^{''}}\left( {x - ay} \right)
  \end{cases}$,
  所以 $ {u_{yy}} = {a^2}{u_{xx}}. $

  15.解: 令$ x = \rho \cos \theta ,y = \rho \sin \theta  $得
  $ \dfrac{{\rd \rho \sin \theta }}{{\rd t}} =  - \rho \cos \theta  + k{\rho ^3}\sin \theta . $

  进而得
  $\begin{cases}
  \dfrac{{\rd \theta }}{{\rd t}}\rho \cos \theta  =  - \rho \cos \theta
  \dfrac{{\rd \rho }}{{\rd t}}\sin \theta  = k{\rho ^3}\sin \theta
  \end{cases}$,
  所以
  $\begin{cases}
  \dfrac{{\rd \theta }}{{\rd t}} =  - 1
  \dfrac{{\rd \rho }}{{\rd t}} = k{\rho ^3}.
  \end{cases}$

  16.证明:$ \rd F\left( {tx + ty + tz} \right) = \rd {t^k}F\left( {x,y,z} \right) $(对t求微分)

  $  \Leftrightarrow \left[{xF_1' + yF_2' + zF_3'} \right] \rd t = k{t^{k - 1}}F(x,y,z)\rd t $
  对上式两侧同时对t积分,得
  \[x{F_x} + y{F_y} + z{F_z} = k{t^{k - 1}}F\left( {x,y,z} \right)\]
  当$ t = 1 $时,符合题意.证毕.

\section{隐函数的微分法}
\begin{flushright}
  \color{zhanqing!80}
  \ding{43} 教材见129页
\end{flushright}

  1.解: $ \dfrac{{\rd y}}{{\rd x}} =  - \dfrac{{{F_x}}}{{{F_y}}} = \dfrac{{{y^2} - {e^x}}}{{\cos y - 2y}}. $

  3.解: $ \dfrac{{\partial z}}{{\partial x}} = \dfrac{z}{{x + z}}, \qquad
  \dfrac{{\partial z}}{{\partial y}} = \dfrac{{{z^2}}}{{y\left( {z + x} \right)}}.$

  5.解: 因为$ x = x\left( {y,z} \right) $,
  所以$ F\left[ {x\left( {y,z} \right),y,z} \right] = 0$,
  两侧微分,得$ F_1' \times \dfrac{{\partial x}}{{\partial y}} + F_2' = 0$;
  从而$ \dfrac{{\partial x}}{{\partial y}} =  - \dfrac{F_2'}{F_1'}$

  同理$\dfrac{{\partial y}}{{\partial z}} =  - \dfrac{F_3'}{F_2'},\dfrac{{\partial z}}{{\partial x}} =  - \dfrac{F_1'}{F_3'}$,
  \( \therefore~\dfrac{{\partial x}}{{\partial y}} \cdot \dfrac{{\partial y}}{{\partial z}} \cdot \dfrac{{\partial z}}{{\partial x}} =  - 1.\)

  \begin{flalign*} \indent
  \begin{split}
  \text{9.解: } \dfrac{{\partial u}}{{\partial x}}
  & = \dfrac{{\partial u}}{{\partial z}} \cdot \dfrac{{\partial z}}{{\partial x}}
  = \left( {{f'_{1}}\dfrac{{\partial x}}{{\partial z}} + {f'_{3}}} \right) \cdot \dfrac{{\partial z}}{{\partial x}}
  = {f_{1'}} + {f_{3'}} \cdot \dfrac{{\partial z}}{{\partial x}} = {f_{1'}} + {f_{3'}}\left( { - \dfrac{{5y}}{{5{z^4} + 5}}} \right)
  = {f_{1'}} - {f_{3'}} \times \dfrac{y}{{{z^4} + 1}}.
  \end{split}&
  \end{flalign*}

  11.解: 令$ u = x + \dfrac{z}{y},v = y + \dfrac{z}{x}$,
  所以 $  F\left( {u,v} \right) = 0$

  易知
  $\begin{cases}
  {F_x} = \dfrac{{\partial F}}{{\partial x}} \cdot \dfrac{{\partial F}}{{\partial u}} + \dfrac{{\partial F}}{{\partial v}} \cdot \dfrac{{\partial v}}{{\partial x}} = \dfrac{{\partial F}}{{\partial u}} - \dfrac{z}{{{x^2}}} \cdot \dfrac{{\partial F}}{{\partial v}}.
  {F_y} =  - \dfrac{z}{{{y^2}}}\dfrac{{\partial F}}{{\partial u}} + \dfrac{{\partial F}}{{\partial v}},{F_z} = \dfrac{1}{y}\dfrac{{\partial F}}{{\partial u}} + \dfrac{1}{x}\dfrac{{\partial F}}{{\partial v}}
  \end{cases}$,
  故令$ s = \dfrac{{\partial F}}{{\partial u}},t = \dfrac{{\partial F}}{{\partial v}}$,

  则$ {F_x} = s - \dfrac{z}{{{x^2}}}t$,
  $F{_y} =  - \dfrac{z}{{{y^2}}}s + t$,
  ${F_z} = \dfrac{s}{y} + \dfrac{t}{x}$,
  又$\dfrac{{\partial z}}{{\partial x}}
  =  - \dfrac{{{F_x}}}{{{F_z}}}
  =  - \dfrac{{s - \frac{z}{{{x^2}}}t}}{{\dfrac{s}{y} + \dfrac{t}{x}}}
  = \dfrac{{ - xys + \frac{{zy}}{x}t}}{{xs + yt}}$

  故而
  $ x\dfrac{{\partial z}}{{\partial x}} + y\dfrac{{\partial z}}{{\partial y}}
  = x\dfrac{{ - xsy + \frac{{zy}}{x}t}}{{xs + ty}} + y\dfrac{{\frac{{xz}}{y}t - xyt}}{{xs + ty}}
  = \dfrac{{\left( {xs + yt} \right)\left( {z - xy} \right)}}{{xs + yt}}
  = z - xy$.

  13.解: $ \dfrac{{{\partial ^2}z}}{{\partial x\partial y}} = {\kern 1pt} \dfrac{\partial }{{\partial y}}\left( { - \dfrac{{Fx}}{{{F_z}}}} \right) =  - \dfrac{{{F_{xy}}{F_{{z}}}^2 - {F_{xz}}{F_y}{F_z} - {F_{yz}}{F_x}{F_z} + y{F_{zz}}{F_x}{F_{}}}}{{{F_z}^3}}$

  易知$ {F_x} =  - 3xy$,
  ${F_y} =  - 3xz$,
  ${F_z} = 3{z^2} - 3xy$

  $\Rightarrow$
  ${F_{xx}} = {F_{yy}} = 0$,
  ${F_{zz}} = 6z$,
  ${F_{xy}} =  - 3z$,
  ${F_{xz}} =  - 3y$,
  ${F_{yz}} =  - 3x. $
  $\Rightarrow~\dfrac{{\partial z}}{{\partial x\partial y}} = \dfrac{{{z^5} - 2xy{z^3} - {x^2}{y^2}z}}{{{{({z^2} - xy)}^3}}}. $

  15.解: (1)$ \dfrac{{\rd y}}{{\rd x}} =  - \dfrac{{{F_x}}}{{{F_y}}} =  - \dfrac{{x(6z + 1)}}{{2y(3z + 1)}}$;
  因为$\rd z = 2x\rd x + 2y\rd y$,
  所以$\dfrac{{\rd y}}{{\rd x}} = 2x + 2y\dfrac{{\rd y}}{{\rd x}} = \dfrac{x}{{3z + 1}}.$

  (3)由$\begin{cases}
  x = {e^u}\cos v & \ding{192}\\
  y = {e^u}\sin v & \ding{193}
  \end{cases}$
  得${x^2} + {y^2} = {e^{2u}}$,
  所以$ u = \dfrac{1}{2}\ln \left( {{x^2} + {y^2}} \right)$,
  由$\dfrac{{{\text{\ding{192}}^2}}}{{{\text{\ding{193}}^2}}},$得$\tan v = \dfrac{y}{x}$

  $v = \arctan \dfrac{y}{x}$,
  $\dfrac{{\partial z}}{{\partial x}}
  = \dfrac{{\partial z}}{{\partial u}} \cdot \dfrac{{\partial u}}{{\partial x}} + \dfrac{{\partial z}}{{\partial v}} \cdot \dfrac{{\partial v}}{{\partial x}}
  = v\dfrac{{\partial u}}{{\partial x}} + u\dfrac{{\partial v}}{{\partial x}}
  = \dfrac{{2x\arctan \frac{y}{x} - y\ln \left( {{x^2} + {y^2}} \right)}}{{2\left( {{x^2} + {y^2}} \right)}}$

  $\dfrac{{\partial z}}{{\partial y}} = \dfrac{{y\arctan \dfrac{y}{x} - x\ln \left( {{x^2} + {y^2}} \right)}}{{2\left( {{x^2} + {y^2}} \right)}}$,
  将\ding{192}\ding{193}式代回,得
  $\dfrac{{\partial z}}{{\partial x}} = \dfrac{{v\cos v - u\sin v}}{{{e^u}}}, \quad
  \dfrac{{\partial z}}{{\partial y}} = \dfrac{{u\cos v + v\sin v}}{{{e^u}}}$.

\section{方向导数与梯度}
\begin{flushright}
  \color{zhanqing!80}
  \ding{43} 教材见137页
\end{flushright}

  2.解: 计算得$u_x'\left( A \right) =  - 2$, $u_y'\left( A \right) = 4$, $u_z'\left( A \right) =  - 2$,
  因此$ \grad~u\left( A \right) = \left( { - 2,4, - 2} \right)$

  又$ \overrightarrow{AB} = \left( {1,2, - 2} \right)$,
  因此其方向余弦向量为$ \left( {\dfrac{1}{3},\dfrac{2}{3}, - \dfrac{2}{3}} \right)$,
  $\left.\dfrac{{\partial u}}{{\partial \overrightarrow{AB} }}\right|_A
  = \left( { - 2,4, - 2} \right) \cdot \left( {\dfrac{1}{3},\dfrac{2}{3}, - \dfrac{2}{3}} \right) = \dfrac{{10}}{3}$.

  4.解: (1)$\grad~z\left(2,1\right) = \left.\dfrac{{\partial z}}{{\partial x}}\right|_{\left( {2,1} \right)} \vec{i}   + \left. \dfrac{{\partial z}}{{\partial y}}\right|_{\left( {2,1} \right)} \vec{j}  = \left( {16,18} \right);$

  (3)$ \grad~z\left(\dfrac{\pi }{3},\dfrac{\pi }{3}\right)
  = \left. \dfrac{{\partial z}}{{\partial x}}\right|_{\left( {\frac{\pi }{3},\frac{\pi }{3}} \right)}\vec{i}
  = \left.\dfrac{{\partial z}}{{\partial y}}\right|_{\left( {\frac{\pi }{3},\frac{\pi }{3}} \right)}\vec{j}
  = \left( { - \dfrac{{9\sqrt 3 }}{2}, - \dfrac{{3\sqrt 3 }}{2}} \right)$.

  (5)$ \grad~u\left( {1,1,0} \right)
  = \left.\dfrac{{\partial u}}{{\partial x}}\right|_{\left( {1,1,0} \right)}\vec i + \left.\dfrac{{\partial u}}{{\partial y}}\right|_{\left( {1,1,0} \right)}\vec j + \dfrac{{\partial u}}{{\partial z}}{|_{\left( {1,1,0} \right)}}\vec k = (1,1,1)$.

  6.解: 将$ {y^2} = 4x $两侧对x求微分,得:
  $ 2y\dfrac{{\rd y}}{{\rd x}} = 4 \Rightarrow \dfrac{{\rd y}}{{\rd x}} = \dfrac{2}{y}$,
  所以$  \tan \theta  = \dfrac{{\rd y}}{{\rd x}}{|_{\left( {1,2} \right)}} = 1$

  $\Rightarrow \overrightarrow{e_1}  = \left( {\dfrac{{\sqrt 2 }}{2},\dfrac{{\sqrt 2 }}{2}} \right)$
  所以$\left. \dfrac{{\partial z}}{{\partial x}}\right|_{\left( {1,2} \right)}
  = \left. \dfrac{1}{3},\dfrac{{\partial z}}{{\partial y}}\right|_{\left( {1,2} \right)} = \dfrac{1}{3}$,
  所以$ z\left( {1,2} \right) = \left( {\dfrac{{\partial z}}{{\partial x}},\dfrac{{\partial z}}{{\partial y}}} \right) \cdot \mathop {{e_1}}\limits^ \to   = \dfrac{{\sqrt 2 }}{3}$.

  8.解: 易知方向向量$ \mathop {{e_1}}\limits^ \to   = \left( {\cos \alpha ,\sin \alpha } \right)$

  $ \cos \alpha  = 0 $时, $\left. \dfrac{{\partial f}}{{\partial l}} \right|_{\left( {0,0} \right)}
  = \mathop {\lim }\limits_{\rho  \to 0} \dfrac{{f\left( {\rho \cos \theta ,\rho \sin \theta } \right) - f\left( {0,0} \right)}}{\rho }
  = 0 = \cos \alpha \sin \alpha \big|_{\alpha  = \frac{\pi }{2}}$

  $ \cos \alpha  \ne 0 $时, $\left. \dfrac{{\partial f}}{{\partial l}}\right|_{\left( {0,0} \right)} = \mathop {\lim }\limits_{\rho  \to 0} \dfrac{{f\left( {\rho \cos \theta ,\rho \sin \theta } \right) - f\left( {0,0} \right)}}{\rho } = \cos \alpha \sin \alpha $

  所以$ \dfrac{{\partial f}}{{\partial l}} = \cos \alpha \sin \alpha$.

\section{微分法在几何上的应用}
\begin{flushright}
  \color{zhanqing!80}
  \ding{43} 教材见146页
\end{flushright}

  1.解: $\dfrac{{\rd x}}{{\rd t}} =  - \sin t$, $\dfrac{{\rd y}}{{\rd t}} = \cos t$, $\dfrac{{\rd z}}{{\rd t}} = \dfrac{ {{\sec \left( \frac{t}{2} \right)^2}  }}{2}$, 易知$t = \dfrac{\pi }{2}$

  所以切线方程为$\begin{cases}x+z-1=0\\y=1 \end{cases}$,
  法平面方程为$-x+z-1=0$

  3.解: 平面的一个法向量为$\overrightarrow m  = (1,2,1)$, 且$\dfrac{{\rd x}}{{\rd t}} = 1$, $\dfrac{{\rd y}}{{\rd t}} = 2t$, $\dfrac{{\rd z}}{{\rd t}} = 3t^2$

  所以切线的法向量为$\overrightarrow n  = (1,2t,3t^2)$.

  因为$ \overrightarrow n \parallel $平面,
  所以$\overrightarrow m  \cdot \overrightarrow n  = 0\Rightarrow3t^2+4t+1=0$

  解得$t=-1$或$t=-\dfrac{1}{3}$,
  所以点为$P_1(-1,1,-1)$或$P_2(-\dfrac{1}{3},\dfrac{1}{9},-\dfrac{1}{27})$.

  5.易知$z = \dfrac{2}{{xy}}        \left. {\dfrac{{\partial z}}{{\partial x}}}  \right|_{(2,1,1)}=-\dfrac{1}{2}           \left. {\dfrac{{\partial z}}{{\partial y}}} \right|_{(2,1,1)}=-1$

  所以切平面方程为$z - 1 + \dfrac{1}{2}(x - 2) + y - 1 = 0 \Rightarrow x + 2y + 2z = 6$

  所以该切平面的法向量为$\overrightarrow n_1  = (1,2,2)$,
  面$x-y-z=0$的一个法向量为$\overrightarrow n_2  = (1,-1,-1)$

  所以切向量$\overrightarrow n  = \overrightarrow n_1  \times \overrightarrow n_2\Rightarrow \overrightarrow n=(0,3,-3)$或$(0,-3,3)$,
  易知$\overrightarrow n=(0,-3,3)$符合题意

  设$y$轴正向的一个方向向量$\overrightarrow p=(0,1,0)$

  所以$ \cos \left\langle {\vec p,\vec n} \right\rangle  = \dfrac{{\vec p \cdot \vec n}}{{|\vec p||\vec n|}} =  - \dfrac{{\sqrt 2 }}{2}$,
  所以切向量与$y$轴正向夹角为$\dfrac{3}{4}\pi $,
  所以$ F_x=0,F_y=1,F_z=1$

  所以在点$(2,1,1)$处的法线方程为$\dfrac{{x - 2}}{0} = \dfrac{{y - 1}}{1} = \dfrac{{z - 1}}{1}$,即$\begin{cases}y-z=0\\x=2\end{cases}$.

  9.解: 且平面方程为$F_x(x-x_0)+F_y(y-y_0)+F_z(z-z_0)=0 \Rightarrow3x+2y-3z-4=0$

  所以平面法向量$\overrightarrow m=(3,2,-3)$,
  易知平面$xOy$的法向量$\overrightarrow n=(0,0,1)$

  所以$ cos<\overrightarrow{m},\overrightarrow{n}>=\dfrac{{\overrightarrow{m}  \cdot \overrightarrow{n} }}{{\left| {\overrightarrow{m} } \right|\left| {\overrightarrow{n} } \right|}} = -\dfrac{3}{{\sqrt {22} }}$,
  易知切平面上有点$H(1,2,1)$, 面$xOy$上有点$(0,1,0)$

  所以$\overrightarrow {OH}  = \left( {1,1,1} \right)$,
  $\left\{ {\begin{array}{*{20}{l}}
    {\overrightarrow {OH}  \cdot \vec m = 2 > 0}\\
    {\overrightarrow {OH}  \cdot \vec n = 1 > 0}
    \end{array}} \right.$,
  所以夹角余弦值$cos\alpha  =  - \cos  < \overrightarrow m ,\overrightarrow n  >  = \dfrac{{3\sqrt {22} }}{{22}}$.

  11.解: $x=\dfrac{{\cos te^t}}{2} + \dfrac{{\sin te^t}}{2} - \dfrac{1}{2}$,
  $\dfrac{{\rd x}}{{\rd t}} = \cos te^t$, $\dfrac{{\rd y}}{{\rd t}} = 2\cos t-sint$, $\dfrac{{\rd z}}{{\rd t}} = 3e^{3t}$

  所以切线为$\dfrac{x}{1} = \dfrac{{y - 1}}{2} = \dfrac{{z - 2}}{3}$,
  法平面方程为$x+2(y-1)+3(z-2)=0 \Rightarrow x +2y+3z-8=0$.

\section{多元函数的极值}
\begin{flushright}
  \color{zhanqing!80}
  \ding{43} 教材见159页
\end{flushright}

  1.解: (1)$\dfrac{{\partial z}}{{\partial x}} = 2x + y - 2$, $\dfrac{{\partial z}}{{\partial y}} = x + 2y - 1$,
  $\begin{cases}\dfrac{{\partial z}}{{\partial x}} =0\\\dfrac{{\partial z}}{{\partial y}} =0\end{cases}\Rightarrow \begin{cases}x=1\\y=0\end{cases}$,则驻点$P(1,0)$

  $H_f(P)=\left( {\begin{array}{*{20}{c}}
    {\dfrac{{\partial^2 z}}{{\partial x^2}}}&{\dfrac{{\partial^2 z}}{{\partial x\partial y}}}\\
    {\dfrac{{\partial^2 z}}{{\partial x\partial y}}}&{\dfrac{{\partial^2 z}}{{\partial y^2}}}_{(1,0)}
    \end{array}} \right) = \left( {\begin{array}{*{20}{c}}
    2&1\\
    1&2
    \end{array}} \right) $,
  则$A>0,AC-B^2>0$

  所以$H_f(P)$是正定矩阵, $f(x,y)$在$(1,0)$处有最小值.

  (2)$\dfrac{{\partial z}}{{\partial x}} = 18x^2y^2 - 4x^3y^2 - 3x^2y^2$,
  $\dfrac{{\partial z}}{{\partial y}} = 12x^3y - 2x^4y - 3x^3y^2$

  易知当$x=0$或$y=0$时, $z$无法取得极值

  所以$\begin{cases}\dfrac{{\partial z}}{{\partial x}}=0\\\dfrac{{\partial z}}{{\partial y}}=0\end{cases} \Rightarrow \begin{cases}18-4x-3y=0\\12-2x-3y=0\end{cases} \Rightarrow\begin{cases}x=3\\y=2\end{cases}$,
  所以一个可能驻点为$(3,2)$

  $H_f = \left( {\begin{array}{*{20}{c}}
    { - 4}&{ - 3}\\
    { - 3}&{ - 3}
    \end{array}} \right)$,
  则$A<0,AC-B^2>0$

  所以$H_f(P)$是负定矩阵,在$(3,2)$处有极大值$108$.

  (3)令$U(x,y)=x^2+y^2$,易知$\delta(x,y)$与$U(x,y)$的增减性恰好相反

  对于$U(x,y)$, $\dfrac{{\partial U}}{{\partial x}} = 2x,\dfrac{{\partial U}}{{\partial y}} = 2y$,
  所以$\begin{cases}\dfrac{{\partial U}}{{\partial x}} =0\\\dfrac{{\partial U}}{{\partial y}} =0\end{cases}\Rightarrow$驻点$(0,0)$

  由$U(x,y)=x^2+y^2$的几何意义知

  $U(x,y)=x^2+y^2$是开口向上的一个旋转抛物面,顶点为$(0,0,0)$

  所以极小值点为$(0,0)$,$z=f(x,y)$在$(0,0)$处取得最大值$1$.

  (4)有基本不等式且$x>0,y>0$得,$z \ge 3\sqrt[3]{{\dfrac{8}{x} \times \dfrac{x}{y} \times y}} = 6$

  当且仅当$\dfrac{8}{x} = \dfrac{x}{y} = y$时,即$\begin{cases}y^2=x\\\delta y=x^2\end{cases}$时取等,此时极小值点$P(4,2)$,极小值为$6$.

  3.解: $\dfrac{{\partial z}}{{\partial x}} =  - \dfrac{F'_x}{F_z} =  - \dfrac{{2x - 2}}{{2z - 4}},\dfrac{{\partial z}}{{\partial y}} =  - \dfrac{F_y}{F_z} =  - \dfrac{{2y + 2}}{{2z - 4}}$,
  $\begin{cases}\dfrac{{\partial z}}{{\partial x}} = 0\\\dfrac{{\partial z}}{{\partial y}} = 0 \end{cases} \Rightarrow$有驻点$(1,-1)$

    将$x=1,y=1$代入$F(x,y,z)=0$得$z^2-4z-12=0  z=2$或$z=6$

    所以易知$z$的极大值为6,极小值为-2.

    5.解: 令$z=f(x,y), \dfrac{{\partial z}}{{\partial x}} = 4-2x \dfrac{{\partial z}}{{\partial y}} =-4-2y$,
    所以$\begin{cases} \dfrac{{\partial z}}{{\partial x}} = 0\\\dfrac{{\partial z}}{{\partial y}} = 0 \end{cases} \Rightarrow$驻点$(2,-2)$,

    $H_f(2,-2)=\left( {\begin{array}{*{20}{c}}
      {\dfrac{{\partial^2 z}}{{\partial x^2}}}&{\dfrac{{\partial^2 z}}{{\partial x\partial y}}}\\
      {\dfrac{{\partial^2 z}}{{\partial x\partial y}}}&{\dfrac{{\partial^2 z}}{{\partial y^2}}}
      \end{array}} \right)=\left( {\begin{array}{*{20}{c}}
      { - 2}&0\\
      0&{ - 2}
      \end{array}} \right)$,
    则$A<,AC-B^2>0$

    所以$ H_f$是负定矩阵,$f(x,y)$在$(2,-2)$处有极大值16.

    7.解: $\begin{cases}2p=2x+2y\\z=x^2\pi xy\end{cases}$,
    $p=x+y=\dfrac{x}{2} + \dfrac{x}{2} + y \ge 3\sqrt[3]{{\dfrac{x^2}{4}y}}$,当且仅当$\dfrac{x}{2} = \dfrac{x}{2} = y$时取等,

    所以$\dfrac{x^2}{4}y \le \dfrac{p^3}{{27}}$,
    所以$ z \le \dfrac{{\pi p^3}}{{27}}$,
    所以长与宽分别是$\dfrac{2}{3}p$与$\dfrac{1}{3}p$.

    9.解:椭球的参数方程为$\begin{cases}x=sin\varphi cos\theta \\y=sin\varphi sin\theta ,&0<\theta<\dfrac{\pi }{2}\\z=2cos\varphi ,&0<\varphi <\dfrac{\pi }{2}\end{cases}$

    易知切平面方程为$2x_0(x-x_0)+2y_0(y-y_0)+\dfrac{z_0}{2}(z-z_0)=0$

    令$y,z=0$,得$zx_0x=\dfrac{z^2_0}{2}+2{y_0}^2+2{x_0}^2=2$;
    令$x_0=\dfrac{1}{x_0}$,同理得$y=\dfrac{1}{y_0},z=\dfrac{4}{z_0}$

    所以$ f(x,y,z)=\dfrac{1}{x^2}+\dfrac{1}{y^2}+\dfrac{16}{z^2}$

    $L=\dfrac{1}{x^2}+\dfrac{1}{y^2}+\dfrac{16}{z^2}+a(x^2+y^2+\dfrac{{z_0}^2}{4}-1),x_0>0,y_0>0,z_0>0$

    $\begin{cases}L_x=-\dfrac{2}{{x_0}^3}+2ax_0=0\\L_y=-\dfrac{2}{{y_0}^3}+2ay_0=0\\L_z=-\dfrac{32}{{z_0}^3}+\dfrac{az_0}{2}=0\\{x_0}^2+{y_0}^2-\dfrac{{z_0}^2}{4}=1\end{cases}\Rightarrow \begin{cases}x_0=\dfrac{1}{2}\\y_0=\dfrac{1}{2}\\z_0=\sqrt 2 \end{cases}$,
    所以点为$(\dfrac{1}{2},\dfrac{1}{2},\sqrt 2)$.

    13.解:易知切平面方程为$\dfrac{2x_0}{a^2}(x-x_0)+\dfrac{2y_0}{b^2}(y-y_0)+\dfrac{2z_0}{c^2}(z-z_0)=0$

    令$y,z=0$得$x=\dfrac{a^2}{x_0}$,同理得$y=\dfrac{b^2}{y_0},z=\dfrac{c^2}{z_0}$,
    所以$V=\dfrac{a^2b^2c^2}{6x_0y_0z_0}$,
    $1=\dfrac{x_0}{a^2}+\dfrac{y_0}{b^2}+\dfrac{z_0}{c^2}\geq3\sqrt[3]{{\dfrac{{x_0y_0z_0}}{{abc}}}}$

    所以$ \sqrt[3]{{\dfrac{{x_0y_0z_0}}{{abc}}}}\leq \dfrac{1}{3}$,当且仅当$\dfrac{x_0}{a} = \dfrac{y_0}{b} = \dfrac{z_0}{c}$时取等,且$\dfrac{x_0}{a^2}+\dfrac{y_0}{b^2}+\dfrac{z_0}{c^2}=1$

    所以$\begin{cases}x_0=\dfrac{a}{{\sqrt 3 }}\\y_0=\dfrac{b}{{\sqrt 3 }}\\z_0=\dfrac{c}{{\sqrt 3 }} \end{cases}$, 即点
    $\left(\dfrac{a}{\sqrt 3 },\dfrac{b}{\sqrt 3 },\dfrac{c}{\sqrt 3 }\right)$.

\section*{总习题七}
\addcontentsline{toc}{section}{总习题七}
\begin{flushright}
  \color{zhanqing!80}
  \ding{43} 教材见176页
\end{flushright}

  2.证明:易知函数等价于$ {z^2} = {x^2} + {y^2}, $在空间中为顶点在$ \left( {0,0,0} \right) $处的椭圆锥面.

  故函数在$ \left( {0,0} \right) $的各个方向均连续,
  又有$\dfrac{{\partial z}}{{\partial x}} = \dfrac{x}{{\sqrt {{x^2} + {y^2}} }},\dfrac{{\partial z}}{{\partial y}} = \dfrac{y}{{\sqrt {{x^2} + {y^2}} }}$

  二者在$ \left( {0,0} \right) $处均无意义.所以两个一阶偏导数均不存在.

  4.解:
  $\dfrac{{\partial u}}{{\partial x}} = {y^z}{x^{{y^{z - 1}}}}$; \qquad
  $\dfrac{{\partial u}}{{\partial y}} = z{x^{{y^z}}}{y^{z - 1}}\ln x$; \qquad
  $\dfrac{{\partial u}}{{\partial z}} = {y^z}{x^{{y^z}}}\ln x \cdot \ln y$.

  6.解: 对组中两个式子分别对x,z求导,得:
  \[\begin{cases}
  1 =  - 2u{u_x} + {v_x}\left( 1 \right)\\
  0 = {u_x} + z{v_x}\left( 2 \right) \\
  0 =  - 2uu + {v_z} + 1\left( 3 \right)\\
  0 = {u_z} + v + z{v_z}\left( 4 \right)
  \end{cases}\]

  由$ \left( 1 \right)\left( 2 \right) $解得:
  \[\dfrac{{\partial u}}{{\partial x}} =  - \dfrac{z}{{2uz + 1}}, \quad
  \dfrac{{\partial v}}{{\partial x}} = \dfrac{1}{{2uz + 1}}\]

  由$ \left( 3 \right)\left( 4 \right) $解得:
  \[\dfrac{{\partial u}}{{\partial z}} = \dfrac{{z - v}}{{2uz + 1}}.\]

  8.解: \[\dfrac{{\partial z}}{{\partial x}} =  - \dfrac{{{F_x}}}{{{F_z}}} =  \dfrac{{4x + 2y - 2}}{{4 - 2z}}; \quad
  \dfrac{{\partial z}}{{\partial y}} =  - \dfrac{{{F_y}}}{{{F_z}}} = \dfrac{{2y + 2x - 2}}{{4 - 2z}}.\]

  令两偏导均等于零,得$x  = 0,y = 1. $代回,得${z^2} - 4z + 3 = 0$解得$ z = 1 $或$ z=3$

  对方程左右求二阶偏导,得:
  \[\begin{cases}
  {z_{xx}} = \dfrac{{2{z_x}^2 + 4}}{{4 - 2z}};\\
  {z_{yy}} = \dfrac{{2{z_y}^2 + 2}}{{4 - 2z}};\\
  {z_{xy}} = \dfrac{{2 + z{}_x{z_y}}}{{4 - 2z}}.
  \end{cases}\]

  $ H_f\left( {x,y,z} \right) = \left( {\begin{array}{*{20}{c}}
    {{z_{xx}}}&{{z_{xy}}}\\
    {{z_{xy}}}&{{z_{yy}}}
    \end{array}} \right)$,
  所以$ H_f\left( {0,1,1} \right) = \left( {\begin{array}{*{20}{c}}
    2&1\\
    1&1
    \end{array}} \right) = 1 > 0,{z_{xx}} > 0 $

  所以$z=1$为$ z\left( {x,y} \right) $的极小值;

  因为$  Hf\left( {0,1,3} \right) = \left( {\begin{array}{*{20}{c}}
    { - 2}&{ - 1}\\
    { - 1}&{ - 1}
    \end{array}} \right) = 1 > 0,{z_{xx}} < 0 $,
  所以z=3为$z\left( {x,y} \right) $的极大值.

  10.解: $f\left( {x,y,z} \right) = \ln \left( {xy{z^3}} \right)$

  所以$ 5{r^2} = {x^2} + {y^2} + \dfrac{{{z^2}}}{3} + \dfrac{{{z^2}}}{3} + \dfrac{{{z^2}}}{3} \ge 5\sqrt[5]{{\dfrac{{{{(xy{z^3})}^2}}}{{27}}}}$,
  所以$xy{z^3} \le 3\sqrt 3 {r^5}$

  所以$ f\left( {x,y,z} \right) \le \ln \left( {3\sqrt 3 {r^5}} \right) = \ln \left[ {3\sqrt 3 {{\left( {\dfrac{{{x^2} + {y^2} + {z^2}}}{5}} \right)}^{\frac{5}{2}}}} \right]$
  $ \Rightarrow {x^2}{y^2}{z^6} \le 27{\left( {\dfrac{{{x^2} + {y^2} + {z^2}}}{5}} \right)^5}$

  当$ {x^2} = a,{y^2} = b,{z^2} = c$,
  有\[ab{c^3} \le 27{\left( {\dfrac{{a + b + c}}{5}} \right)^5}.\]

  12.解: $ V = \dfrac{1}{3}a\pi {h^2} = \dfrac{{4\pi p\left( {p - a} \right)\left( {p - b} \right)\left( {p - c} \right)}}{{3a}}. $

  两侧取对数,得\[\ln V = \ln \left( {p - a} \right) + \ln \left( {p - b} \right) + \ln \left( {p - c} \right) - \ln a + \ln \dfrac{{4\pi p}}{3}.\]

  令$ f\left( {a,b,c} \right) = \ln \left( {p - a} \right) + \ln \left( {p - b} \right) + \ln \left( {p - c} \right) - \ln a. $

  记\(L = \ln \left( {p - a} \right) + \ln \left( {p - b} \right) + \left( {p - c} \right) - \ln a + \lambda \left( {a + b + c - 2p} \right) \)

  $ \therefore \begin{cases}
  {L_a} =  - \dfrac{1}{{p - a}} + \lambda  - \dfrac{1}{a} = 0\\
  {L_b} =  - \dfrac{1}{{p - b}} + \lambda \\
  {L_c} =  - \dfrac{1}{{p - c}} + \lambda \\
  a + b + c - 2p = 0
  \end{cases} $
  \( \Rightarrow \begin{cases}
  a = \dfrac{p}{2}\\
  b = c = \dfrac{{3p}}{4}
  \end{cases} \),
  则\( {V_{\max }} = V\left( {\dfrac{p}{2},\dfrac{{3p}}{4},\dfrac{{3p}}{4}} \right) = \dfrac{\pi }{{12}}{p^3}\).