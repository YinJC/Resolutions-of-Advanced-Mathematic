% !TEX root = ../HTNotes-Demo.tex
\begin{flushright}
  \color{zhanqing!80}
  \ding{43} 习题见\autopageref{cha:9}
\end{flushright}
\section{第一型曲线积分——对弧长的曲线积分}
\begin{flushright}
  \color{zhanqing!80}
  \ding{43} 教材见265页
\end{flushright}

  1.
  (1) 解:$I{\rm{ = }}\displaystyle\int_C {\left( {{x^2} + {y^2}} \right) \rd s = \left\{ {\displaystyle\int {_{\overline {OA} }}  + \displaystyle\int {_{\overline {AB} }}  + \displaystyle\int {_{\overline {BO} }} } \right\}} \left( {x + y} \right) \rd s$.

	在直线$\overline {OA} $上$y = 0,\rd s = \rd x$,得$\displaystyle\int_{\overline {OA} } {\left( {{x^2} + {y^2}} \right)\rd x = \displaystyle\int_0^2 {{x^2}\rd x = \dfrac{8}{3};} } $

	在直线段$\overline {AB} $上$y =  - \dfrac{1}{2}x + 1,\rd s = \dfrac{{\sqrt 5 }}{2}\rd x$,

  得
	$\displaystyle\int_{\overline {AB} } {\left( {{x^2} + {y^2}} \right)\rd s = \displaystyle\int_2^0 {\left( {{x^2} + {{\left( { - \dfrac{1}{2}x + 1} \right)}^2}} \right) \cdot \dfrac{{\sqrt 5 }}{2}\rd x =  - \dfrac{{5\sqrt 5 }}{3};} } $

	在直线段$\overline {BO}$ 上$x = 0,\rd s = \rd y,$得$\displaystyle\int_{\overline {BO} } {\left( {{x^2} + {y^2}} \right)\rd x = \displaystyle\int_0^1 {{y^2} \rd y  = \dfrac{1}{3};} } $

	故$I = \dfrac{8}{3} + \dfrac{{5\sqrt 5 }}{3} + \dfrac{1}{3} = 3 + \dfrac{{5\sqrt 5 }}{3}.$

	(2)解:$x = a{\cos ^3}t,y = a{\sin ^3}t,0 \le t \le 2\pi $

  \begin{flalign*}\indent
   	\begin{split}
  		I&= \displaystyle\int_0^{2\pi } {\left( {{{\left( {{x^{\frac{2}{3}}} + {y^{\frac{2}{3}}}} \right)}^2} - 2{x^{\frac{2}{3}}}{y^{\frac{2}{3}}}} \right)} \rd s \\
  		&= {a^{\frac{4}{3}}} \cdot 4\displaystyle\int_0^{\frac{\pi }{2}} {3a\sin t\cos t \rd t - 8\displaystyle\int_0^{\frac{\pi }{2}} {\left( {{a^{\frac{2}{3}}}{{\cos }^2}t} \right)} } \left( {{a^{\frac{2}{3}}}{{\sin }^2}t} \right)\left( {3a\sin t\cos t} \right)\rd t\\
  		&= 12{a^{\frac{7}{3}}} \cdot \dfrac{1}{2} - 24{a^{\frac{7}{3}}} \cdot \dfrac{1}{{12}} = 4{a^{\frac{7}{3}}}.
  	\end{split}&
  \end{flalign*}

  (3)解:$x = a\cos t$ , $y = a\sin t$, $0 \le t \le \dfrac{\pi }{4}$

    弧段积分为:${I_1} = \displaystyle\int_0^{\pi /4} {{e^a}\sqrt {\left( {{{\left( {a\sin t} \right)}^2} + {{\left( {a\cos t} \right)}^2}} \right)} } \rd t = \dfrac{\pi }{4}a{e^a}$

    直线段为:${I_2} + {I_3} = \displaystyle\int_0^a {{e^x} \rd x + \displaystyle\int_{\frac{{\sqrt 2 }}{2}a}^0 {{e^{\sqrt 2 x}} \rd \left( {\sqrt 2 x} \right)} }  = 2\left( {{e^a} - 1} \right).$
      
    $\therefore I = 2\left( {{e^a} - 1} \right) + \dfrac{\pi }{4}a{e^a}.$
    
  \begin{flalign*}\indent
     \begin{split}
       (4)~\text{解:}I & ={\displaystyle\int_{C}{{y}^{2}} \rd s=\displaystyle\int_{0}^{2\pi }{\left( a( 1-\cos t \right))}^{2}}\sqrt{{{\left( a\left( 1-\cos t \right) \right)}^{2}}+{{\left( a\sin t \right)}^{2}}} \rd t \\
       & ={{a}^{3}}\displaystyle\int_{0}^{2\pi }{{( 1-\cos t )^{2}}\sqrt{{{\left( 1-\cos t \right)}^{2}}+{{\left( \sin t \right)}^{2}}} \rd t={{a}^{3}}\displaystyle\int_{0}^{2\pi }{4{{\sin }^{4}}\dfrac{t}{2}\cdot 2\sin \dfrac{t}{2} \rd t}}=\dfrac{256}{15}{{a}^{3}}.
    \end{split}&
  \end{flalign*}

  (5)解:$I = \displaystyle\int_C {xyz\rd s = \displaystyle\int_0^1 {\left( {t \cdot \dfrac{{2\sqrt 2 }}{3}{t^{\frac{3}{2}}} \cdot \dfrac{1}{2}{t^2}} \right)\sqrt {1 + 2t + {t^2}} \rd t} }  = \dfrac{{\sqrt 2 }}{3}\displaystyle\int_0^1 {{t^{\frac{9}{2}}}(t + 1)\rd t = \dfrac{{16\sqrt 2 }}{{143}}.} $

  4.解:由积分轮换性可知$\displaystyle\int {{x^2}\rd x = } \displaystyle\int {{y^2}\rd y = } \displaystyle\int {{z^2}\rd z} $

  $\therefore  I= \dfrac{1}{3}\displaystyle\int_C {\left( {{x^2} + {y^2} + {z^2}} \right)} \rd s = \dfrac{{{a^2}}}{3}\displaystyle\int_C {\rd s = \dfrac{{2\pi {a^3}}}{3}.} $

  \begin{flalign*}
    \begin{split}
      5.\text{解:} I & = \displaystyle\int_0^2 {x \cdot \sqrt {2x}  \cdot \sqrt {1 + y{'^2}\left( x \right)} \rd } x
      = \displaystyle\int_0^2 x \cdot \sqrt {2x}  \cdot \sqrt {\dfrac{{2x + 1}}{{2x}}} \rd x \\
      & = \displaystyle\int_0^2 {x\sqrt {2x + 1} \rd x}
      = \displaystyle\int_0^2 \left( {\dfrac{1}{3}{{\left( {2x + 1} \right)}^{\frac{3}{2}}}} \right) \rd x \\
      & = \dfrac{1}{3} x \left. {{{\left( {2x + 1} \right)}^{\frac{3}{2}}}} \right|_0^2 - \dfrac{1}{3}\displaystyle\int_0^2 {{{\left( {2x + 1} \right)}^{\frac{3}{2}}}\rd x
      = } \dfrac{{5\sqrt 5 }}{3} + \dfrac{1}{{15}}.
    \end{split}&
  \end{flalign*}

  6.解:$I = \displaystyle\int_{ - \infty }^0 {\rho \cos \theta  \cdot } \sqrt {{\rho ^2} + \rho {'^2}} \rd \theta  = {a^2}\sqrt {1 + {k^2}} \displaystyle\int_{ - \infty }^0 {{e^{2k\theta }}\cos \theta \rd \theta } $

  下面计算$\displaystyle\int_{-\infty }^0 {{e^{2k\theta }}} \cos \theta \rd \theta  = {e^{2k\theta }}\left. {\sin \theta } \right|_{ - \infty }^0 - 2k\displaystyle\int_{ - \infty }^0 {{e^{2k\theta }}\sin \theta \rd \theta  = 2k\displaystyle\int_{ - \infty }^0 {{e^{2k\theta }}\rd \cos \theta } } $

  $= 2k{e^{2k\theta }}\left. {\cos \theta } \right|_{ - \infty }^0 - 4{k^2}\displaystyle\int_{ - \infty }^0 {{e^{2k\theta }}} \cos \theta \rd \theta  = 2k - 4{k^2}\displaystyle\int_{ - \infty }^0 {{e^{2k\theta }}} \cos \theta \rd \theta $

  $\therefore \displaystyle\int_{ - \infty }^0 {{e^{2k\theta }}} \cos \theta \rd \theta  = \dfrac{{2k}}{{4{k^2} + 1}}$, $\therefore I = \displaystyle\dfrac{{2k{a^2}\sqrt {1 + {k^2}} }}{{4{k^2} + 1}}.$

  7.解:$I = \displaystyle\int\limits_L {\rho \rd s = \displaystyle\int_0^1 {t\sqrt {{{\left( a \right)}^2} + {{\left( {at} \right)}^2} + {{\left( {a{t^2}} \right)}^2}} } } \rd t$

  $= \displaystyle\dfrac{a}{2} \displaystyle\int_0^1 {\sqrt {1 + {t^2} + {t^4}} } \rd \left( {{t^2}} \right) = \dfrac{a}{2}\displaystyle\int_0^1 {\sqrt {1 + h + {h^2}} \rd h = } \dfrac{a}{{16}}\left( {6\sqrt 3  - 2 + 3\ln \dfrac{{3 + 2\sqrt 3 }}{3}} \right).$

\section{第一型曲面积分——对面积的曲面积分}
\begin{flushright}
  \color{zhanqing!80}
  \ding{43} 教材见271页
\end{flushright}

  1.(2)解:$\rd S = \sqrt {1 + z_x^2 + z_y^2} \rd x\rd y = \sqrt 3 \rd x\rd y$

  $\therefore I = \displaystyle\sqrt 3 \displaystyle\int_0^1 {\rd x\displaystyle\int_0^{1 - x} {\dfrac{1}{{{{\left( {1 + x + y} \right)}^2}}}\rd y = \left( {\sqrt 3  - 1} \right)\left( {\dfrac{{\sqrt 3 }}{2} + \ln 2} \right).} }$

  (4)解:$\rd S = \sqrt {EG - {F^2}} \rd\varphi  = \sqrt {1 + {r^2}} \rd\varphi $

  $\therefore I = {\pi ^2}\left( {a\sqrt {1 + {a^2}}  + \ln \left( {a + \sqrt {1 + {a^2}} } \right)} \right).$

  (6)解:$\displaystyle\iint\limits_\sum {xyzdS = \sqrt 3 } \rd y = \dfrac{{\sqrt 3 }}{{120}}.$

  (8)解:曲面$\Sigma$在$xOy$平面的投影区域$Dxy:{x^2} + {y^2} \le 2ax,z = 0$

   $\rd S = \sqrt {1 + z_x^2 + z_y^2} \rd x\rd y = \sqrt 2 \rd x\rd y$

   $I = \displaystyle\iint\limits_\sum  {\left( {xy + yz + zs} \right)\rd S = \iint\limits_{\rd xy} {\sqrt 2 \left( {xy - \left( {y + x} \right)\sqrt {{x^2} + {y^2}} } \right)\rd x\rd y}}$

   $= \sqrt 2 \displaystyle\int_{ - \frac{\pi }{2}}^{\frac{\pi }{2}} {\rd \theta \displaystyle\int_0^{2a\cos \theta } {{\rho ^2}\left( {\sin \theta \cos \theta  - \sin \theta  - \cos \theta } \right)} } \rho \rd \rho  =  - \dfrac{{64}}{{15}}\sqrt 2 {a^3}.$

  (10)解析:由于积分轮换性可知
    $\displaystyle\iint {{x^2}\rd S = \iint {{y^2}\rd S = \dfrac{1}{2}\iint {\left( {{x^2} + {y^2}} \right)\rd S}}}
    = \dfrac{1}{2}{a^2}\iint\limits_{\rd xy} {\rd S = \pi {a^3}h}$

  (12)解:$\rd S = \sqrt {1 + z_x^2 + z_y^2} \rd x\rd y = \dfrac{R}{{\sqrt {{R^2} - {x^2} - {y^2}} }}\rd x\rd y$

  2.解析:$\because \rd m = \rho \rd S,\therefore m = \iint\limits_\sum  {\rho \rd S}$

  $m = \dfrac{1}{2}\displaystyle\iint\limits_{\rd xy} {({x^2} + {y^2})\sqrt {1 + {x^2} + {y^2}} }\rd x\rd y = \dfrac{1}{2}\displaystyle\int_0^{2\pi } {\rd \theta } \displaystyle\int_0^1 {{\rho ^2}} \sqrt {1 + {\rho ^2}} \rho \rd \rho  = \dfrac{{2\pi }}{{15}}\left( {6\sqrt 3  + 1} \right)$

\section{第二型曲线积分——对坐标的曲线积分}
\begin{flushright}
  \color{zhanqing!80}
  \ding{43} 教材见281页
\end{flushright}

  1.(2)解析:$I = \displaystyle\int\limits_C {\left( {{x^2} - 2{x^3}} \right)\rd x + 2x\left( {{x^4} - 2{x^3}} \right)} \rd x = \displaystyle\int_{ - 1}^1 {\left( { - 4{x^4} + {x^2}} \right)\rd x = }  - \dfrac{{14}}{{15}}.$

  (4)解析:$x = y = z,I = \displaystyle\int_0^1 {3{x^2}dx = 1.}$

  (6)解析:$\displaystyle\oint_C {x{y^2}\rd x - {x^2}y\rd x = \displaystyle\int_0^{2\pi } {({R^4}\cos t{{\sin }^2}t\left( { - \sin t} \right) - } } {R^4}{\cos ^2}t\sin t\cos t)\rd t$

  $= {R^4}\displaystyle\int_0^{2\pi } {\sin t\cos t\rd t = 0} $

  (8)$\because y = \begin{cases}
    x, & 0 \le x \le 1 \\
  	2 - x, & 1 \le x \le 2 \\
  	\end{cases}$

  $\therefore I = \displaystyle\int_0^1 {2{x^2}\rd x + 0\rd x + \displaystyle\int_1^2 {\left( {{x^2} + {{\left( {2 - x} \right)}^2}} \right)} } \rd x + \left( {{x^2} - {{\left( {2 - x} \right)}^2}} \right)\left( { - \rd x} \right) = \dfrac{4}{3}$

  (10)解: $\displaystyle\oint_C {\dfrac{{\left( {x + y} \right)\rd x - \left( {x - y} \right)\rd y}}{{{x^2} + {y^2}}}}  = \displaystyle\int_0^{2\pi } {\left( {\cos \theta  + \sin \theta } \right)\rd \left( {\cos \theta } \right) - \left( {\cos \theta  - \sin \theta } \right)\rd \left( {\sin \theta } \right)} $

  $= \displaystyle\int_0^{2\pi } {\left( { - \cos \theta \sin \theta  - {{\sin }^2}\theta  - {{\cos }^2}\theta  + \sin \theta \cos \theta } \right)} \rd \theta  =  - 2\pi .$

  2.解析:观察可知$\left| x \right| + \left| y \right| = 1.$

  $\therefore \displaystyle\oint\limits_{ABCDA} {\dfrac{{dx +  \rd y }}{{\left| x \right| + \left| y \right|}} = \displaystyle\oint\limits_{ABCDA} {dx +  \rd y  = 0} } $(对称性)

  3.解析:$\displaystyle\oint\limits_C {\sqrt {{x^2} + {y^2} + 1} \rd x + y\left( {xy + \ln \left( {x + \sqrt {{x^2} + {y^2} + 1} } \right)} \right)} \rd y$

  原式$= \displaystyle\int_0^{2\pi } {a\sqrt {1 + {a^2}} d} \left( {\cos \theta } \right) + {a^2}\sin \theta \left( {{a^2}\cos \theta \sin \theta  + \ln \left( {a\cos \theta  + \sqrt {1 + {a^2}} } \right)} \right)d\left( {\sin \theta } \right)$

  根据对称性以及倍角公式可得:

  原式$= \displaystyle\int_0^{2\pi } {\left( { - a\sqrt {1 + {a^2}} \sin \theta  + {a^4}{{\cos }^2}\theta {{\sin }^2}\theta } \right)} \rd \theta  = \dfrac{{\pi {a^4}}}{4}.$

\section{格林公式及其应用}
\begin{flushright}
  \color{zhanqing!80}
  \ding{43} 教材见299页
\end{flushright}

  1. 解: (1)原式$ = \displaystyle\iint\limits _D {(2x - x)\rd x\rd y} = \iint\limits _D {x\rd x\rd y = 0}$.

  (3)原式=$ \displaystyle\iint\limits _D {( - \dfrac{1}{{{x^2}}}} + \dfrac{1}{{{y^2}}})\rd x\rd y = \displaystyle\int _1^4 {\rd x\displaystyle\int _1^{\sqrt x } {\left( { - \dfrac{1}{{{x^2}}} + \dfrac{1}{{{y^2}}}} \right)\rd y} }  = \dfrac{3}{4}$.

  (5)原式$  = \displaystyle\iint\limits _D {{e^x}}\left( { - y + \sin y - \sin y} \right)\rd x \! \rd y =  - \displaystyle\int _0^\pi  {\rd x} \displaystyle\int _0^{\sin x} {{e^x} \cdot y\rd x\rd y = \dfrac{1}{5}} \left( {1 - {e^\pi }} \right) $.

  2. 解: $\dfrac{{\partial P}}{{\partial y}} = \dfrac{1}{{{x^2}}} =    \dfrac{{\partial Q}}{{\partial x}}\left( {x \ne 0} \right)$只要积分路径不通过y轴,则该曲线积分与路径无关.

  $\therefore \displaystyle\int _{\left( {2,1} \right)}^{\left( {1,2} \right)} {\dfrac{y}{{{x^2}}}} \rd x - \dfrac{1}{{{x^2}}}\rd y = \displaystyle\int _{\left( {2,1} \right)}^{\left( {2,2} \right)} { - \dfrac{1}{2}} \rd y + \displaystyle\int _{\left( {2,2} \right)}^{\left( {1,2} \right)} {\dfrac{2}{{{x^2}}}} \rd x =  - \dfrac{3}{2}$.

  3. 解: $\because \dfrac{{\partial P}}{{\partial y}} =  - ax\sin y - 2y\sin x = \dfrac{{\partial Q}}{{\partial x}} =  - by - 2x\sin y,\therefore a = 2,b = 2$.

  $I = \displaystyle\int _{\left( {0,0} \right)}^{\left( {0,1} \right)} {2y\rd y + \displaystyle\int _{\left( {0,1} \right)}^{\left( {1,1} \right)} {\left( {2x\cos 1 - \sin x} \right)\rd x = 2\cos 1} } $.

  4. 解: $\because \dfrac{{\partial P}}{{\partial y}} = {e^x} + f\left( x \right) = \dfrac{{\partial Q}}{{\partial x}} = f'\left( x \right),\therefore {e^x} = f'\left( x \right) - f\left( x \right)  $.

  5. 解: 原式$ = \displaystyle\iint\limits _D {\left( {2x - 2y} \right)}\rd x\rd y = \displaystyle\int _0^1 {\rd x} \displaystyle\int _0^{\sqrt {1 - {x^2}} } {\left( {2x - 2y} \right)} \rd y =  - 1$

  6. 解: 原式$ = \displaystyle\iint\limits _D {\left( {2x{e^{2y}} - 1 - 2x{e^{2y}}} \right)}\rd x \! \rd y = \iint\limits _D {\rd x \! \rd y -  {\displaystyle\int _2^0 {x\rd x} } } -  {\displaystyle\int _2^0 {\left( {4{e^{2y}} - y} \right)\rd y} } = \pi  + 2{e^4} - 2$.

  7. 解: $ \because \dfrac{\partial P}{\partial y}=\dfrac{{{y}^{2}}-2xy-{{x}^{2}}}{\left( {{x}^{2}}+{{y}^{2}} \right)}=\dfrac{\partial Q}{\partial x},\therefore $只要不经过原点,积分就与路径无关

  故原积分$=\displaystyle\int_{-1}^{1}{\dfrac{x+1}{{{x}^{2}}+1}}\rd x=\displaystyle\int_{-1}^{1}{\dfrac{1}{{{x}^{2}}+1}\rd x}=\dfrac{\pi }{2}$.

  9. 解: 做小椭圆域${{x}^{2}}+4{{y}^{2}}\le {{r}^{2}}$使得椭圆区域包含在积分圆周内

  $\therefore \displaystyle\oint_{L}{\dfrac{x\rd y-y\rd x}{{{x}^{2}}+4{{y}^{2}}}=}\displaystyle\oint_{\Gamma }{\dfrac{x\rd y-y\rd x}{{{x}^{2}}+4{{y}^{2}}}=\displaystyle\int_{0}^{2\pi }{\dfrac{\frac{1}{2}r\cos \varphi r\cos \varphi -\frac{1}{2}r\sin \varphi \left( -r\sin \varphi  \right)}{{{r}^{2}}}}}=\pi $.

\section{第二型曲面积分——对坐标的曲面积分}
\begin{flushright}
  \color{zhanqing!80}
  \ding{43} 教材见308页
\end{flushright}

  1. 解: (1)$\displaystyle\iint\limits  \limits_{\Sigma} {\left( {{x}^{2}}+{{y}^{2}} \right)}\rd x\rd y=-\iint\limits _{\rd xy}{{{\rho }^{2}}\cdot \rho \rd \rho }\rd \theta =-2\pi \displaystyle\int_{0}^{R}{{{\rho }^{3}}\rd \rho =-\dfrac{\pi {{R}^{4}}}{2}.}$

  (3) 原式$=\displaystyle\iint\limits  \limits_{\Sigma} {x\rd y\rd z+y\rd z\rd x+\left( {{z}^{2}}-2z \right)\rd x\rd y}$

  注意到${{z}_{x}}=\dfrac{x}{\sqrt{{{x}^{2}}+{{y}^{2}}}},{{z}_{y}}=\dfrac{y}{\sqrt{{{x}^{2}}+{{y}^{2}}}}$

  $I=\displaystyle\iint \limits\limits \limits_{\Sigma} {\left( x,y,\left( {{z}^{2}}-2z \right) \right)\left( \dfrac{x}{\sqrt{{{x}^{2}}+{{y}^{2}}}},\dfrac{y}{\sqrt{{{x}^{2}}+{{y}^{2}}}},1 \right)}\rd x\rd y$

  $\therefore I=\displaystyle\iint\limits  \limits_{\Sigma} {\left( {{z}^{2}}-z \right)\rd x\rd y=\iint\limits _{ D_{xy} }{\rho \left( {{\rho }^{2}}-\rho  \right)}}\rd \rho \rd \theta =\dfrac{3}{2}\pi $.

  (7) 由轮换性可知$\displaystyle\iint\limits\limits  \limits_{\Sigma} {{{x}^{3}}\rd y\rd z=}\iint\limits  \limits_{\Sigma} {{{y}^{3}}\rd x\rd z=}\iint\limits  \limits_{\Sigma} {{{z}^{3}}\rd y\rd x}$

  $\therefore I=3\displaystyle\iint\limits _{ D_{xy} }{{{z}^{3}}\rd x\rd y=3\iint\limits _{ D_{xy} }{\rho {{\left( {{a}^{2}}-{{\rho }^{2}} \right)}^{\frac{3}{2}}}\rd \rho }}\rd \theta =\dfrac{12}{5}\pi {{a}^{5}}$.

  (9) $x=a\sin \theta \cos \varphi ,y=b\sin \theta \sin \varphi ,z=c\cos \varphi $

  \[\therefore \left[ \begin{matrix}
     \begin{matrix}
     x{{'}_{\varphi }} & y{{'}_{\varphi }} & z{{'}_{\varphi }}  \\
  \end{matrix}  \\
     \begin{matrix}
     x{{'}_{\theta }} & y{{'}_{\theta }} & z{{'}_{\theta }}  \\
  \end{matrix}  \\
  \end{matrix} \right]=\left[ \begin{matrix}
     \begin{matrix}
     a\cos \theta \cos \varphi   \\
     -a\sin \theta \sin \varphi   \\
  \end{matrix} & \begin{matrix}
     b\sin \theta \sin \varphi   \\
     b\cos \theta \sin \varphi   \\
  \end{matrix} & \begin{matrix}
     -c\sin \varphi   \\
     0  \\
  \end{matrix}  \\
  \end{matrix} \right]\]

  $\therefore A=bc\cos \theta {{\sin }^{2}}\varphi ,B=ac\sin \theta {{\cos }^{2}}\varphi ,C=ab\sin \theta \sin \varphi $.

  $\because \displaystyle\iint\limits  \limits_{\Sigma} {P\rd y \rd z +Q\rd x\rd z+R\rd x\rd y=\iint\limits\limits  \limits_{\Sigma} {\left( PA+QB+RC \right)}}\rd \varphi \rd \theta $

  $\therefore I=\displaystyle\iint\limits _{ D_{xy} }{\left( \dfrac{bc}{a}\sin \varphi +\dfrac{ac}{b}\sin \varphi +\dfrac{ab}{c}\sin \varphi  \right)}\rd \varphi \rd \theta =\dfrac{4\pi }{abc}\left( {{a}^{2}}{{b}^{2}}+{{c}^{2}}{{b}^{2}}+{{a}^{2}}{{c}^{2}} \right)$.

  2. 解: (1)$\because {{z}_{x}}=-\dfrac{\sqrt{3}}{2},{{z}_{y}}=-\dfrac{1}{\sqrt{3}},\sqrt{1+z_{x}^{2}+z_{y}^{2}}=\dfrac{5}{2\sqrt{3}}$

  $\therefore \cos \alpha =\dfrac{3}{5},\cos \beta =\dfrac{2}{5},\cos \gamma =\dfrac{2\sqrt{3}}{5}$

  $I=\displaystyle\iint\limits  \limits_{\Sigma} {\left( \dfrac{3}{5}P+\dfrac{2}{5}Q+\dfrac{2\sqrt{3}}{5}R \right)\rd S}$.

  (2) $\because {{z}_{x}}=-2x,{{z}_{y}}=-2y,\sqrt{1+z_{x}^{2}+z_{y}^{2}}=\sqrt{1+4{{x}^{2}}+4{{y}^{2}}}$

  $\therefore \cos \alpha =\dfrac{-2x}{\sqrt{1+4{{x}^{2}}+4{{y}^{2}}}},\cos \beta =\dfrac{-2y}{\sqrt{1+4{{x}^{2}}+4{{y}^{2}}}},\cos \gamma =\dfrac{1}{\sqrt{1+4{{x}^{2}}+4{{y}^{2}}}}$

  $\therefore I=\displaystyle\iint\limits  \limits_{\Sigma} {\dfrac{2xP+2yQ+R}{\sqrt{1+4{{x}^{2}}+4{{y}^{2}}}}}\rd S $.

\section{高斯公式与斯托克斯公式}
\begin{flushright}
  \color{zhanqing!80}
  \ding{43} 教材见319页
\end{flushright}

  1. 解: (1)原式$=\displaystyle\iiint\limits\limits _{\Omega }{\left( {{z}^{2}}+{{x}^{2}}+{{y}^{2}} \right)\rd v\text{+}\iint\limits _{ D_{xy} }{2xy\rd x\rd y}=\dfrac{2\pi {{a}^{5}}}{5}}.$

  (2) 原式$=-\displaystyle\iiint\limits _{\Omega }{3 \rd v =-2\pi {{R}^{3}}}$.

  (3) 原式$=\displaystyle\iiint\limits _{\Omega }{-3\rd x\rd y\rd z+\iint\limits_{z=1,\rd xy}{\left( {{x}^{2}}-1 \right)\rd x\rd y}=}-\dfrac{15}{4}\pi $.

  (4) (i)原式$=\displaystyle\iiint\limits _{\Omega }{\left( 3{{x}^{2}}-2{{x}^{2}}+1 \right)dv=\iiint\limits _{\Omega }{\left( {{x}^{2}}+1 \right)\rd x\rd y\rd z=\dfrac{1}{3}{{a}^{5}}+{{a}^{3}}}}$.

  (ii) 补上平面$z=0,z=1$形成封闭曲面

  $I=\displaystyle\iiint\limits _{\Omega }{\left( {{x}^{2}}+1 \right)\rd x\rd y\rd z+\iint\limits _{z=1, D_{xy} }{\rd x\rd y}=}\dfrac{1}{4}\pi {{R}^{4}}+\pi {{R}^{2}}$.

  (5) 补上$\sum_1:z=0$形成闭曲面

  $\therefore \displaystyle\iiint\limits _{\Omega }{\left( 2x+2y+2 \right)\rd x\rd y\rd z=2\displaystyle\int_{0}^{2\pi }{\rd \theta \displaystyle\int_{0}^{1}{\rd \rho \displaystyle\int_{0}^{1-{{\rho }^{2}}}{\rho \left( \rho \left( \cos \theta +\sin \theta  \right)+1 \right)\rd z}}}}=\dfrac{2\pi }{3}$.

  (6) 原式$=\displaystyle\iiint\limits _{\Omega} {\left( 2x+2y+2z \right)}\rd x\rd y\rd z=3\times 2\iiint\limits _{\Omega} {x\rd x\rd y\rd z=}3{{a}^{4}}$.

  \begin{flalign*}
    \begin{split}
      (7) \displaystyle\iiint\limits _{\Omega }{\left( 2x+2y \right)}\rd v
      & = 2\iiint\limits _{\Omega }{\left( \left( x-a \right)+a \right)}\rd v+2\iiint\limits\limits _{\Omega }{\left( \left( y-b \right)+b \right)}\rd v\\
            & = 2\left( a+b \right)\iiint\limits _{\Omega }{\rd x\rd y\rd z}=\dfrac{8}{3}\pi {{R}^{3}}\left( a+b \right).
        \end{split}&
  \end{flalign*}

  2. 解: (1)$=\displaystyle\iint\limits _{ D_{xy} }{-3{{x}^{2}}{{y}^{2}}\rd x\rd y=}-3\displaystyle\int_{0}^{2\pi }{{{a}^{6}}{{\sin }^{2}}\theta {{\cos }^{2}}\theta \rd \theta }=-\dfrac{\pi }{8}{{a}^{6}}$.

  (2) 原式$=\displaystyle\iint\limits  \limits_{\Sigma} {\left( {{z}^{2}}-x \right)\rd y\rd z-\left( z+3 \right)\rd x\rd y=-\iiint\limits _{\Omega }{2\rd x\rd y\rd z}}$

  $=-2\displaystyle\int_{0}^{2\pi }{\rd \theta }\displaystyle\int_{0}^{\sqrt{2}}{\rho \rd \rho }\displaystyle\int_{\frac{{{\rho }^{2}}}{2}}^{2}{\rd z}=-20\pi $.

  (4) 原式$=\displaystyle\iint\limits  \limits_{\Sigma} {\left( 2y-2z \right)\rd y\rd z+\left( 2z-2x \right)\rd x\rd z+\left( 2x-2y \right)\rd x\rd y}$

  $I=\displaystyle\iint\limits  \limits_{\Sigma} {\left( \left( y-z \right)\dfrac{x-R}{R}+\left( z-x \right)\dfrac{y}{R}+\left( x-y \right)\dfrac{z}{R} \right)\rd S=2\iint\limits  \limits_{\Sigma} {\left( z-y \right)\rd S}}$

  $\because \rd S=\rd x\rd y\times \dfrac{R}{z},\rd S=\rd x\rd z\times \dfrac{R}{y}\therefore z\rd S=R\rd x\rd y,y\rd S=R\rd x\rd z$

  $\therefore I=2R\displaystyle\iint\limits\limits\limits  \limits_{\Sigma} {\rd x\rd y}-2R\iint  \limits_{\Sigma} {\rd x\rd z}=2R\pi {{r}^{2}}$.

  (6) 原式$=-\displaystyle\iint\limits  \limits_{\Sigma} {\rd x\rd y+\rd x\rd z+\rd y\rd z=-3\iint\limits_{ D_{xy} }{\rd x\rd y=-\sqrt{3}\pi {{a}^{2}}}}$.

  3. 解: $\sum_1:z={{e}^{a}},{{x}^{2}}+{{y}^{2}}\le {{a}^{2}}$

  $\therefore I=\displaystyle\iint\limits _{\sum +\sum_1}{4xz\rd y\rd z-2yz\rd x\rd z+\left( 1-{{z}^{2}} \right)\rd x\rd y-\iint\limits _{\sum_1}{4xz\rd y\rd z-2yz\rd x\rd z+\left( 1-{{z}^{2}} \right)\rd x\rd y}}$

  $=\displaystyle\iiint\limits _{\Omega }{\left( 4z-2z-2z \right)\rd x\rd y\rd z}-\iint _{\rd xy}{\left( 1-{{e}^{2a}} \right)\rd x\rd y=\left( {{e}^{2a}}-1 \right)\pi {{a}^{2}}}$.

\section{场论初步}
\begin{flushright}
  \color{zhanqing!80}
  \ding{43} 教材见328页
\end{flushright}

  1. 解: (1) $F=\left( {{P}_{1}},{{Q}_{1}},{{R}_{1}} \right),G=\left( {{P}_{2}},{{Q}_{2}},{{R}_{2}} \right)$

  $\therefore \text{div}  \left( F+G \right)=\dfrac{\partial \left( {{P}_{1}}+{{P}_{2}} \right)}{\partial x}+\dfrac{\partial \left( {{Q}_{1}}+{{Q}_{2}} \right)}{\partial y}+\dfrac{\partial \left( {{R}_{1}}+{{R}_{2}} \right)}{\partial z}$

  原式$=\dfrac{\partial {{P}_{1}}}{\partial x}+\dfrac{\partial {{Q}_{1}}}{\partial y}+\dfrac{\partial {{R}_{1}}}{\partial z}+\dfrac{\partial {{P}_{2}}}{\partial x}+\dfrac{\partial {{Q}_{2}}}{\partial y}+\dfrac{\partial {{R}_{2}}}{\partial z}=\text{div}  F+\text{div}  G$.

  (2) $\text{div}  \left( UF \right)=\dfrac{\partial \left( UF \right)}{\partial x}+\dfrac{\partial \left( UF \right)}{\partial y}+\dfrac{\partial \left( UF \right)}{\partial z}$

  原式$=U\left( \dfrac{\partial P}{\partial x}+\dfrac{\partial Q}{\partial y}+\dfrac{\partial R}{\partial z} \right)+F\left( \dfrac{\partial U}{\partial x}+\dfrac{\partial U}{\partial y}+\dfrac{\partial U}{\partial z} \right)=U\text{div}  F+F\text{div}  U$.
  \begin{flalign*}\indent
    \begin{split}
   \textbf{(3)rot}  \left( F+G \right) & = \begin{vmatrix}
        i & j & k  \\
       \dfrac{\partial }{\partial x} & \dfrac{\partial }{\partial y} & \dfrac{\partial }{\partial z}  \\
        {{f}_{x}}+{{g}_{x}} & {{f}_{y}}+{{g}_{y}} & {{f}_{z}}+{{g}_{z}}  \\
  \end{vmatrix} \\
   \textbf{rot}  F+ \textbf{rot}  G & = \left| \begin{matrix}
     i & j & k  \\
     \dfrac{\partial }{\partial x} & \dfrac{\partial }{\partial y} & \dfrac{\partial }{\partial z}  \\
     {{f}_{x}} & {{f}_{y}} & {{f}_{z}}  \\
  \end{matrix} \right|+\left| \begin{matrix}
     i & j & k  \\
     \dfrac{\partial }{\partial x} & \dfrac{\partial }{\partial y} & \dfrac{\partial }{\partial z}  \\
     {{g}_{x}} & {{g}_{y}} & {{g}_{z}}  \\
  \end{matrix} \right|= \textbf{rot}  \left( F+G \right).
    \end{split}&
  \end{flalign*}

  \((4) \textbf{rot}  \left( UF \right)=\left| \begin{matrix}
     i & j & k  \\
     \dfrac{\partial }{\partial x} & \dfrac{\partial }{\partial y} & \dfrac{\partial }{\partial z}  \\
     UF & UF & UF \\
  \end{matrix} \right|=U \textbf{rot}  F+ \textbf{grad} U\times F\)

  2. 解: (1)$\text{div}  \left(  \textbf{gradU} \right)=\nabla \cdot \left( \nabla U \right)={{\nabla }^{2}}U=\Delta U$.

  (2) $\text{div}  \left( U \textbf{gradU} \right)=U\text{div}  \left(  \textbf{gradU} \right)+ \textbf{gradU}\cdot  \textbf{gradU}=U\Delta U+ \textbf{gradU}\cdot  \textbf{gradU}$.

  (3) $ \textbf{rot}  \left(  \textbf{gradU} \right)=\nabla \times \left( \nabla U \right)=\left( 0,0,0 \right)$.

  (4) $\text{div}  \left(  \textbf{rot}  F \right)=\nabla \cdot \left( \nabla \times A \right)=\left( \nabla \times \nabla  \right)\cdot F=0$.

  \begin{flalign*}
    \begin{split}
      \text{3. 解:}\displaystyle \text{div}  F
      & = \lim_{\Delta \tau \to 0}{\mathop{\lim }},\dfrac{1}{\Delta \tau }\iint\limits _{\sigma }{F\cdot n\rd S}\\
      & = \lim_{\Delta \tau \to 0}{\mathop{\lim }}\,\dfrac{1}{\Delta \tau }\iint\limits _{\sigma }{\left( P\cos \alpha +Q\cos \beta +R\cos \gamma  \right)\rd S}\\
      & = \lim_{\Delta \tau \to 0}{\mathop{\lim }}\,\dfrac{1}{\Delta \tau }\iint\limits _{\sigma }{P\rd y \rd z +Q\rd x \rd z +R\rd x\rd y}
      = 0
      \end{split}&
  \end{flalign*}

  $\therefore \text{div}  F=0\Leftarrow \dfrac{\partial P}{\partial x}+\dfrac{\partial Q}{\partial y}+\dfrac{\partial R}{\partial z}=0$.(反推可证充分性)

\section*{总习题九}
\addcontentsline{toc}{section}{总习题九}
\begin{flushright}
  \color{zhanqing!80}
  \ding{43} 教材见339页
\end{flushright}

  1. 解: (1) 原式$=\displaystyle\int_{C}{\sqrt{2{{y}^{2}}+{{z}^{2}}}\rd s=\displaystyle\int_{C}{\sqrt{{{x}^{2}}+{{y}^{2}}+{{z}^{2}}}\rd s}=2\pi {{a}^{2}}}$.

  (3) 原式$=\displaystyle\int_{3}^{8}{\dfrac{2}{3}x\sqrt{1+x}\rd x=\dfrac{2152}{45}}$.

  (5) 补上x轴使曲线封闭,而后使用格林公式

  原式$=-\displaystyle\iint\limits _{D}{\left( 2x+1 \right)}\rd x\rd y=-\displaystyle\int_{0}^{2}{\rd x}\displaystyle\int_{0}^{2x-{{x}^{2}}}{\left( 2x+1 \right)}\rd y=-4$.

  (7) $\displaystyle\int_{L}{\left( 2a-a\left( 1-\cos t \right)a\left( 1-\cos t \right)+a\left( t-\sin t \right)a\sin t \right) \rd t}$

  原式$=\displaystyle\int_{0}^{2\pi }{{{a}^{2}}t\sin t \rd t=-2\pi {{a}^{2}}}$.

  (9) $\because y=z,\therefore {{x}^{2}}+2{{y}^{2}}=1\to x=\cos t,y=\dfrac{1}{\sqrt{2}}\sin t$

  $\displaystyle\int_{L}{xyz\rd x=\displaystyle\int_{0}^{2\pi }{\cos t\cdot \dfrac{1}{2}{{\sin }^{2}}t\cdot \dfrac{1}{\sqrt{2}}\cos t \rd t=\dfrac{\sqrt{2}\pi }{16}}}$.

  2. 解: $m=\displaystyle\int_{0}^{2\pi }{\sqrt{{{x}^{2}}+{{y}^{2}}+{{z}^{2}}}\rd s=\displaystyle\int_{0}^{2\pi }{\sqrt{1+{{t}^{2}}}\sqrt{2} \rd t=\sqrt{2}\left( \pi \sqrt{1+4{{\pi }^{2}}}+\dfrac{1}{2}\ln \left( 2\pi +\sqrt{1+4{{\pi }^{2}}} \right) \right)}}$.

  4. 解: (1) 原式$=\displaystyle\iint\limits _{D}{\left( \dfrac{\partial Q}{\partial x}-\dfrac{\partial P}{\partial y} \right)\rd x\rd y}=\iint\limits _{D}{{{y}^{2}}\rd x\rd y}=\displaystyle\int_{1}^{4}{\rd x}\displaystyle\int_{0}^{2}{{{y}^{2}}\rd y}=8$.

  (2) 原式$=\displaystyle\iint\limits _{D}{\left( \dfrac{\partial Q}{\partial x}-\dfrac{\partial P}{\partial y} \right)\rd x\rd y}=\iint\limits _{D}{{{y}^{2}}\rd x\rd y}=\displaystyle\int_{0}^{a}{{{\rho }^{2}}\cdot \rho \rd \rho }\displaystyle\int_{0}^{2\pi }{{{\cos }^{2}}\theta }=\dfrac{{{a}^{4}}}{4}\pi $.

  6. 解: $I=-2\displaystyle\iint\limits  \limits_{\Sigma} {\rd x \rd z +\rd x\rd y+\rd y \rd z }=-2\sigma \left( \dfrac{h}{\sqrt{{{a}^{2}}+{{h}^{2}}}}+\dfrac{a}{\sqrt{{{a}^{2}}+{{h}^{2}}}} \right)=-2\pi a\left( a+h \right)$.

  7. 解: (1)补充平面$\sum_1 z=0,\sum_2 z=h$构成封闭曲面
    \begin{flalign*}
      \begin{split}
        \text{原式}&=\iint\limits _{\Omega }{0\rd x\rd y \rd z }-\iint\limits _{\sum_1}{\left( {{x}^{2}}-y \right)}\rd x\rd y-\iint\limits _{\sum 2}{\left( {{x}^{2}}-y \right)\rd x\rd y} \\
        &=-\displaystyle\int_{0}^{2\pi }{\displaystyle\int_{0}^{h}{\rho \left( {{\rho }^{2}}{{\cos }^{2}}\theta -\rho \sin \theta  \right)\rd \rho \rd \theta }=-\dfrac{\pi }{4}{{h}^{4}}}.
      \end{split}&
    \end{flalign*}

  (3) 取$\varepsilon >0$充分小,${{S}_{\varepsilon }}:{{x}^{2}}+{{y}^{2}}+{{z}^{2}}={{\varepsilon }^{2}}\left( z\ge 0 \right),{{D}_{\varepsilon }}:{{x}^{2}}+{{y}^{2}}={{\varepsilon }^{2}}\left( z=0 \right)$

  $\therefore I = \displaystyle\iint\limits _{{{S}_{\varepsilon }}+{{D}_{\varepsilon }}}{\dfrac{x\rd y \rd z +y \rd z \rd x+z\rd x\rd y}{\sqrt{{{\left( {{x}^{2}}+{{y}^{2}}+{{z}^{2}} \right)}^{3}}}}}=\dfrac{1}{{{\varepsilon }^{3}}}\iint _{{{S}_{\varepsilon }}}{\dfrac{1}{\varepsilon }\left( {{x}^{2}}+{{y}^{2}}+{{z}^{2}} \right)}\rd s=2\pi $.

  (5) 补平面$\sum_1:z=0$构成封闭曲面
    \begin{flalign*}
      \begin{split}
        \text{原式} & =\dfrac{1}{a}\iint\limits_{\sum +\sum_1-\sum_1}{ax}\rd y\rd z+{\left( z+a \right)^{2}}\rd x\rd y=-\iiint\limits _{\Omega }{{\left[ a+2\left( z+a \right] \right)}\rd x\rd y\rd z}+\iint\limits _{ D_{xy} }{{{a}^{2}}\rd x\rd y} \\
        & = -3a\iiint\limits _{\Omega }{\rd x\rd y\rd z}-2\iiint\limits _{\Omega }{z\rd x\rd y\rd z}+{{a}^{2}}\iint\limits _{ D_{xy} }{\rd x\rd y}=-\dfrac{1}{2}\pi {{a}^{3}}.
      \end{split}&
    \end{flalign*}

  8. 解: (1) 原式$=-\displaystyle\iint\limits  \limits_{\Sigma} {\rd y\rd z+\rd x\rd z+\rd x\rd y}=-\iint\limits _{D}{\left( 1,1,1 \right)\left( 1,1,1 \right)\rd x\rd y=}-\iint \limits_{D}{\rd x\rd y}$

  $D_{xy}:{{x}^{2}}+{{y}^{2}}+{{\left( x+y-1 \right)}^{2}}=1$

  $\therefore I=-\dfrac{2\sqrt{3}}{3}\pi $.